\documentclass{article}

% SYMBOLS, MATH
\usepackage{amsmath}
\usepackage[version=4]{mhchem}
\usepackage{textgreek}
\usepackage{mathtools}
\usepackage{cancel}
\usepackage{empheq}
\newcommand*\widefbox[1]{\fbox{\vspace{0.5em}\hspace{2em}#1\hspace{2em}\vspace{0.5em}}}

% FIGURES, TABLES, LISTS
\usepackage{graphicx}
\usepackage{booktabs}
\usepackage{enumitem}

% FONT
\usepackage[utf8]{inputenc}
\usepackage{sectsty}
\allsectionsfont{\sffamily}
\renewcommand{\contentsname}{{\sffamily Table of Contents}}

% TOC STYLE
\usepackage{tocloft}
\renewcommand{\cftsecleader}{\cftdotfill{\cftdotsep}}
\renewcommand{\cftsecfont}{\normalfont}
\renewcommand{\cftsecpagefont}{\normalfont}

% BIBLIOGRAPHY
%\usepackage{biblatex}
%\bibliography{}
\usepackage{hyperref}

\title{Chapter 3 Problems}
\author{Jian Carlo M. Guevara}
\date{~}

% UNITS
\usepackage{siunitx}
\DeclareSIUnit\bar{bar}
\DeclareSIUnit\feet{ft}
\DeclareSIUnit\atmosphere{atm}

% SOLUTION ENVIRONMENT
\newenvironment{solution}{\par\noindent\textbf{\\Solution:\\}}{\par\medskip}


\begin{document}

\maketitle
\vspace{-3em}
\tableofcontents

\section*{Problem 3.1}
\addcontentsline{toc}{section}{Problem 3.1}
How many phase rule variables must be specified to fix the
thermodynamic state of each of the following systems?
\begin{enumerate}[label=(\alph*)]
  \item A sealed flask containing a liquid ethanol-water mixture in
    equilibrium with its vapor.
  \item A sealed flask containing a liquid ethanol-water mixture in
    equilibrium with its vapor and nitrogen.
  \item A sealed flask containing ethanol, toluene, and water as two
    liquid phases plus vapor.
\end{enumerate}

\begin{solution}
  Using the phase rule, $F=2-\pi+N$.
  \begin{enumerate}[label=(\alph*)]
    \item $N=2$ and $\pi=2$ so $\boxed{ F=2 }$
    \item $N=3$ and $\pi=2$ so $\boxed{ F=3 }$
    \item $N=3$ and $\pi=3$ so $\boxed{ F=2 }$
  \end{enumerate}
\end{solution}

\section*{Problem 3.2}
\addcontentsline{toc}{section}{Problem 3.2}
A renowned laboratory reports quadruple-point coordinates of
$10.2~\unit{\mega\bar}$ and $24.1~\unit{\degreeCelsius}$ for
four-phase equilibrium of allotropic solid forms of the exotic
chemical $\beta$-miasmone. Evaluate the claim.

\begin{solution}
  \begin{equation*}
    F=2-\pi+N
  \end{equation*}
  If $N=1$ and $\pi=4$, the phase rule suggests that $F=-1$. The
  claim is impossible unless there are other components present in the system.
\end{solution}

\section*{Problem 3.3}
\addcontentsline{toc}{section}{Problem 3.3}
A closed, nonreactive system contains species 1 and 2 in vapor/liquid
equilibrium. Species 2 is a very light gas, essentially insoluble in
the liquid phase. The vapor phase contains both species 1 and 2. Some
additional moles of species 2 are added to the system, which is then
restored to its initial temperature \( T \) and pressure \( P \). As
a result of the process, does the total number of moles of liquid
increase, decrease, or remain unchanged?

\begin{solution}
  \begin{equation*}
    N=2, \qquad \pi=2 \qquad \to \qquad F=2.
  \end{equation*}
  With constant temperature and pressure, the system's equilibrium is
  constrained. Introducing species 2 increases $y_2$ and decreases
  $y_1$. Therefore, to maintain the original vapor equilibrium,
  liquid species 1 must evaporate, resulting in a decrease in the
  total moles of liquid.
\end{solution}


\section*{Problem 3.4}
\addcontentsline{toc}{section}{Problem 3.4}
A system comprised of chloroform, 1,4-dioxane, and ethanol exists as
a two-phase vapor/liquid system at $50~\unit{ \degreeCelsius }$ and
$55~\unit{ \kilo\pascal }$. After the addition of some pure ethanol,
the system can be returned to two-phase equilibrium at the initial T
and P. In what respect has the system changed, and in what respect
has it not changed?

\begin{solution}
  \begin{equation*}
    N=3, \qquad \pi=2 \qquad \to \qquad F=3
  \end{equation*}
  The addition of ethanol increases the $y_{EtOH}$ and
  $x_{\text{EtOH}}$ of the system thus decreasing $\left(
  y_{\text{EtOH}} \right)'$ and $\left( x_{EtOH} \right)'$. Since
  only two \textit{independent} intensive variables were changed, the
  number of phases will remain unchanged.
\end{solution}

\section*{Problem 3.5}
\addcontentsline{toc}{section}{Problem 3.5}
For the system described in Prob. 3.4:
\begin{enumerate}
  \item[(a)] How many phase-rule variables in addition to $T$ and $P$
    must be chosen so as to fix the compositions of both phases?
  \item[(b)] If the temperature and pressure are to remain the same,
    can the overall composition of the system be changed (by adding
    or removing material) without affecting the compositions of the
    liquid and vapor phases?
\end{enumerate}

\begin{solution}
  \begin{enumerate}[label=(\alph*)]
    \item 3 degrees of freedom.
    \item One more intensive variable (mole fraction of one
      component) must be fixed in order to fix the intensive state of
      the system.
  \end{enumerate}
\end{solution}


\section*{Problem 3.6}
\addcontentsline{toc}{section}{Problem 3.6}
Express the volume expansivity and the isothermal compressibility as
functions of density $\rho$ and its partial derivatives. For water at
$50~\unit{ \degreeCelsius }$ and $1~\unit{ \bar }$, $\kappa = 44.18
\times 10^{-6}$ $\unit{ \bar }$. To what pressure must water be
compressed at $50~\unit{ \degreeCelsius }$ to change its density by
1\%? Assume that $\kappa$ is independent of P.

\begin{solution}
  \begin{gather*}
    \kappa\equiv\frac{1}{V}\left( \frac{\partial V}{\partial T} \right)_{P} \\
    \int \kappa \, dP = \int -\rho \, d\left( \frac{1}{\rho} \right) \\
  \end{gather*}
  let $u=1/\rho$ \quad $\to$ \quad  $\rho\,d\left( 1/\rho
  \right)=1/u\,d\left( u \right)$:
  \begin{gather*}
    \kappa\Delta P = -\ln\left( \frac{u_{2}}{u_{1}} \right)=
    \ln\left( \frac{\rho_{2}}{\rho_{1}} \right) \\
    \left( 44.18\times10^{-6} \right)\left( P_{2}-1 \right)=
    \ln\left( 1.01 \right) \\
    \boxed{ P_{2}=226~\unit{ \bar } }
  \end{gather*}
\end{solution}

\section*{Problem 3.7}
\addcontentsline{toc}{section}{Problem 3.7}
Generally, volume expansivity $\beta$ and isothermal compressibility
$\kappa$ depend on T and P. Prove that:
$$ \left(\frac{\partial \beta}{\partial P}\right)_T =
-\left(\frac{\partial \kappa}{\partial T}\right)_P $$

\begin{solution}
  \begin{gather*}
    \beta \equiv \frac{1}{V}\left( \frac{\partial V}{\partial T}
    \right)_{P}, \qquad \kappa \equiv \frac{-1}{V}\left(
    \frac{\partial V}{\partial P} \right)_{T} \\
    dV = \frac{1}{d\beta}\left( \frac{\partial V}{\partial T}
    \right)_{P} = \frac{-1}{d\kappa}\left( \frac{\partial V}{\partial
    P} \right)_{T} \\
    -\left( \frac{\partial \kappa}{\partial T} \right)_{P}=\left(
    \frac{\partial \beta}{\partial P} \right)_{T}
  \end{gather*}
\end{solution}

\section*{Problem 3.8}
\addcontentsline{toc}{section}{Problem 3.8}
The Tait equation for liquids is written for an isotherm as:
$$ V = V_0\left(1 - \frac{AP}{B+P}\right) $$
where V is molar or specific volume, $V_0$ is the hypothetical molar
or specific volume at zero pressure, and A and B are positive
constants. Find an expression for the isothermal compressibility
consistent with this equation.

\begin{solution}
  \begin{align*}
    \kappa &\equiv \frac{-1}{V}\left( \frac{\partial V}{\partial P}
    \right)_{T} \\
    &=\frac{-1}{V}\frac{d}{dP}\left[ V_{0}\left( 1-\frac{AP}{B+P}
    \right) \right]
    \intertext{after differentiation:}
    &=\frac{-1}{V}\left[ V_{0}A\left( B+P \right)^{-1}-V_{0}\left(
    AP \right)\left( B+P \right)^{-2} \right]
    \intertext{simplification:}
    \Aboxed{ $\kappa & = \dfrac{-V_{0}}{V}\left[ \dfrac{A}{B+P} -
    \dfrac{AP}{\left( B+P \right)^{2}} \right]$ }
  \end{align*}
\end{solution}

\section*{Problem 3.9}
\addcontentsline{toc}{section}{Problem 3.9}
For liquid water the isothermal compressibility is given by:
$$ \kappa = \frac{c}{V(P+b)} $$
where c and b are functions of temperature only. If $1~\unit{
\kilo\gram }$ of water is compressed isothermally and reversibly from
$1$ to $500~\unit{ \bar }$ at $60~\unit{ \degreeCelsius }$, how much
work is required? At $60~\unit{ \degreeCelsius }$, $b=2700~\unit{
\bar }$ and $c=0.125~\unit{ \centi\meter\cubed\per\gram }$.

\begin{solution}
  \begin{gather*}
    W = -\int_{V_{1}}^{V_{2}} P \, dV \\
    \kappa \equiv \frac{-1}{V}\left( \frac{\partial V}{\partial P}
    \right)_{T} \\
    dV = -\kappa V \, dP \\
    dV = -\frac{c}{P+b}\,dP \\
  \end{gather*}
  \vspace{-3em}
  \begin{align*}
    W & = \frac{c}{3}\left( 500^{3}-1 \right)~\unit{
    \bar\centi\meter\cubed\per\gram }\times 1000\unit{ \gram } \times
    10^{2}\unit{ \kilo\pascal\per\bar }\times 0.01^{3}\unit{
    \meter\cubed\per\centi\meter\cubed } \\
    \Aboxed{ W & = 1562500\unit{ \kilo\joule } }
  \end{align*}
\end{solution}

\section*{Problem 3.10}
\addcontentsline{toc}{section}{Problem 3.10}
Calculate the reversible work done in compressing 1 ft$^3$ of mercury
at a constant temperature of 32°F from 1(atm) to 3000(atm). The
isothermal compressibility of mercury at 32°F is:
$$ \kappa(\text{atm})^{-1} = 3.9 \times 10^{-6} - 0.1 \times 10^{-9}
P(\text{atm}) $$

\begin{solution}
  \begin{gather*}
    W = -\int_{V_{1}}^{V_{2}} P \, dV \\
    \kappa = \frac{-1}{V}\left( \frac{\partial V}{\partial P} \right)_{T} \\
    dV = -\kappa V\,dP
  \end{gather*}
  \begin{align*}
    W &= -\int_{P_{1}}^{P_{2}} -\left(
    3.9\times10^{-6}-0.1\times10^{-9}P \right)V \, dP \\
    W &= 0.0112461~\unit{ \atmosphere\feet\cubed } \times
    101325~\unit{ \pascal\per\atmosphere }\times 3.28^{-3}~\unit{
    \meter\cubed\per\feet\cubed } \\
    \Aboxed{ W &= 32.29~\unit{ \joule } }
  \end{align*}
\end{solution}

\section*{Problem 3.11}
\addcontentsline{toc}{section}{Problem 3.11}
Five kilograms of liquid carbon tetrachloride undergo a mechanically
reversible, isobaric change at $1~\unit{ \bar }$ during which the
temperature changes from $0~\unit{ \degreeCelsius }$ to $20~\unit{
\degreeCelsius }$. Determine $\Delta V$, $W$, $Q$, $\Delta H$, and
$\Delta U$. The properties for liquid carbon tetrachloride at
$1~\unit{ \bar }$ and $0~\unit{ \degreeCelsius }$ may be assumed
independent of temperature: $\beta = 1.2 \times 10^{-3}~\unit{
\per\kelvin }$, $C_P = 0.84~\unit{
\kilo\joule\per\kilo\gram\per\kelvin }$, and $\rho = 1590~\unit{
\kilo\gram\per\meter\cubed }$.

\begin{solution}
  \begin{enumerate}[label=(\alph*)]
    \item for $\Delta V$
      \begin{gather*}
        \beta \equiv \frac{1}{V}\left( \frac{\partial V}{\partial T}
        \right)_{P} \\
        \Delta V = \frac{\beta }{\rho }\times\Delta T \\
        \boxed{ \Delta V = 1.509\times10^{-5}~\unit{
        \meter\cubed\per\kilo\gram } } \\
      \end{gather*}
    \item for $W$
      \begin{gather*}
        W = -P\Delta V \\
        W = -100000\left( 1.509\times10^{-5} \right) \\
        \boxed{ W = -1.509~\unit{ \joule\per\kilo\gram } }
      \end{gather*}
    \item for $Q$ and $\Delta H$
      \begin{align*}
        Q = \Delta H &= C_{P}\Delta T \\
        &= 0.84\times\left( 20 \right) \\
        \Aboxed{ Q = \Delta H &= 16.8~\unit{ \kilo\joule\per\kilo\gram } }
      \end{align*}
    \item for $\Delta U$
      \begin{gather*}
        \Delta U = Q + W \\
        \Delta U = 16800 - 1.509 \\
        \boxed{ \Delta U \approx 16.8~\unit{ \kilo\joule\per\kilo\gram } }
      \end{gather*}
  \end{enumerate}
\end{solution}

\section*{Problem 3.12}
\addcontentsline{toc}{section}{Problem 3.12}
Various species of hagfish, or slime eels, live on the ocean floor,
where they burrow inside other fish, eating them from the inside out
and secreting copious amounts of slime. Their skins are widely used
to make eelskin wallets and accessories. Suppose a hagfish is caught
in a trap at a depth of $200~\unit{ \meter }$ below the ocean
surface, where the water temperature is $10~\unit{ \degreeCelsius }$,
then brought to the surface where the temperature is $15~\unit{
\degreeCelsius }$. If the isothermal compressibility and volume
expansivity are assumed constant and equal to the values for water,
$$ (\beta = 10^{-4} \text{K}^{-1} \text{ and } \kappa = 4.8 \times
10^{-5} \text{ bar}^{-1}) $$
what is the fractional change in the volume of the hagfish when it is
brought to the surface?

\begin{solution}
  \begin{gather*}
    \frac{dV}{V}=\beta dT-\kappa dP\\
    \frac{\Delta V}{V}=\beta \Delta T-\kappa \Delta P\\
    \frac{\Delta V}{V}=10^{-4}\left( 10-15
    \right)-4.8\times10^{-5}\left( 1000\cdot9.81\cdot\left( 0-200
    \right)\cdot10^{-5} \right)\\
    \boxed{ \frac{\Delta V}{V}=4.4176\times10^{-4}~\text{volume
    change per unit volume of the hagfish.} }
  \end{gather*}
  It will expand.
\end{solution}

\section*{Table 3.2: Volumetric Properties of Liquids at 20$^\circ$C}
\addtocontents{toc}{}
Table 3.2 provides the specific volume, isothermal compressibility,
and volume expansivity of several liquids at 20$^\circ$C and 1
bar$^{25}$ for use in Problems 3.13 to 3.15, where $\beta$ and
$\kappa$ may be assumed constant.


\begin{table}[h]
  \begin{tabular}{@{}llccc@{}}
    \toprule
    Molecular       & Chemical Name      & Specific            &
    Isothermal                    & Volume
    \\ \midrule
    Formula         &                    & Volume              &
    Compressibility               & Expansivity                        \\
    &                    & V/L$\cdot$kg$^{-1}$ & $\kappa$/10$^{-5}$
    bar$^{-1}$ & $\beta$/10$^{-3}$ $^\circ$C$^{-1}$ \\ \midrule
    C$_2$H$_4$O$_2$ & Acetic Acid        & 0.951               & 9.08
    & 1.08                               \\
    C$_6$H$_7$N     & Aniline            & 0.976               & 4.53
    & 0.81                               \\
    CS$_2$          & Carbon Disulfide   & 0.792               & 9.38
    & 1.12                               \\
    C$_6$H$_5$Cl    & Chlorobenzene      & 0.904               & 7.45
    & 0.94                               \\
    C$_6$H$_{12}$   & Cyclohexane        & 1.285               & 11.3
    & 1.15                               \\
    C$_4$H$_{10}$O  & Diethyl Ether      & 1.401               &
    18.65                         & 1.65                               \\
    C$_2$H$_6$O     & Ethanol            & 1.265               &
    11.19                         & 1.40                               \\
    C$_4$H$_8$O$_2$ & Ethyl Acetate      & 1.110               &
    11.32                         & 1.35                               \\
    C$_8$H$_{10}$   & $m$-Xylene         & 1.157               & 8.46
    & 0.99                               \\
    CH$_4$O         & Methanol           & 1.262               &
    12.14                         & 1.49                               \\
    CCl$_4$         & Tetrachloromethane & 0.628               & 10.5
    & 1.14                               \\
    C$_7$H$_8$      & Toluene            & 1.154               & 8.96
    & 1.05                               \\
    CHCl$_3$        & Trichloromethane   & 0.672               & 9.96
    & 1.21                               \\ \bottomrule
  \end{tabular}
\end{table}

\pagebreak

\section*{Problem 3.13}
\addcontentsline{toc}{section}{Problem 3.13}
For one of the substances in Table 3.2, compute the change in volume
and work done when one kilogram of the substance is heated from
15$^\circ$C to 25$^\circ$C at a constant pressure of 1 bar.

\begin{solution}
  \begin{gather*}
    \Delta V=V\left( \beta \Delta T-\kappa \cancel{\Delta P}   \right)\\
    W=-P\Delta V
  \end{gather*}
  \vspace{-1em}
  \begin{empheq}[box=\widefbox]{gather*}
    \Delta V =
    \begin{bmatrix}
      1.03e-05 \\
      7.91e-06 \\
      8.87e-06 \\
      8.50e-06 \\
      1.48e-05 \\
      2.31e-05 \\
      1.77e-05 \\
      1.50e-05 \\
      1.15e-05 \\
      1.88e-05 \\
      7.16e-06 \\
      1.21e-05 \\
      8.13e-06 \\
    \end{bmatrix}
    \si{\meter\cubed} \qquad
    W =
    \begin{bmatrix}
      -1.03 \\
      -0.79 \\
      -0.89 \\
      -0.85 \\
      -1.48 \\
      -2.31 \\
      -1.77 \\
      -1.50 \\
      -1.15 \\
      -1.88 \\
      -0.72 \\
      -1.21 \\
      -0.81 \\
    \end{bmatrix}
    \si{\joule}
  \end{empheq}
\end{solution}

\section*{Problem 3.14}
\addcontentsline{toc}{section}{Problem 3.14}
For one of the substances in Table 3.2, compute the change in volume
and work done when one kilogram of the substance is compressed from 1
bar to 100 bar at a constant temperature of 20$^\circ$C.

\begin{solution}
  \begin{gather*}
    dV = -V\kappa dP \\
    W = \int_{V_{1}}^{V_{2}} -P \, dV \\
    W = \frac{V\kappa }{2}\left( P_{2}^{2}-P_{1}^{2} \right)
  \end{gather*}
  \begin{empheq}[box=\widefbox]{gather*}
    \Delta V =
    \begin{bmatrix}
      -8.55e-06 \\
      -4.38e-06 \\
      -7.35e-06 \\
      -6.67e-06 \\
      -1.44e-05 \\
      -2.59e-05 \\
      -1.40e-05 \\
      -1.24e-05 \\
      -9.69e-06 \\
      -1.52e-05 \\
      -6.53e-06 \\
      -1.02e-05 \\
      -6.63e-06
    \end{bmatrix} \, \si{\meter\cubed} \qquad
    W =
    \begin{bmatrix}
      4.32e-04 \\
      2.21e-04 \\
      3.71e-04 \\
      3.37e-04 \\
      7.26e-04 \\
      1.31e-03 \\
      7.08e-04 \\
      6.28e-04 \\
      4.89e-04 \\
      7.66e-04 \\
      3.30e-04 \\
      5.17e-04 \\
      3.35e-04
    \end{bmatrix} \, \si{\joule}
  \end{empheq}
\end{solution}

\section*{Problem 3.15}
\addcontentsline{toc}{section}{Problem 3.15}
For one of the substances in Table 3.2, compute the final pressure
when the substance is heated from 15$^\circ$C and 1 bar to
25$^\circ$C at constant volume.

\begin{solution}
  \begin{gather*}
    P_{2}=P_{1}+\frac{\beta \Delta T}{\kappa }
  \end{gather*}
  \begin{empheq}[box=\widefbox]{gather*}
    P_{2} =
    \begin{bmatrix}
      2.19 \\
      2.79 \\
      2.19 \\
      2.26 \\
      2.02 \\
      1.88 \\
      2.25 \\
      2.19 \\
      2.17 \\
      2.23 \\
      2.09 \\
      2.17 \\
      2.21 \\
    \end{bmatrix}
    \si{\bar}
  \end{empheq}
\end{solution}

\section*{Problem 3.16}
\addcontentsline{toc}{section}{Problem 3.16}
A substance for which $\kappa$ is a constant undergoes an isothermal,
mechanically reversible process from initial state ($P_1, V_1$) to
final state ($P_2, V_2$), where $V$ is molar volume.
\begin{enumerate}
  \item[(a)] Starting with the definition of $\kappa$, show that the
    path of the process is described by:
    $$V = A(T)\exp(-\kappa P)$$
  \item[(b)] Determine an exact expression which gives the isothermal
    work done on 1 mol of this constant-$\kappa$ substance.
\end{enumerate}

\begin{solution}
  \begin{enumerate}[label=(\alph*)]
    \item
      \begin{gather*}
        \kappa \equiv -\frac{1}{V}\left( \frac{\partial V}{\partial P}
        \right)_{T} \\
        \kappa dP = \frac{-dV}{V} \\
        \kappa P = -\ln V + A \\
        \boxed{ V = A\left( T \right)e^{-\kappa P} }
      \end{gather*}
    \item Derivation of the previous answer to solve for $P$ in
      $W=-PdV$, then solving for the integral:
      \begin{gather*}
        P = \frac{1}{\kappa }\ln\left( \frac{A\left( T \right)}{V} \right) \\
        W = -\int_{V_{1}}^{V_{2}} \frac{1}{\kappa }\ln\left(
        \frac{A\left( T \right)}{V} \right) \, dV \\
        W = \frac{1}{\kappa }\left[ V\ln\left( \frac{V}{A\left( T
          \right)} \right)-\int \frac{1}{A\left( T \right)} \, dV
        \right]_{V_{1}}^{V_{2}}\\
        \boxed{ W = \Delta V \ln \frac{V_{2}}{V_{1}} - \frac{\Delta
        V}{A\left( T \right)} }
      \end{gather*}
  \end{enumerate}
\end{solution}

\section*{Problem 3.17}
\addcontentsline{toc}{section}{Problem 3.17}
One mole of an ideal gas with $C_P = \frac{7}{2}R$ and $C_V =
\frac{5}{2}R$ expands from $P_1 = \SI{8}{\bar}$ and $T_1 =
\SI{600}{\kelvin}$ to $P_2 = \SI{1}{\bar}$ by each of the following paths:
\begin{enumerate}[label=(\alph*)]
  \item Constant volume;
  \item Constant temperature;
  \item Adiabatically.
\end{enumerate}
Assuming mechanical reversibility, calculate $W$, $Q$, $\Delta U$,
and $\Delta H$ for each process. Sketch each path on a single $PV$ diagram.

\begin{solution}
  \begin{enumerate}[label=(\alph*)]
    \item
      \begin{gather*}
        Q=\Delta U=C_{V}\Delta T\\
        \Delta H=\Delta U+\Delta \left( PV \right)\\
        \frac{P_{1}}{P_{2}}=\frac{T_{1}}{T_{2}}  \to  T_{2}=75~\unit{
        \kelvin }\\
        V = \frac{RT}{P}  \to  V = 6.235854\times10^{-3}~\unit{ \meter\cubed }\\
        \fbox{%
          \parbox{\widthof{$Q = \Delta U = \SI{-10.9}{\kilo\joule}$}}{%
            \centering
            $W = 0$ \\
            $Q = \Delta U = \SI{-10.9}{\kilo\joule}$ \\
            $\Delta H = \SI{-15.3}{\kilo\joule}$
          }
        }
      \end{gather*}
    \item
      \begin{gather*}
        W=-RT\ln\frac{V_{2}}{V_{1}}=-RT\ln\frac{P_{1}}{P_{2}} \\
        W=-Q
      \end{gather*}
      \begin{empheq}[box=\widefbox]{gather*}
        \Delta H = \Delta U = 0 \\
        W = -10.4~\unit{ \kilo\joule }\\
        Q = 10.4~\unit{ \kilo\joule }
      \end{empheq}
    \item
      % FORMULAE FOR ADIABATIC EXPANSION
  \end{enumerate}
\end{solution}

\section*{Problem 3.18}
\addcontentsline{toc}{section}{Problem 3.18}
One mole of an ideal gas with $C_P = \frac{5}{2}R$ and $C_V =
\frac{3}{2}R$ expands from $P_1 = \SI{6}{\bar}$ and $T_1 =
\SI{800}{\kelvin}$ to $P_2 = \SI{1}{\bar}$ by each of the following paths:
\begin{enumerate}[label=(\alph*)]
  \item Constant volume;
  \item Constant temperature;
  \item Adiabatically.
\end{enumerate}
Assuming mechanical reversibility, calculate $W$, $Q$, $\Delta U$,
and $\Delta H$ for each process. Sketch each path on a single $PV$ diagram.

\section*{Problem 3.19}
\addcontentsline{toc}{section}{Problem 3.19}
An ideal gas initially at $\SI{600}{\kelvin}$ and $\SI{10}{\bar}$
undergoes a four-step mechanically reversible cycle in a closed
system. In step $12$, pressure decreases isothermally to
$\SI{3}{\bar}$; in step $23$, pressure decreases at constant volume
to $\SI{2}{\bar}$; in step $34$, volume decreases at constant
pressure; and in step $41$, the gas returns adiabatically to its
initial state. Take $C_P = \frac{7}{2}R$ and $C_V = \frac{5}{2}R$.
\begin{enumerate}[label=(\alph*)]
  \item Sketch the cycle on a $PV$ diagram.
  \item Determine (where unknown) both $T$ and $P$ for states $1$, $2$,
    $3$, and $4$.
  \item Calculate $Q$, $W$, $\Delta U$, and $\Delta H$ for each step
    of the cycle.
\end{enumerate}

\section*{Problem 3.20}
\addcontentsline{toc}{section}{Problem 3.20}
An ideal gas initially at $\SI{300}{\kelvin}$ and $\SI{1}{\bar}$
undergoes a three-step mechanically reversible cycle in a closed
system. In step $12$, pressure increases isothermally to
$\SI{5}{\bar}$; in step $23$, pressure increases at constant volume;
and in step $31$, the gas returns adiabatically to its initial state.
Take $C_P = \frac{7}{2}R$ and $C_V = \frac{5}{2}R$.
\begin{enumerate}[label=(\alph*)]
  \item Sketch the cycle on a $PV$ diagram.
  \item Determine (where unknown) $V$, $T$, and $P$ for states $1$,
    $2$, and $3$.
  \item Calculate $Q$, $W$, $\Delta U$, and $\Delta H$ for each step
    of the cycle.
\end{enumerate}

\section*{Problem 3.21}
\addcontentsline{toc}{section}{Problem 3.21}
The state of an ideal gas with $C_P = \frac{5}{2}R$ is changed from
$P_1 = \SI{1}{\bar}$ and $V_1 = \SI{12}{\meter\cubed}$ to $P_2 =
\SI{12}{\bar}$ and $V_2 = \SI{1}{\meter\cubed}$ by the following
mechanically reversible processes:
\begin{enumerate}[label=(\alph*)]
  \item Isothermal compression.
  \item Adiabatic compression followed by cooling at constant pressure.
  \item Adiabatic compression followed by cooling at constant volume.
  \item Heating at constant volume followed by cooling at constant pressure.
  \item Cooling at constant pressure followed by heating at constant volume.
\end{enumerate}
Calculate $Q$, $W$, $\Delta U$, and $\Delta H$ for all processes, and
sketch the paths of all processes on a single $PV$ diagram.


\end{document}
