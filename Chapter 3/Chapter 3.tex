\documentclass{article}

% SYMBOLS, MATH
\usepackage{amsmath}
\usepackage[version=4]{mhchem}
\usepackage{textgreek}
\usepackage{mathtools}
\usepackage{cancel}
\usepackage{empheq}
\newcommand*\widefbox[1]{\fbox{\vspace{0.5em}\hspace{2em}#1\hspace{2em}\vspace{0.5em}}}

% FIGURES, TABLES, LISTS
\usepackage{graphicx}
\usepackage{pgf}
\usepackage{booktabs}
\usepackage{enumitem}

% FONT
\usepackage[utf8]{inputenc}
\usepackage{sectsty}
\usepackage{lmodern}
\allsectionsfont{\sffamily}
\renewcommand{\contentsname}{{\sffamily Table of Contents}}

% TOC STYLE
\usepackage{tocloft}
\renewcommand{\cftsecleader}{\cftdotfill{\cftdotsep}}
\renewcommand{\cftsecfont}{\normalfont}
\renewcommand{\cftsecpagefont}{\normalfont}

% BIBLIOGRAPHY
%\usepackage{biblatex}
%\bibliography{}
\usepackage{hyperref}

% UNITS
\usepackage{siunitx}
\DeclareSIUnit\bar{bar}
\DeclareSIUnit\feet{ft}
\DeclareSIUnit\atmosphere{atm}

% SOLUTION ENVIRONMENT
\newenvironment{solution}{\par\noindent\textbf{\\Solution:\\}}{\par\medskip}

\title{Chapter 3 Problems}
\author{Jian Carlo M. Guevara}
\date{~}

\begin{document}

\maketitle
\vspace{-3em}
\tableofcontents

\section*{Problem 3.1}
\addcontentsline{toc}{section}{Problem 3.1}
How many phase rule variables must be specified to fix the
thermodynamic state of each of the following systems?
\begin{enumerate}[label=(\alph*)]
  \item A sealed flask containing a liquid ethanol-water mixture in
    equilibrium with its vapor.
  \item A sealed flask containing a liquid ethanol-water mixture in
    equilibrium with its vapor and nitrogen.
  \item A sealed flask containing ethanol, toluene, and water as two
    liquid phases plus vapor.
\end{enumerate}

\begin{solution}
  Using the phase rule, $F=2-\pi+N$.
  \begin{enumerate}[label=(\alph*)]
    \item $N=2$ and $\pi=2$ so $\boxed{ F=2 }$
    \item $N=3$ and $\pi=2$ so $\boxed{ F=3 }$
    \item $N=3$ and $\pi=3$ so $\boxed{ F=2 }$
  \end{enumerate}
\end{solution}

\section*{Problem 3.2}
\addcontentsline{toc}{section}{Problem 3.2}
A renowned laboratory reports quadruple-point coordinates of
$10.2~\unit{\mega\bar}$ and $24.1~\unit{\degreeCelsius}$ for
four-phase equilibrium of allotropic solid forms of the exotic
chemical $\beta$-miasmone. Evaluate the claim.

\begin{solution}
  \begin{equation*}
    F=2-\pi+N
  \end{equation*}
  If $N=1$ and $\pi=4$, the phase rule suggests that $F=-1$. The
  claim is impossible unless there are other components present in the system.
\end{solution}

\section*{Problem 3.3}
\addcontentsline{toc}{section}{Problem 3.3}
A closed, nonreactive system contains species 1 and 2 in vapor/liquid
equilibrium. Species 2 is a very light gas, essentially insoluble in
the liquid phase. The vapor phase contains both species 1 and 2. Some
additional moles of species 2 are added to the system, which is then
restored to its initial temperature \( T \) and pressure \( P \). As
a result of the process, does the total number of moles of liquid
increase, decrease, or remain unchanged?

\begin{solution}
  \begin{equation*}
    N=2, \qquad \pi=2 \qquad \to \qquad F=2.
  \end{equation*}
  With constant temperature and pressure, the system's equilibrium is
  constrained. Introducing species 2 increases $y_2$ and decreases
  $y_1$. Therefore, to maintain the original vapor equilibrium,
  liquid species 1 must evaporate, resulting in a decrease in the
  total moles of liquid.
\end{solution}

\section*{Problem 3.4}
\addcontentsline{toc}{section}{Problem 3.4}
A system comprised of chloroform, 1,4-dioxane, and ethanol exists as
a two-phase vapor/liquid system at $50~\unit{ \degreeCelsius }$ and
$55~\unit{ \kilo\pascal }$. After the addition of some pure ethanol,
the system can be returned to two-phase equilibrium at the initial T
and P. In what respect has the system changed, and in what respect
has it not changed?

\begin{solution}
  \begin{equation*}
    N=3, \qquad \pi=2 \qquad \to \qquad F=3
  \end{equation*}
  The addition of ethanol increases the $y_{EtOH}$ and
  $x_{\text{EtOH}}$ of the system thus decreasing $\left(
  y_{\text{EtOH}} \right)'$ and $\left( x_{EtOH} \right)'$. Since
  only two \textit{independent} intensive variables were changed, the
  number of phases will remain unchanged.
\end{solution}

\section*{Problem 3.5}
\addcontentsline{toc}{section}{Problem 3.5}
For the system described in Prob. 3.4:
\begin{enumerate}
  \item[(a)] How many phase-rule variables in addition to $T$ and $P$
    must be chosen so as to fix the compositions of both phases?
  \item[(b)] If the temperature and pressure are to remain the same,
    can the overall composition of the system be changed (by adding
    or removing material) without affecting the compositions of the
    liquid and vapor phases?
\end{enumerate}

\begin{solution}
  \begin{enumerate}[label=(\alph*)]
    \item 3 degrees of freedom.
    \item One more intensive variable (mole fraction of one
      component) must be fixed in order to fix the intensive state of
      the system.
  \end{enumerate}
\end{solution}

\section*{Problem 3.6}
\addcontentsline{toc}{section}{Problem 3.6}
Express the volume expansivity and the isothermal compressibility as
functions of density $\rho$ and its partial derivatives. For water at
$50~\unit{ \degreeCelsius }$ and $1~\unit{ \bar }$, $\kappa = 44.18
\times 10^{-6}$ $\unit{ \bar }$. To what pressure must water be
compressed at $50~\unit{ \degreeCelsius }$ to change its density by
1\%? Assume that $\kappa$ is independent of P.

\begin{solution}
  \begin{gather*}
    \kappa\equiv\frac{1}{V}\left( \frac{\partial V}{\partial T}
    \right)_{P} \\
    \int \kappa \, dP = \int -\rho \, d\left( \frac{1}{\rho} \right) \\
  \end{gather*}
  let $u=1/\rho$ \quad $\to$ \quad  $\rho\,d\left( 1/\rho
  \right)=1/u\,d\left( u \right)$:
  \begin{gather*}
    \kappa\Delta P = -\ln\left( \frac{u_{2}}{u_{1}} \right)=
    \ln\left( \frac{\rho_{2}}{\rho_{1}} \right) \\
    \left( 44.18\times10^{-6} \right)\left( P_{2}-1 \right)=
    \ln\left( 1.01 \right) \\
    \boxed{ P_{2}=226~\unit{ \bar } }
  \end{gather*}
\end{solution}

\section*{Problem 3.7}
\addcontentsline{toc}{section}{Problem 3.7}
Generally, volume expansivity $\beta$ and isothermal compressibility
$\kappa$ depend on T and P. Prove that:
$$ \left(\frac{\partial \beta}{\partial P}\right)_T =
-\left(\frac{\partial \kappa}{\partial T}\right)_P $$

\begin{solution}
  \begin{gather*}
    \beta \equiv \frac{1}{V}\left( \frac{\partial V}{\partial T}
    \right)_{P}, \qquad \kappa \equiv \frac{-1}{V}\left(
    \frac{\partial V}{\partial P} \right)_{T} \\
    dV = \frac{1}{d\beta}\left( \frac{\partial V}{\partial T}
    \right)_{P} = \frac{-1}{d\kappa}\left( \frac{\partial V}{\partial
    P} \right)_{T} \\
    -\left( \frac{\partial \kappa}{\partial T} \right)_{P}=\left(
    \frac{\partial \beta}{\partial P} \right)_{T}
  \end{gather*}
\end{solution}

\section*{Problem 3.8}
\addcontentsline{toc}{section}{Problem 3.8}
The Tait equation for liquids is written for an isotherm as:
$$ V = V_0\left(1 - \frac{AP}{B+P}\right) $$
where V is molar or specific volume, $V_0$ is the hypothetical molar
or specific volume at zero pressure, and A and B are positive
constants. Find an expression for the isothermal compressibility
consistent with this equation.

\begin{solution}
  \begin{align*}
    \kappa &\equiv \frac{-1}{V}\left( \frac{\partial V}{\partial P}
    \right)_{T} \\
    &=\frac{-1}{V}\frac{d}{dP}\left[ V_{0}\left( 1-\frac{AP}{B+P}
    \right) \right]
    \intertext{after differentiation:}
    &=\frac{-1}{V}\left[ V_{0}A\left( B+P \right)^{-1}-V_{0}\left(
    AP \right)\left( B+P \right)^{-2} \right]
    \intertext{simplification:}
    \Aboxed{ $\kappa & = \dfrac{-V_{0}}{V}\left[ \dfrac{A}{B+P} -
    \dfrac{AP}{\left( B+P \right)^{2}} \right]$ }
  \end{align*}
\end{solution}

\section*{Problem 3.9}
\addcontentsline{toc}{section}{Problem 3.9}
For liquid water the isothermal compressibility is given by:
$$ \kappa = \frac{c}{V(P+b)} $$
where c and b are functions of temperature only. If $1~\unit{
\kilo\gram }$ of water is compressed isothermally and reversibly from
$1$ to $500~\unit{ \bar }$ at $60~\unit{ \degreeCelsius }$, how much
work is required? At $60~\unit{ \degreeCelsius }$, $b=2700~\unit{
\bar }$ and $c=0.125~\unit{ \centi\meter\cubed\per\gram }$.

\begin{solution}
  \begin{gather*}
    W = -\int_{V_{1}}^{V_{2}} P \, dV \\
    \kappa \equiv \frac{-1}{V}\left( \frac{\partial V}{\partial P}
    \right)_{T} \\
    dV = -\kappa V \, dP \\
    dV = -\frac{c}{P+b}\,dP \\
  \end{gather*}
  \vspace{-3em}
  \begin{align*}
    W & = \frac{c}{3}\left( 500^{3}-1 \right)~\unit{
    \bar\centi\meter\cubed\per\gram }\times 1000\unit{ \gram } \times
    10^{2}\unit{ \kilo\pascal\per\bar }\times 0.01^{3}\unit{
    \meter\cubed\per\centi\meter\cubed } \\
    \Aboxed{ W & = 1562500\unit{ \kilo\joule } }
  \end{align*}
\end{solution}

\section*{Problem 3.10}
\addcontentsline{toc}{section}{Problem 3.10}
Calculate the reversible work done in compressing 1 ft$^3$ of mercury
at a constant temperature of 32°F from 1(atm) to 3000(atm). The
isothermal compressibility of mercury at 32°F is:
$$ \kappa(\text{atm})^{-1} = 3.9 \times 10^{-6} - 0.1 \times 10^{-9}
P(\text{atm}) $$

\begin{solution}
  \begin{gather*}
    W = -\int_{V_{1}}^{V_{2}} P \, dV \\
    \kappa = \frac{-1}{V}\left( \frac{\partial V}{\partial P} \right)_{T} \\
    dV = -\kappa V\,dP
  \end{gather*}
  \begin{align*}
    W &= -\int_{P_{1}}^{P_{2}} -\left(
    3.9\times10^{-6}-0.1\times10^{-9}P \right)V \, dP \\
    W &= 0.0112461~\unit{ \atmosphere\feet\cubed } \times
    101325~\unit{ \pascal\per\atmosphere }\times 3.28^{-3}~\unit{
    \meter\cubed\per\feet\cubed } \\
    \Aboxed{ W &= 32.29~\unit{ \joule } }
  \end{align*}
\end{solution}

\section*{Problem 3.11}
\addcontentsline{toc}{section}{Problem 3.11}
Five kilograms of liquid carbon tetrachloride undergo a mechanically
reversible, isobaric change at $1~\unit{ \bar }$ during which the
temperature changes from $0~\unit{ \degreeCelsius }$ to $20~\unit{
\degreeCelsius }$. Determine $\Delta V$, $W$, $Q$, $\Delta H$, and
$\Delta U$. The properties for liquid carbon tetrachloride at
$1~\unit{ \bar }$ and $0~\unit{ \degreeCelsius }$ may be assumed
independent of temperature: $\beta = 1.2 \times 10^{-3}~\unit{
\per\kelvin }$, $C_P = 0.84~\unit{
\kilo\joule\per\kilo\gram\per\kelvin }$, and $\rho = 1590~\unit{
\kilo\gram\per\meter\cubed }$.

\begin{solution}
  \begin{enumerate}[label=(\alph*)]
    \item for $\Delta V$
      \begin{gather*}
        \beta \equiv \frac{1}{V}\left( \frac{\partial V}{\partial T}
        \right)_{P} \\
        \Delta V = \frac{\beta }{\rho }\times\Delta T \\
        \boxed{ \Delta V = 1.509\times10^{-5}~\unit{
        \meter\cubed\per\kilo\gram } } \\
      \end{gather*}
    \item for $W$
      \begin{gather*}
        W = -P\Delta V \\
        W = -100000\left( 1.509\times10^{-5} \right) \\
        \boxed{ W = -1.509~\unit{ \joule\per\kilo\gram } }
      \end{gather*}
    \item for $Q$ and $\Delta H$
      \begin{align*}
        Q = \Delta H &= C_{P}\Delta T \\
        &= 0.84\times\left( 20 \right) \\
        \Aboxed{ Q = \Delta H &= 16.8~\unit{
        \kilo\joule\per\kilo\gram } }
      \end{align*}
    \item for $\Delta U$
      \begin{gather*}
        \Delta U = Q + W \\
        \Delta U = 16800 - 1.509 \\
        \boxed{ \Delta U \approx 16.8~\unit{
        \kilo\joule\per\kilo\gram } }
      \end{gather*}
  \end{enumerate}
\end{solution}

\section*{Problem 3.12}
\addcontentsline{toc}{section}{Problem 3.12}
Various species of hagfish, or slime eels, live on the ocean floor,
where they burrow inside other fish, eating them from the inside out
and secreting copious amounts of slime. Their skins are widely used
to make eelskin wallets and accessories. Suppose a hagfish is caught
in a trap at a depth of $200~\unit{ \meter }$ below the ocean
surface, where the water temperature is $10~\unit{ \degreeCelsius }$,
then brought to the surface where the temperature is $15~\unit{
\degreeCelsius }$. If the isothermal compressibility and volume
expansivity are assumed constant and equal to the values for water,
$$ (\beta = 10^{-4} \text{K}^{-1} \text{ and } \kappa = 4.8 \times
10^{-5} \text{ bar}^{-1}) $$
what is the fractional change in the volume of the hagfish when it is
brought to the surface?

\begin{solution}
  \begin{gather*}
    \frac{dV}{V}=\beta dT-\kappa dP\\
    \frac{\Delta V}{V}=\beta \Delta T-\kappa \Delta P\\
    \frac{\Delta V}{V}=10^{-4}\left( 10-15
    \right)-4.8\times10^{-5}\left( 1000\cdot9.81\cdot\left( 0-200
    \right)\cdot10^{-5} \right)\\
    \boxed{ \frac{\Delta V}{V}=4.4176\times10^{-4}~\text{volume
    change per unit volume of the hagfish.} }
  \end{gather*}
  It will expand.
\end{solution}

\section*{Table 3.2: Volumetric Properties of Liquids at 20$^\circ$C}
\addtocontents{toc}{}
Table 3.2 provides the specific volume, isothermal compressibility,
and volume expansivity of several liquids at 20$^\circ$C and 1
bar$^{25}$ for use in Problems 3.13 to 3.15, where $\beta$ and
$\kappa$ may be assumed constant.

\begin{table}[h]
  \begin{tabular}{@{}llccc@{}}
    \toprule
    Molecular       & Chemical Name      & Specific            &
    Isothermal                    & Volume
    \\ \midrule
    Formula         &                    & Volume              &
    Compressibility               & Expansivity                        \\
    &                    & V/L$\cdot$kg$^{-1}$ & $\kappa$/10$^{-5}$
    bar$^{-1}$ & $\beta$/10$^{-3}$ $^\circ$C$^{-1}$ \\ \midrule
    C$_2$H$_4$O$_2$ & Acetic Acid        & 0.951               & 9.08
    & 1.08                               \\
    C$_6$H$_7$N     & Aniline            & 0.976               & 4.53
    & 0.81                               \\
    CS$_2$          & Carbon Disulfide   & 0.792               & 9.38
    & 1.12                               \\
    C$_6$H$_5$Cl    & Chlorobenzene      & 0.904               & 7.45
    & 0.94                               \\
    C$_6$H$_{12}$   & Cyclohexane        & 1.285               & 11.3
    & 1.15                               \\
    C$_4$H$_{10}$O  & Diethyl Ether      & 1.401               &
    18.65                         & 1.65                               \\
    C$_2$H$_6$O     & Ethanol            & 1.265               &
    11.19                         & 1.40                               \\
    C$_4$H$_8$O$_2$ & Ethyl Acetate      & 1.110               &
    11.32                         & 1.35                               \\
    C$_8$H$_{10}$   & $m$-Xylene         & 1.157               & 8.46
    & 0.99                               \\
    CH$_4$O         & Methanol           & 1.262               &
    12.14                         & 1.49                               \\
    CCl$_4$         & Tetrachloromethane & 0.628               & 10.5
    & 1.14                               \\
    C$_7$H$_8$      & Toluene            & 1.154               & 8.96
    & 1.05                               \\
    CHCl$_3$        & Trichloromethane   & 0.672               & 9.96
    & 1.21                               \\ \bottomrule
  \end{tabular}
\end{table}

\pagebreak

\section*{Problem 3.13}
\addcontentsline{toc}{section}{Problem 3.13}
For one of the substances in Table 3.2, compute the change in volume
and work done when one kilogram of the substance is heated from
15$^\circ$C to 25$^\circ$C at a constant pressure of 1 bar.

\begin{solution}
  \begin{gather*}
    \Delta V=V\left( \beta \Delta T-\kappa \cancel{\Delta P}   \right)\\
    W=-P\Delta V
  \end{gather*}
  \vspace{-1em}
  \begin{empheq}[box=\widefbox]{gather*}
    \Delta V =
    \begin{bmatrix}
      1.03e-05 \\
      7.91e-06 \\
      8.87e-06 \\
      8.50e-06 \\
      1.48e-05 \\
      2.31e-05 \\
      1.77e-05 \\
      1.50e-05 \\
      1.15e-05 \\
      1.88e-05 \\
      7.16e-06 \\
      1.21e-05 \\
      8.13e-06 \\
    \end{bmatrix}
    \si{\meter\cubed} \qquad
    W =
    \begin{bmatrix}
      -1.03 \\
      -0.79 \\
      -0.89 \\
      -0.85 \\
      -1.48 \\
      -2.31 \\
      -1.77 \\
      -1.50 \\
      -1.15 \\
      -1.88 \\
      -0.72 \\
      -1.21 \\
      -0.81 \\
    \end{bmatrix}
    \si{\joule}
  \end{empheq}
\end{solution}

\section*{Problem 3.14}
\addcontentsline{toc}{section}{Problem 3.14}
For one of the substances in Table 3.2, compute the change in volume
and work done when one kilogram of the substance is compressed from 1
bar to 100 bar at a constant temperature of 20$^\circ$C.

\begin{solution}
  \begin{gather*}
    dV = -V\kappa dP \\
    W = \int_{V_{1}}^{V_{2}} -P \, dV \\
    W = \frac{V\kappa }{2}\left( P_{2}^{2}-P_{1}^{2} \right)
  \end{gather*}
  \begin{empheq}[box=\widefbox]{gather*}
    \Delta V =
    \begin{bmatrix}
      -8.55e-06 \\
      -4.38e-06 \\
      -7.35e-06 \\
      -6.67e-06 \\
      -1.44e-05 \\
      -2.59e-05 \\
      -1.40e-05 \\
      -1.24e-05 \\
      -9.69e-06 \\
      -1.52e-05 \\
      -6.53e-06 \\
      -1.02e-05 \\
      -6.63e-06
    \end{bmatrix} \, \si{\meter\cubed} \qquad
    W =
    \begin{bmatrix}
      4.32e-04 \\
      2.21e-04 \\
      3.71e-04 \\
      3.37e-04 \\
      7.26e-04 \\
      1.31e-03 \\
      7.08e-04 \\
      6.28e-04 \\
      4.89e-04 \\
      7.66e-04 \\
      3.30e-04 \\
      5.17e-04 \\
      3.35e-04
    \end{bmatrix} \, \si{\joule}
  \end{empheq}
\end{solution}

\section*{Problem 3.15}
\addcontentsline{toc}{section}{Problem 3.15}
For one of the substances in Table 3.2, compute the final pressure
when the substance is heated from 15$^\circ$C and 1 bar to
25$^\circ$C at constant volume.

\begin{solution}
  \begin{gather*}
    P_{2}=P_{1}+\frac{\beta \Delta T}{\kappa }
  \end{gather*}
  \begin{empheq}[box=\widefbox]{gather*}
    P_{2} =
    \begin{bmatrix}
      2.19 \\
      2.79 \\
      2.19 \\
      2.26 \\
      2.02 \\
      1.88 \\
      2.25 \\
      2.19 \\
      2.17 \\
      2.23 \\
      2.09 \\
      2.17 \\
      2.21 \\
    \end{bmatrix}
    \si{\bar}
  \end{empheq}
\end{solution}

\section*{Problem 3.16}
\addcontentsline{toc}{section}{Problem 3.16}
A substance for which $\kappa$ is a constant undergoes an isothermal,
mechanically reversible process from initial state ($P_1, V_1$) to
final state ($P_2, V_2$), where $V$ is molar volume.
\begin{enumerate}
  \item[(a)] Starting with the definition of $\kappa$, show that the
    path of the process is described by:
    $$V = A(T)\exp(-\kappa P)$$
  \item[(b)] Determine an exact expression which gives the isothermal
    work done on 1 mol of this constant-$\kappa$ substance.
\end{enumerate}

\begin{solution}
  \begin{enumerate}[label=(\alph*)]
    \item
      \begin{gather*}
        \kappa \equiv -\frac{1}{V}\left( \frac{\partial V}{\partial P}
        \right)_{T} \\
        \kappa dP = \frac{-dV}{V} \\
        \kappa P = -\ln V + A \\
        \boxed{ V = A\left( T \right)e^{-\kappa P} }
      \end{gather*}
    \item Derivation of the previous answer to solve for $P$ in
      $W=-PdV$, then solving for the integral:
      \begin{gather*}
        P = \frac{1}{\kappa }\ln\left( \frac{A\left( T
        \right)}{V} \right) \\
        W = -\int_{V_{1}}^{V_{2}} \frac{1}{\kappa }\ln\left(
        \frac{A\left( T \right)}{V} \right) \, dV \\
        W = \frac{1}{\kappa }\left[ V\ln\left( \frac{V}{A\left( T
          \right)} \right)-\int \frac{1}{A\left( T \right)} \, dV
        \right]_{V_{1}}^{V_{2}}\\
        \boxed{ W = \Delta V \ln \frac{V_{2}}{V_{1}} - \frac{\Delta
        V}{A\left( T \right)} }
      \end{gather*}
  \end{enumerate}
\end{solution}

\section*{Problem 3.17}
\addcontentsline{toc}{section}{Problem 3.17}
One mole of an ideal gas with $C_P = \frac{7}{2}R$ and $C_V =
\frac{5}{2}R$ expands from $P_1 = \SI{8}{\bar}$ and $T_1 =
\SI{600}{\kelvin}$ to $P_2 = \SI{1}{\bar}$ by each of the following paths:
\begin{enumerate}[label=(\alph*)]
  \item Constant volume;
  \item Constant temperature;
  \item Adiabatically.
\end{enumerate}
Assuming mechanical reversibility, calculate $W$, $Q$, $\Delta U$,
and $\Delta H$ for each process. Sketch each path on a single $PV$ diagram.

\begin{solution}
  \begin{enumerate}[label=(\alph*)]
    \item
      \begin{gather*}
        Q=\Delta U=C_{V}\Delta T\\
        \Delta H=\Delta U+\Delta \left( PV \right)\\
        \frac{P_{1}}{P_{2}}=\frac{T_{1}}{T_{2}}  \to  T_{2}=75~\unit{
        \kelvin }\\
        V = \frac{RT}{P}  \to  V =
        6.235854\times10^{-3}~\unit{ \meter\cubed }\\
        \fbox{%
          \parbox{\widthof{$Q = \Delta U =
          \SI{-10.9}{\kilo\joule}$}}{%
            \centering
            $W = 0$ \\
            $Q = \Delta U = \SI{-10.9}{\kilo\joule}$ \\
            $\Delta H = \SI{-15.3}{\kilo\joule}$
          }
        }
      \end{gather*}
    \item
      \begin{gather*}
        W=-RT\ln\frac{V_{2}}{V_{1}}=-RT\ln\frac{P_{1}}{P_{2}} \\
        W=-Q
      \end{gather*}
      \begin{empheq}[box=\widefbox]{gather*}
        \Delta H = \Delta U = 0 \\
        W = -10.4~\unit{ \kilo\joule }\\
        Q = 10.4~\unit{ \kilo\joule }
      \end{empheq}
    \item
      \begin{gather*}
        T_{1}P_{1}^{\left( 1-\gamma
        \right)/\gamma}=T_{2}P_{2}^{\left( 1-\gamma  \right)/\gamma }\\
        T_{2}=331.227~\unit{ \kelvin }\\
        \Delta U=W=C_{V}\Delta T\\
        \Delta H=C_{P}\Delta T
      \end{gather*}
      \begin{empheq}[box=\widefbox]{gather*}
        Q=0\\
        \Delta U=W=-5.59~\unit{ \kilo\joule\per\mole }\\
        \Delta H=-7.82~\unit{ \kilo\joule\per\mole }
      \end{empheq}
  \end{enumerate}
  \begin{figure}[h!]
    \centering
    \scalebox{0.5}{%% Creator: Matplotlib, PGF backend
%%
%% To include the figure in your LaTeX document, write
%%   \input{<filename>.pgf}
%%
%% Make sure the required packages are loaded in your preamble
%%   \usepackage{pgf}
%%
%% Also ensure that all the required font packages are loaded; for instance,
%% the lmodern package is sometimes necessary when using math font.
%%   \usepackage{lmodern}
%%
%% Figures using additional raster images can only be included by \input if
%% they are in the same directory as the main LaTeX file. For loading figures
%% from other directories you can use the `import` package
%%   \usepackage{import}
%%
%% and then include the figures with
%%   \import{<path to file>}{<filename>.pgf}
%%
%% Matplotlib used the following preamble
%%   \def\mathdefault#1{#1}
%%   \everymath=\expandafter{\the\everymath\displaystyle}
%%   \IfFileExists{scrextend.sty}{
%%     \usepackage[fontsize=10.000000pt]{scrextend}
%%   }{
%%     \renewcommand{\normalsize}{\fontsize{10.000000}{12.000000}\selectfont}
%%     \normalsize
%%   }
%%   
%%   \ifdefined\pdftexversion\else  % non-pdftex case.
%%     \usepackage{fontspec}
%%     \setmainfont{DejaVuSerif.ttf}[Path=\detokenize{C:/Users/Jian/AppData/Local/Programs/Python/Python313/Lib/site-packages/matplotlib/mpl-data/fonts/ttf/}]
%%     \setsansfont{DejaVuSans.ttf}[Path=\detokenize{C:/Users/Jian/AppData/Local/Programs/Python/Python313/Lib/site-packages/matplotlib/mpl-data/fonts/ttf/}]
%%     \setmonofont{DejaVuSansMono.ttf}[Path=\detokenize{C:/Users/Jian/AppData/Local/Programs/Python/Python313/Lib/site-packages/matplotlib/mpl-data/fonts/ttf/}]
%%   \fi
%%   \makeatletter\@ifpackageloaded{underscore}{}{\usepackage[strings]{underscore}}\makeatother
%%
\begingroup%
\makeatletter%
\begin{pgfpicture}%
\pgfpathrectangle{\pgfpointorigin}{\pgfqpoint{9.360000in}{9.400000in}}%
\pgfusepath{use as bounding box, clip}%
\begin{pgfscope}%
\pgfsetbuttcap%
\pgfsetmiterjoin%
\definecolor{currentfill}{rgb}{1.000000,1.000000,1.000000}%
\pgfsetfillcolor{currentfill}%
\pgfsetlinewidth{0.000000pt}%
\definecolor{currentstroke}{rgb}{1.000000,1.000000,1.000000}%
\pgfsetstrokecolor{currentstroke}%
\pgfsetdash{}{0pt}%
\pgfpathmoveto{\pgfqpoint{0.000000in}{0.000000in}}%
\pgfpathlineto{\pgfqpoint{9.360000in}{0.000000in}}%
\pgfpathlineto{\pgfqpoint{9.360000in}{9.400000in}}%
\pgfpathlineto{\pgfqpoint{0.000000in}{9.400000in}}%
\pgfpathlineto{\pgfqpoint{0.000000in}{0.000000in}}%
\pgfpathclose%
\pgfusepath{fill}%
\end{pgfscope}%
\begin{pgfscope}%
\pgfsetbuttcap%
\pgfsetmiterjoin%
\definecolor{currentfill}{rgb}{1.000000,1.000000,1.000000}%
\pgfsetfillcolor{currentfill}%
\pgfsetlinewidth{0.000000pt}%
\definecolor{currentstroke}{rgb}{0.000000,0.000000,0.000000}%
\pgfsetstrokecolor{currentstroke}%
\pgfsetstrokeopacity{0.000000}%
\pgfsetdash{}{0pt}%
\pgfpathmoveto{\pgfqpoint{1.170000in}{1.034000in}}%
\pgfpathlineto{\pgfqpoint{8.424000in}{1.034000in}}%
\pgfpathlineto{\pgfqpoint{8.424000in}{8.272000in}}%
\pgfpathlineto{\pgfqpoint{1.170000in}{8.272000in}}%
\pgfpathlineto{\pgfqpoint{1.170000in}{1.034000in}}%
\pgfpathclose%
\pgfusepath{fill}%
\end{pgfscope}%
\begin{pgfscope}%
\pgfpathrectangle{\pgfqpoint{1.170000in}{1.034000in}}{\pgfqpoint{7.254000in}{7.238000in}}%
\pgfusepath{clip}%
\pgfsetrectcap%
\pgfsetroundjoin%
\pgfsetlinewidth{0.803000pt}%
\definecolor{currentstroke}{rgb}{0.690196,0.690196,0.690196}%
\pgfsetstrokecolor{currentstroke}%
\pgfsetdash{}{0pt}%
\pgfpathmoveto{\pgfqpoint{2.232455in}{1.034000in}}%
\pgfpathlineto{\pgfqpoint{2.232455in}{8.272000in}}%
\pgfusepath{stroke}%
\end{pgfscope}%
\begin{pgfscope}%
\pgfsetbuttcap%
\pgfsetroundjoin%
\definecolor{currentfill}{rgb}{0.000000,0.000000,0.000000}%
\pgfsetfillcolor{currentfill}%
\pgfsetlinewidth{0.803000pt}%
\definecolor{currentstroke}{rgb}{0.000000,0.000000,0.000000}%
\pgfsetstrokecolor{currentstroke}%
\pgfsetdash{}{0pt}%
\pgfsys@defobject{currentmarker}{\pgfqpoint{0.000000in}{-0.048611in}}{\pgfqpoint{0.000000in}{0.000000in}}{%
\pgfpathmoveto{\pgfqpoint{0.000000in}{0.000000in}}%
\pgfpathlineto{\pgfqpoint{0.000000in}{-0.048611in}}%
\pgfusepath{stroke,fill}%
}%
\begin{pgfscope}%
\pgfsys@transformshift{2.232455in}{1.034000in}%
\pgfsys@useobject{currentmarker}{}%
\end{pgfscope}%
\end{pgfscope}%
\begin{pgfscope}%
\definecolor{textcolor}{rgb}{0.000000,0.000000,0.000000}%
\pgfsetstrokecolor{textcolor}%
\pgfsetfillcolor{textcolor}%
\pgftext[x=2.232455in,y=0.936778in,,top]{\color{textcolor}{\rmfamily\fontsize{10.000000}{12.000000}\selectfont\catcode`\^=\active\def^{\ifmmode\sp\else\^{}\fi}\catcode`\%=\active\def%{\%}$\mathdefault{2}$}}%
\end{pgfscope}%
\begin{pgfscope}%
\pgfpathrectangle{\pgfqpoint{1.170000in}{1.034000in}}{\pgfqpoint{7.254000in}{7.238000in}}%
\pgfusepath{clip}%
\pgfsetrectcap%
\pgfsetroundjoin%
\pgfsetlinewidth{0.803000pt}%
\definecolor{currentstroke}{rgb}{0.690196,0.690196,0.690196}%
\pgfsetstrokecolor{currentstroke}%
\pgfsetdash{}{0pt}%
\pgfpathmoveto{\pgfqpoint{3.697909in}{1.034000in}}%
\pgfpathlineto{\pgfqpoint{3.697909in}{8.272000in}}%
\pgfusepath{stroke}%
\end{pgfscope}%
\begin{pgfscope}%
\pgfsetbuttcap%
\pgfsetroundjoin%
\definecolor{currentfill}{rgb}{0.000000,0.000000,0.000000}%
\pgfsetfillcolor{currentfill}%
\pgfsetlinewidth{0.803000pt}%
\definecolor{currentstroke}{rgb}{0.000000,0.000000,0.000000}%
\pgfsetstrokecolor{currentstroke}%
\pgfsetdash{}{0pt}%
\pgfsys@defobject{currentmarker}{\pgfqpoint{0.000000in}{-0.048611in}}{\pgfqpoint{0.000000in}{0.000000in}}{%
\pgfpathmoveto{\pgfqpoint{0.000000in}{0.000000in}}%
\pgfpathlineto{\pgfqpoint{0.000000in}{-0.048611in}}%
\pgfusepath{stroke,fill}%
}%
\begin{pgfscope}%
\pgfsys@transformshift{3.697909in}{1.034000in}%
\pgfsys@useobject{currentmarker}{}%
\end{pgfscope}%
\end{pgfscope}%
\begin{pgfscope}%
\definecolor{textcolor}{rgb}{0.000000,0.000000,0.000000}%
\pgfsetstrokecolor{textcolor}%
\pgfsetfillcolor{textcolor}%
\pgftext[x=3.697909in,y=0.936778in,,top]{\color{textcolor}{\rmfamily\fontsize{10.000000}{12.000000}\selectfont\catcode`\^=\active\def^{\ifmmode\sp\else\^{}\fi}\catcode`\%=\active\def%{\%}$\mathdefault{4}$}}%
\end{pgfscope}%
\begin{pgfscope}%
\pgfpathrectangle{\pgfqpoint{1.170000in}{1.034000in}}{\pgfqpoint{7.254000in}{7.238000in}}%
\pgfusepath{clip}%
\pgfsetrectcap%
\pgfsetroundjoin%
\pgfsetlinewidth{0.803000pt}%
\definecolor{currentstroke}{rgb}{0.690196,0.690196,0.690196}%
\pgfsetstrokecolor{currentstroke}%
\pgfsetdash{}{0pt}%
\pgfpathmoveto{\pgfqpoint{5.163364in}{1.034000in}}%
\pgfpathlineto{\pgfqpoint{5.163364in}{8.272000in}}%
\pgfusepath{stroke}%
\end{pgfscope}%
\begin{pgfscope}%
\pgfsetbuttcap%
\pgfsetroundjoin%
\definecolor{currentfill}{rgb}{0.000000,0.000000,0.000000}%
\pgfsetfillcolor{currentfill}%
\pgfsetlinewidth{0.803000pt}%
\definecolor{currentstroke}{rgb}{0.000000,0.000000,0.000000}%
\pgfsetstrokecolor{currentstroke}%
\pgfsetdash{}{0pt}%
\pgfsys@defobject{currentmarker}{\pgfqpoint{0.000000in}{-0.048611in}}{\pgfqpoint{0.000000in}{0.000000in}}{%
\pgfpathmoveto{\pgfqpoint{0.000000in}{0.000000in}}%
\pgfpathlineto{\pgfqpoint{0.000000in}{-0.048611in}}%
\pgfusepath{stroke,fill}%
}%
\begin{pgfscope}%
\pgfsys@transformshift{5.163364in}{1.034000in}%
\pgfsys@useobject{currentmarker}{}%
\end{pgfscope}%
\end{pgfscope}%
\begin{pgfscope}%
\definecolor{textcolor}{rgb}{0.000000,0.000000,0.000000}%
\pgfsetstrokecolor{textcolor}%
\pgfsetfillcolor{textcolor}%
\pgftext[x=5.163364in,y=0.936778in,,top]{\color{textcolor}{\rmfamily\fontsize{10.000000}{12.000000}\selectfont\catcode`\^=\active\def^{\ifmmode\sp\else\^{}\fi}\catcode`\%=\active\def%{\%}$\mathdefault{6}$}}%
\end{pgfscope}%
\begin{pgfscope}%
\pgfpathrectangle{\pgfqpoint{1.170000in}{1.034000in}}{\pgfqpoint{7.254000in}{7.238000in}}%
\pgfusepath{clip}%
\pgfsetrectcap%
\pgfsetroundjoin%
\pgfsetlinewidth{0.803000pt}%
\definecolor{currentstroke}{rgb}{0.690196,0.690196,0.690196}%
\pgfsetstrokecolor{currentstroke}%
\pgfsetdash{}{0pt}%
\pgfpathmoveto{\pgfqpoint{6.628818in}{1.034000in}}%
\pgfpathlineto{\pgfqpoint{6.628818in}{8.272000in}}%
\pgfusepath{stroke}%
\end{pgfscope}%
\begin{pgfscope}%
\pgfsetbuttcap%
\pgfsetroundjoin%
\definecolor{currentfill}{rgb}{0.000000,0.000000,0.000000}%
\pgfsetfillcolor{currentfill}%
\pgfsetlinewidth{0.803000pt}%
\definecolor{currentstroke}{rgb}{0.000000,0.000000,0.000000}%
\pgfsetstrokecolor{currentstroke}%
\pgfsetdash{}{0pt}%
\pgfsys@defobject{currentmarker}{\pgfqpoint{0.000000in}{-0.048611in}}{\pgfqpoint{0.000000in}{0.000000in}}{%
\pgfpathmoveto{\pgfqpoint{0.000000in}{0.000000in}}%
\pgfpathlineto{\pgfqpoint{0.000000in}{-0.048611in}}%
\pgfusepath{stroke,fill}%
}%
\begin{pgfscope}%
\pgfsys@transformshift{6.628818in}{1.034000in}%
\pgfsys@useobject{currentmarker}{}%
\end{pgfscope}%
\end{pgfscope}%
\begin{pgfscope}%
\definecolor{textcolor}{rgb}{0.000000,0.000000,0.000000}%
\pgfsetstrokecolor{textcolor}%
\pgfsetfillcolor{textcolor}%
\pgftext[x=6.628818in,y=0.936778in,,top]{\color{textcolor}{\rmfamily\fontsize{10.000000}{12.000000}\selectfont\catcode`\^=\active\def^{\ifmmode\sp\else\^{}\fi}\catcode`\%=\active\def%{\%}$\mathdefault{8}$}}%
\end{pgfscope}%
\begin{pgfscope}%
\pgfpathrectangle{\pgfqpoint{1.170000in}{1.034000in}}{\pgfqpoint{7.254000in}{7.238000in}}%
\pgfusepath{clip}%
\pgfsetrectcap%
\pgfsetroundjoin%
\pgfsetlinewidth{0.803000pt}%
\definecolor{currentstroke}{rgb}{0.690196,0.690196,0.690196}%
\pgfsetstrokecolor{currentstroke}%
\pgfsetdash{}{0pt}%
\pgfpathmoveto{\pgfqpoint{8.094273in}{1.034000in}}%
\pgfpathlineto{\pgfqpoint{8.094273in}{8.272000in}}%
\pgfusepath{stroke}%
\end{pgfscope}%
\begin{pgfscope}%
\pgfsetbuttcap%
\pgfsetroundjoin%
\definecolor{currentfill}{rgb}{0.000000,0.000000,0.000000}%
\pgfsetfillcolor{currentfill}%
\pgfsetlinewidth{0.803000pt}%
\definecolor{currentstroke}{rgb}{0.000000,0.000000,0.000000}%
\pgfsetstrokecolor{currentstroke}%
\pgfsetdash{}{0pt}%
\pgfsys@defobject{currentmarker}{\pgfqpoint{0.000000in}{-0.048611in}}{\pgfqpoint{0.000000in}{0.000000in}}{%
\pgfpathmoveto{\pgfqpoint{0.000000in}{0.000000in}}%
\pgfpathlineto{\pgfqpoint{0.000000in}{-0.048611in}}%
\pgfusepath{stroke,fill}%
}%
\begin{pgfscope}%
\pgfsys@transformshift{8.094273in}{1.034000in}%
\pgfsys@useobject{currentmarker}{}%
\end{pgfscope}%
\end{pgfscope}%
\begin{pgfscope}%
\definecolor{textcolor}{rgb}{0.000000,0.000000,0.000000}%
\pgfsetstrokecolor{textcolor}%
\pgfsetfillcolor{textcolor}%
\pgftext[x=8.094273in,y=0.936778in,,top]{\color{textcolor}{\rmfamily\fontsize{10.000000}{12.000000}\selectfont\catcode`\^=\active\def^{\ifmmode\sp\else\^{}\fi}\catcode`\%=\active\def%{\%}$\mathdefault{10}$}}%
\end{pgfscope}%
\begin{pgfscope}%
\definecolor{textcolor}{rgb}{0.000000,0.000000,0.000000}%
\pgfsetstrokecolor{textcolor}%
\pgfsetfillcolor{textcolor}%
\pgftext[x=4.797000in,y=0.746809in,,top]{\color{textcolor}{\rmfamily\fontsize{14.000000}{16.800000}\selectfont\catcode`\^=\active\def^{\ifmmode\sp\else\^{}\fi}\catcode`\%=\active\def%{\%}$P/\mathrm{bar}^{-1}$}}%
\end{pgfscope}%
\begin{pgfscope}%
\pgfpathrectangle{\pgfqpoint{1.170000in}{1.034000in}}{\pgfqpoint{7.254000in}{7.238000in}}%
\pgfusepath{clip}%
\pgfsetrectcap%
\pgfsetroundjoin%
\pgfsetlinewidth{0.803000pt}%
\definecolor{currentstroke}{rgb}{0.690196,0.690196,0.690196}%
\pgfsetstrokecolor{currentstroke}%
\pgfsetdash{}{0pt}%
\pgfpathmoveto{\pgfqpoint{1.170000in}{2.097430in}}%
\pgfpathlineto{\pgfqpoint{8.424000in}{2.097430in}}%
\pgfusepath{stroke}%
\end{pgfscope}%
\begin{pgfscope}%
\pgfsetbuttcap%
\pgfsetroundjoin%
\definecolor{currentfill}{rgb}{0.000000,0.000000,0.000000}%
\pgfsetfillcolor{currentfill}%
\pgfsetlinewidth{0.803000pt}%
\definecolor{currentstroke}{rgb}{0.000000,0.000000,0.000000}%
\pgfsetstrokecolor{currentstroke}%
\pgfsetdash{}{0pt}%
\pgfsys@defobject{currentmarker}{\pgfqpoint{-0.048611in}{0.000000in}}{\pgfqpoint{-0.000000in}{0.000000in}}{%
\pgfpathmoveto{\pgfqpoint{-0.000000in}{0.000000in}}%
\pgfpathlineto{\pgfqpoint{-0.048611in}{0.000000in}}%
\pgfusepath{stroke,fill}%
}%
\begin{pgfscope}%
\pgfsys@transformshift{1.170000in}{2.097430in}%
\pgfsys@useobject{currentmarker}{}%
\end{pgfscope}%
\end{pgfscope}%
\begin{pgfscope}%
\definecolor{textcolor}{rgb}{0.000000,0.000000,0.000000}%
\pgfsetstrokecolor{textcolor}%
\pgfsetfillcolor{textcolor}%
\pgftext[x=0.794999in, y=2.044668in, left, base]{\color{textcolor}{\rmfamily\fontsize{10.000000}{12.000000}\selectfont\catcode`\^=\active\def^{\ifmmode\sp\else\^{}\fi}\catcode`\%=\active\def%{\%}$\mathdefault{1000}$}}%
\end{pgfscope}%
\begin{pgfscope}%
\pgfpathrectangle{\pgfqpoint{1.170000in}{1.034000in}}{\pgfqpoint{7.254000in}{7.238000in}}%
\pgfusepath{clip}%
\pgfsetrectcap%
\pgfsetroundjoin%
\pgfsetlinewidth{0.803000pt}%
\definecolor{currentstroke}{rgb}{0.690196,0.690196,0.690196}%
\pgfsetstrokecolor{currentstroke}%
\pgfsetdash{}{0pt}%
\pgfpathmoveto{\pgfqpoint{1.170000in}{3.562971in}}%
\pgfpathlineto{\pgfqpoint{8.424000in}{3.562971in}}%
\pgfusepath{stroke}%
\end{pgfscope}%
\begin{pgfscope}%
\pgfsetbuttcap%
\pgfsetroundjoin%
\definecolor{currentfill}{rgb}{0.000000,0.000000,0.000000}%
\pgfsetfillcolor{currentfill}%
\pgfsetlinewidth{0.803000pt}%
\definecolor{currentstroke}{rgb}{0.000000,0.000000,0.000000}%
\pgfsetstrokecolor{currentstroke}%
\pgfsetdash{}{0pt}%
\pgfsys@defobject{currentmarker}{\pgfqpoint{-0.048611in}{0.000000in}}{\pgfqpoint{-0.000000in}{0.000000in}}{%
\pgfpathmoveto{\pgfqpoint{-0.000000in}{0.000000in}}%
\pgfpathlineto{\pgfqpoint{-0.048611in}{0.000000in}}%
\pgfusepath{stroke,fill}%
}%
\begin{pgfscope}%
\pgfsys@transformshift{1.170000in}{3.562971in}%
\pgfsys@useobject{currentmarker}{}%
\end{pgfscope}%
\end{pgfscope}%
\begin{pgfscope}%
\definecolor{textcolor}{rgb}{0.000000,0.000000,0.000000}%
\pgfsetstrokecolor{textcolor}%
\pgfsetfillcolor{textcolor}%
\pgftext[x=0.794999in, y=3.510209in, left, base]{\color{textcolor}{\rmfamily\fontsize{10.000000}{12.000000}\selectfont\catcode`\^=\active\def^{\ifmmode\sp\else\^{}\fi}\catcode`\%=\active\def%{\%}$\mathdefault{2000}$}}%
\end{pgfscope}%
\begin{pgfscope}%
\pgfpathrectangle{\pgfqpoint{1.170000in}{1.034000in}}{\pgfqpoint{7.254000in}{7.238000in}}%
\pgfusepath{clip}%
\pgfsetrectcap%
\pgfsetroundjoin%
\pgfsetlinewidth{0.803000pt}%
\definecolor{currentstroke}{rgb}{0.690196,0.690196,0.690196}%
\pgfsetstrokecolor{currentstroke}%
\pgfsetdash{}{0pt}%
\pgfpathmoveto{\pgfqpoint{1.170000in}{5.028512in}}%
\pgfpathlineto{\pgfqpoint{8.424000in}{5.028512in}}%
\pgfusepath{stroke}%
\end{pgfscope}%
\begin{pgfscope}%
\pgfsetbuttcap%
\pgfsetroundjoin%
\definecolor{currentfill}{rgb}{0.000000,0.000000,0.000000}%
\pgfsetfillcolor{currentfill}%
\pgfsetlinewidth{0.803000pt}%
\definecolor{currentstroke}{rgb}{0.000000,0.000000,0.000000}%
\pgfsetstrokecolor{currentstroke}%
\pgfsetdash{}{0pt}%
\pgfsys@defobject{currentmarker}{\pgfqpoint{-0.048611in}{0.000000in}}{\pgfqpoint{-0.000000in}{0.000000in}}{%
\pgfpathmoveto{\pgfqpoint{-0.000000in}{0.000000in}}%
\pgfpathlineto{\pgfqpoint{-0.048611in}{0.000000in}}%
\pgfusepath{stroke,fill}%
}%
\begin{pgfscope}%
\pgfsys@transformshift{1.170000in}{5.028512in}%
\pgfsys@useobject{currentmarker}{}%
\end{pgfscope}%
\end{pgfscope}%
\begin{pgfscope}%
\definecolor{textcolor}{rgb}{0.000000,0.000000,0.000000}%
\pgfsetstrokecolor{textcolor}%
\pgfsetfillcolor{textcolor}%
\pgftext[x=0.794999in, y=4.975750in, left, base]{\color{textcolor}{\rmfamily\fontsize{10.000000}{12.000000}\selectfont\catcode`\^=\active\def^{\ifmmode\sp\else\^{}\fi}\catcode`\%=\active\def%{\%}$\mathdefault{3000}$}}%
\end{pgfscope}%
\begin{pgfscope}%
\pgfpathrectangle{\pgfqpoint{1.170000in}{1.034000in}}{\pgfqpoint{7.254000in}{7.238000in}}%
\pgfusepath{clip}%
\pgfsetrectcap%
\pgfsetroundjoin%
\pgfsetlinewidth{0.803000pt}%
\definecolor{currentstroke}{rgb}{0.690196,0.690196,0.690196}%
\pgfsetstrokecolor{currentstroke}%
\pgfsetdash{}{0pt}%
\pgfpathmoveto{\pgfqpoint{1.170000in}{6.494053in}}%
\pgfpathlineto{\pgfqpoint{8.424000in}{6.494053in}}%
\pgfusepath{stroke}%
\end{pgfscope}%
\begin{pgfscope}%
\pgfsetbuttcap%
\pgfsetroundjoin%
\definecolor{currentfill}{rgb}{0.000000,0.000000,0.000000}%
\pgfsetfillcolor{currentfill}%
\pgfsetlinewidth{0.803000pt}%
\definecolor{currentstroke}{rgb}{0.000000,0.000000,0.000000}%
\pgfsetstrokecolor{currentstroke}%
\pgfsetdash{}{0pt}%
\pgfsys@defobject{currentmarker}{\pgfqpoint{-0.048611in}{0.000000in}}{\pgfqpoint{-0.000000in}{0.000000in}}{%
\pgfpathmoveto{\pgfqpoint{-0.000000in}{0.000000in}}%
\pgfpathlineto{\pgfqpoint{-0.048611in}{0.000000in}}%
\pgfusepath{stroke,fill}%
}%
\begin{pgfscope}%
\pgfsys@transformshift{1.170000in}{6.494053in}%
\pgfsys@useobject{currentmarker}{}%
\end{pgfscope}%
\end{pgfscope}%
\begin{pgfscope}%
\definecolor{textcolor}{rgb}{0.000000,0.000000,0.000000}%
\pgfsetstrokecolor{textcolor}%
\pgfsetfillcolor{textcolor}%
\pgftext[x=0.794999in, y=6.441291in, left, base]{\color{textcolor}{\rmfamily\fontsize{10.000000}{12.000000}\selectfont\catcode`\^=\active\def^{\ifmmode\sp\else\^{}\fi}\catcode`\%=\active\def%{\%}$\mathdefault{4000}$}}%
\end{pgfscope}%
\begin{pgfscope}%
\pgfpathrectangle{\pgfqpoint{1.170000in}{1.034000in}}{\pgfqpoint{7.254000in}{7.238000in}}%
\pgfusepath{clip}%
\pgfsetrectcap%
\pgfsetroundjoin%
\pgfsetlinewidth{0.803000pt}%
\definecolor{currentstroke}{rgb}{0.690196,0.690196,0.690196}%
\pgfsetstrokecolor{currentstroke}%
\pgfsetdash{}{0pt}%
\pgfpathmoveto{\pgfqpoint{1.170000in}{7.959593in}}%
\pgfpathlineto{\pgfqpoint{8.424000in}{7.959593in}}%
\pgfusepath{stroke}%
\end{pgfscope}%
\begin{pgfscope}%
\pgfsetbuttcap%
\pgfsetroundjoin%
\definecolor{currentfill}{rgb}{0.000000,0.000000,0.000000}%
\pgfsetfillcolor{currentfill}%
\pgfsetlinewidth{0.803000pt}%
\definecolor{currentstroke}{rgb}{0.000000,0.000000,0.000000}%
\pgfsetstrokecolor{currentstroke}%
\pgfsetdash{}{0pt}%
\pgfsys@defobject{currentmarker}{\pgfqpoint{-0.048611in}{0.000000in}}{\pgfqpoint{-0.000000in}{0.000000in}}{%
\pgfpathmoveto{\pgfqpoint{-0.000000in}{0.000000in}}%
\pgfpathlineto{\pgfqpoint{-0.048611in}{0.000000in}}%
\pgfusepath{stroke,fill}%
}%
\begin{pgfscope}%
\pgfsys@transformshift{1.170000in}{7.959593in}%
\pgfsys@useobject{currentmarker}{}%
\end{pgfscope}%
\end{pgfscope}%
\begin{pgfscope}%
\definecolor{textcolor}{rgb}{0.000000,0.000000,0.000000}%
\pgfsetstrokecolor{textcolor}%
\pgfsetfillcolor{textcolor}%
\pgftext[x=0.794999in, y=7.906832in, left, base]{\color{textcolor}{\rmfamily\fontsize{10.000000}{12.000000}\selectfont\catcode`\^=\active\def^{\ifmmode\sp\else\^{}\fi}\catcode`\%=\active\def%{\%}$\mathdefault{5000}$}}%
\end{pgfscope}%
\begin{pgfscope}%
\definecolor{textcolor}{rgb}{0.000000,0.000000,0.000000}%
\pgfsetstrokecolor{textcolor}%
\pgfsetfillcolor{textcolor}%
\pgftext[x=0.739444in,y=4.653000in,,bottom,rotate=90.000000]{\color{textcolor}{\rmfamily\fontsize{14.000000}{16.800000}\selectfont\catcode`\^=\active\def^{\ifmmode\sp\else\^{}\fi}\catcode`\%=\active\def%{\%}$V/\mathrm{m}^3\,\mathrm{mol}^{-1}$}}%
\end{pgfscope}%
\begin{pgfscope}%
\pgfpathrectangle{\pgfqpoint{1.170000in}{1.034000in}}{\pgfqpoint{7.254000in}{7.238000in}}%
\pgfusepath{clip}%
\pgfsetrectcap%
\pgfsetroundjoin%
\pgfsetlinewidth{1.505625pt}%
\definecolor{currentstroke}{rgb}{0.000000,0.000000,1.000000}%
\pgfsetstrokecolor{currentstroke}%
\pgfsetdash{}{0pt}%
\pgfpathmoveto{\pgfqpoint{1.499727in}{1.545778in}}%
\pgfpathlineto{\pgfqpoint{8.094273in}{1.545778in}}%
\pgfpathlineto{\pgfqpoint{8.094273in}{1.545778in}}%
\pgfusepath{stroke}%
\end{pgfscope}%
\begin{pgfscope}%
\pgfpathrectangle{\pgfqpoint{1.170000in}{1.034000in}}{\pgfqpoint{7.254000in}{7.238000in}}%
\pgfusepath{clip}%
\pgfsetrectcap%
\pgfsetroundjoin%
\pgfsetlinewidth{1.505625pt}%
\definecolor{currentstroke}{rgb}{0.000000,0.501961,0.000000}%
\pgfsetstrokecolor{currentstroke}%
\pgfsetdash{}{0pt}%
\pgfpathmoveto{\pgfqpoint{1.499727in}{7.943000in}}%
\pgfpathlineto{\pgfqpoint{1.526158in}{7.688455in}}%
\pgfpathlineto{\pgfqpoint{1.552589in}{7.451037in}}%
\pgfpathlineto{\pgfqpoint{1.579020in}{7.229076in}}%
\pgfpathlineto{\pgfqpoint{1.605451in}{7.021109in}}%
\pgfpathlineto{\pgfqpoint{1.631882in}{6.825852in}}%
\pgfpathlineto{\pgfqpoint{1.658314in}{6.642176in}}%
\pgfpathlineto{\pgfqpoint{1.684745in}{6.469080in}}%
\pgfpathlineto{\pgfqpoint{1.724391in}{6.227356in}}%
\pgfpathlineto{\pgfqpoint{1.764038in}{6.004856in}}%
\pgfpathlineto{\pgfqpoint{1.803684in}{5.799374in}}%
\pgfpathlineto{\pgfqpoint{1.843331in}{5.609030in}}%
\pgfpathlineto{\pgfqpoint{1.882977in}{5.432211in}}%
\pgfpathlineto{\pgfqpoint{1.922624in}{5.267524in}}%
\pgfpathlineto{\pgfqpoint{1.962271in}{5.113762in}}%
\pgfpathlineto{\pgfqpoint{2.001917in}{4.969873in}}%
\pgfpathlineto{\pgfqpoint{2.041564in}{4.834935in}}%
\pgfpathlineto{\pgfqpoint{2.081210in}{4.708140in}}%
\pgfpathlineto{\pgfqpoint{2.120857in}{4.588770in}}%
\pgfpathlineto{\pgfqpoint{2.160503in}{4.476193in}}%
\pgfpathlineto{\pgfqpoint{2.200150in}{4.369844in}}%
\pgfpathlineto{\pgfqpoint{2.239797in}{4.269221in}}%
\pgfpathlineto{\pgfqpoint{2.279443in}{4.173874in}}%
\pgfpathlineto{\pgfqpoint{2.319090in}{4.083397in}}%
\pgfpathlineto{\pgfqpoint{2.358736in}{3.997428in}}%
\pgfpathlineto{\pgfqpoint{2.398383in}{3.915637in}}%
\pgfpathlineto{\pgfqpoint{2.438029in}{3.837728in}}%
\pgfpathlineto{\pgfqpoint{2.490891in}{3.739422in}}%
\pgfpathlineto{\pgfqpoint{2.543754in}{3.646967in}}%
\pgfpathlineto{\pgfqpoint{2.596616in}{3.559854in}}%
\pgfpathlineto{\pgfqpoint{2.649478in}{3.477633in}}%
\pgfpathlineto{\pgfqpoint{2.702340in}{3.399904in}}%
\pgfpathlineto{\pgfqpoint{2.755202in}{3.326309in}}%
\pgfpathlineto{\pgfqpoint{2.808064in}{3.256525in}}%
\pgfpathlineto{\pgfqpoint{2.860926in}{3.190265in}}%
\pgfpathlineto{\pgfqpoint{2.913788in}{3.127268in}}%
\pgfpathlineto{\pgfqpoint{2.966650in}{3.067299in}}%
\pgfpathlineto{\pgfqpoint{3.019512in}{3.010145in}}%
\pgfpathlineto{\pgfqpoint{3.085590in}{2.942367in}}%
\pgfpathlineto{\pgfqpoint{3.151668in}{2.878345in}}%
\pgfpathlineto{\pgfqpoint{3.217745in}{2.817775in}}%
\pgfpathlineto{\pgfqpoint{3.283823in}{2.760386in}}%
\pgfpathlineto{\pgfqpoint{3.349900in}{2.705933in}}%
\pgfpathlineto{\pgfqpoint{3.415978in}{2.654197in}}%
\pgfpathlineto{\pgfqpoint{3.482056in}{2.604979in}}%
\pgfpathlineto{\pgfqpoint{3.561349in}{2.548991in}}%
\pgfpathlineto{\pgfqpoint{3.640642in}{2.496091in}}%
\pgfpathlineto{\pgfqpoint{3.719935in}{2.446033in}}%
\pgfpathlineto{\pgfqpoint{3.799228in}{2.398593in}}%
\pgfpathlineto{\pgfqpoint{3.878521in}{2.353571in}}%
\pgfpathlineto{\pgfqpoint{3.971030in}{2.303862in}}%
\pgfpathlineto{\pgfqpoint{4.063539in}{2.256942in}}%
\pgfpathlineto{\pgfqpoint{4.156047in}{2.212584in}}%
\pgfpathlineto{\pgfqpoint{4.248556in}{2.170583in}}%
\pgfpathlineto{\pgfqpoint{4.354280in}{2.125235in}}%
\pgfpathlineto{\pgfqpoint{4.460004in}{2.082483in}}%
\pgfpathlineto{\pgfqpoint{4.565728in}{2.042111in}}%
\pgfpathlineto{\pgfqpoint{4.671453in}{2.003925in}}%
\pgfpathlineto{\pgfqpoint{4.790392in}{1.963365in}}%
\pgfpathlineto{\pgfqpoint{4.909332in}{1.925134in}}%
\pgfpathlineto{\pgfqpoint{5.041487in}{1.885150in}}%
\pgfpathlineto{\pgfqpoint{5.173642in}{1.847565in}}%
\pgfpathlineto{\pgfqpoint{5.305798in}{1.812169in}}%
\pgfpathlineto{\pgfqpoint{5.451168in}{1.775539in}}%
\pgfpathlineto{\pgfqpoint{5.596539in}{1.741115in}}%
\pgfpathlineto{\pgfqpoint{5.755125in}{1.705850in}}%
\pgfpathlineto{\pgfqpoint{5.926927in}{1.670092in}}%
\pgfpathlineto{\pgfqpoint{6.098729in}{1.636638in}}%
\pgfpathlineto{\pgfqpoint{6.283746in}{1.602941in}}%
\pgfpathlineto{\pgfqpoint{6.481979in}{1.569259in}}%
\pgfpathlineto{\pgfqpoint{6.680212in}{1.537835in}}%
\pgfpathlineto{\pgfqpoint{6.891660in}{1.506558in}}%
\pgfpathlineto{\pgfqpoint{7.116324in}{1.475609in}}%
\pgfpathlineto{\pgfqpoint{7.354203in}{1.445140in}}%
\pgfpathlineto{\pgfqpoint{7.605298in}{1.415278in}}%
\pgfpathlineto{\pgfqpoint{7.869609in}{1.386126in}}%
\pgfpathlineto{\pgfqpoint{8.094273in}{1.363000in}}%
\pgfpathlineto{\pgfqpoint{8.094273in}{1.363000in}}%
\pgfusepath{stroke}%
\end{pgfscope}%
\begin{pgfscope}%
\pgfpathrectangle{\pgfqpoint{1.170000in}{1.034000in}}{\pgfqpoint{7.254000in}{7.238000in}}%
\pgfusepath{clip}%
\pgfsetrectcap%
\pgfsetroundjoin%
\pgfsetlinewidth{1.505625pt}%
\definecolor{currentstroke}{rgb}{1.000000,0.000000,0.000000}%
\pgfsetstrokecolor{currentstroke}%
\pgfsetdash{}{0pt}%
\pgfpathmoveto{\pgfqpoint{1.499727in}{4.667949in}}%
\pgfpathlineto{\pgfqpoint{1.526158in}{4.567071in}}%
\pgfpathlineto{\pgfqpoint{1.565805in}{4.426551in}}%
\pgfpathlineto{\pgfqpoint{1.605451in}{4.297501in}}%
\pgfpathlineto{\pgfqpoint{1.645098in}{4.178506in}}%
\pgfpathlineto{\pgfqpoint{1.684745in}{4.068380in}}%
\pgfpathlineto{\pgfqpoint{1.724391in}{3.966118in}}%
\pgfpathlineto{\pgfqpoint{1.764038in}{3.870869in}}%
\pgfpathlineto{\pgfqpoint{1.803684in}{3.781898in}}%
\pgfpathlineto{\pgfqpoint{1.843331in}{3.698576in}}%
\pgfpathlineto{\pgfqpoint{1.882977in}{3.620355in}}%
\pgfpathlineto{\pgfqpoint{1.922624in}{3.546757in}}%
\pgfpathlineto{\pgfqpoint{1.962271in}{3.477365in}}%
\pgfpathlineto{\pgfqpoint{2.001917in}{3.411809in}}%
\pgfpathlineto{\pgfqpoint{2.041564in}{3.349765in}}%
\pgfpathlineto{\pgfqpoint{2.081210in}{3.290944in}}%
\pgfpathlineto{\pgfqpoint{2.134072in}{3.217088in}}%
\pgfpathlineto{\pgfqpoint{2.186934in}{3.147971in}}%
\pgfpathlineto{\pgfqpoint{2.239797in}{3.083130in}}%
\pgfpathlineto{\pgfqpoint{2.292659in}{3.022159in}}%
\pgfpathlineto{\pgfqpoint{2.345521in}{2.964705in}}%
\pgfpathlineto{\pgfqpoint{2.398383in}{2.910459in}}%
\pgfpathlineto{\pgfqpoint{2.451245in}{2.859144in}}%
\pgfpathlineto{\pgfqpoint{2.504107in}{2.810518in}}%
\pgfpathlineto{\pgfqpoint{2.570185in}{2.753189in}}%
\pgfpathlineto{\pgfqpoint{2.636262in}{2.699352in}}%
\pgfpathlineto{\pgfqpoint{2.702340in}{2.648682in}}%
\pgfpathlineto{\pgfqpoint{2.768417in}{2.600894in}}%
\pgfpathlineto{\pgfqpoint{2.834495in}{2.555736in}}%
\pgfpathlineto{\pgfqpoint{2.900573in}{2.512986in}}%
\pgfpathlineto{\pgfqpoint{2.979866in}{2.464589in}}%
\pgfpathlineto{\pgfqpoint{3.059159in}{2.419077in}}%
\pgfpathlineto{\pgfqpoint{3.138452in}{2.376186in}}%
\pgfpathlineto{\pgfqpoint{3.217745in}{2.335686in}}%
\pgfpathlineto{\pgfqpoint{3.310254in}{2.291185in}}%
\pgfpathlineto{\pgfqpoint{3.402762in}{2.249375in}}%
\pgfpathlineto{\pgfqpoint{3.495271in}{2.210008in}}%
\pgfpathlineto{\pgfqpoint{3.587780in}{2.172864in}}%
\pgfpathlineto{\pgfqpoint{3.693504in}{2.132891in}}%
\pgfpathlineto{\pgfqpoint{3.799228in}{2.095320in}}%
\pgfpathlineto{\pgfqpoint{3.904952in}{2.059929in}}%
\pgfpathlineto{\pgfqpoint{4.023892in}{2.022481in}}%
\pgfpathlineto{\pgfqpoint{4.142832in}{1.987306in}}%
\pgfpathlineto{\pgfqpoint{4.274987in}{1.950634in}}%
\pgfpathlineto{\pgfqpoint{4.407142in}{1.916256in}}%
\pgfpathlineto{\pgfqpoint{4.552513in}{1.880829in}}%
\pgfpathlineto{\pgfqpoint{4.697884in}{1.847661in}}%
\pgfpathlineto{\pgfqpoint{4.856470in}{1.813795in}}%
\pgfpathlineto{\pgfqpoint{5.028272in}{1.779559in}}%
\pgfpathlineto{\pgfqpoint{5.200073in}{1.747610in}}%
\pgfpathlineto{\pgfqpoint{5.385091in}{1.715496in}}%
\pgfpathlineto{\pgfqpoint{5.583324in}{1.683448in}}%
\pgfpathlineto{\pgfqpoint{5.794772in}{1.651666in}}%
\pgfpathlineto{\pgfqpoint{6.006220in}{1.622096in}}%
\pgfpathlineto{\pgfqpoint{6.230884in}{1.592840in}}%
\pgfpathlineto{\pgfqpoint{6.468764in}{1.564029in}}%
\pgfpathlineto{\pgfqpoint{6.733074in}{1.534342in}}%
\pgfpathlineto{\pgfqpoint{7.010600in}{1.505504in}}%
\pgfpathlineto{\pgfqpoint{7.301341in}{1.477559in}}%
\pgfpathlineto{\pgfqpoint{7.605298in}{1.450535in}}%
\pgfpathlineto{\pgfqpoint{7.935686in}{1.423405in}}%
\pgfpathlineto{\pgfqpoint{8.094273in}{1.411130in}}%
\pgfpathlineto{\pgfqpoint{8.094273in}{1.411130in}}%
\pgfusepath{stroke}%
\end{pgfscope}%
\begin{pgfscope}%
\pgfsetrectcap%
\pgfsetmiterjoin%
\pgfsetlinewidth{0.803000pt}%
\definecolor{currentstroke}{rgb}{0.000000,0.000000,0.000000}%
\pgfsetstrokecolor{currentstroke}%
\pgfsetdash{}{0pt}%
\pgfpathmoveto{\pgfqpoint{1.170000in}{1.034000in}}%
\pgfpathlineto{\pgfqpoint{1.170000in}{8.272000in}}%
\pgfusepath{stroke}%
\end{pgfscope}%
\begin{pgfscope}%
\pgfsetrectcap%
\pgfsetmiterjoin%
\pgfsetlinewidth{0.803000pt}%
\definecolor{currentstroke}{rgb}{0.000000,0.000000,0.000000}%
\pgfsetstrokecolor{currentstroke}%
\pgfsetdash{}{0pt}%
\pgfpathmoveto{\pgfqpoint{8.424000in}{1.034000in}}%
\pgfpathlineto{\pgfqpoint{8.424000in}{8.272000in}}%
\pgfusepath{stroke}%
\end{pgfscope}%
\begin{pgfscope}%
\pgfsetrectcap%
\pgfsetmiterjoin%
\pgfsetlinewidth{0.803000pt}%
\definecolor{currentstroke}{rgb}{0.000000,0.000000,0.000000}%
\pgfsetstrokecolor{currentstroke}%
\pgfsetdash{}{0pt}%
\pgfpathmoveto{\pgfqpoint{1.170000in}{1.034000in}}%
\pgfpathlineto{\pgfqpoint{8.424000in}{1.034000in}}%
\pgfusepath{stroke}%
\end{pgfscope}%
\begin{pgfscope}%
\pgfsetrectcap%
\pgfsetmiterjoin%
\pgfsetlinewidth{0.803000pt}%
\definecolor{currentstroke}{rgb}{0.000000,0.000000,0.000000}%
\pgfsetstrokecolor{currentstroke}%
\pgfsetdash{}{0pt}%
\pgfpathmoveto{\pgfqpoint{1.170000in}{8.272000in}}%
\pgfpathlineto{\pgfqpoint{8.424000in}{8.272000in}}%
\pgfusepath{stroke}%
\end{pgfscope}%
\begin{pgfscope}%
\pgfsetbuttcap%
\pgfsetmiterjoin%
\definecolor{currentfill}{rgb}{1.000000,1.000000,1.000000}%
\pgfsetfillcolor{currentfill}%
\pgfsetfillopacity{0.800000}%
\pgfsetlinewidth{1.003750pt}%
\definecolor{currentstroke}{rgb}{0.800000,0.800000,0.800000}%
\pgfsetstrokecolor{currentstroke}%
\pgfsetstrokeopacity{0.800000}%
\pgfsetdash{}{0pt}%
\pgfpathmoveto{\pgfqpoint{5.441708in}{7.257490in}}%
\pgfpathlineto{\pgfqpoint{8.287889in}{7.257490in}}%
\pgfpathquadraticcurveto{\pgfqpoint{8.326778in}{7.257490in}}{\pgfqpoint{8.326778in}{7.296379in}}%
\pgfpathlineto{\pgfqpoint{8.326778in}{8.135889in}}%
\pgfpathquadraticcurveto{\pgfqpoint{8.326778in}{8.174778in}}{\pgfqpoint{8.287889in}{8.174778in}}%
\pgfpathlineto{\pgfqpoint{5.441708in}{8.174778in}}%
\pgfpathquadraticcurveto{\pgfqpoint{5.402819in}{8.174778in}}{\pgfqpoint{5.402819in}{8.135889in}}%
\pgfpathlineto{\pgfqpoint{5.402819in}{7.296379in}}%
\pgfpathquadraticcurveto{\pgfqpoint{5.402819in}{7.257490in}}{\pgfqpoint{5.441708in}{7.257490in}}%
\pgfpathlineto{\pgfqpoint{5.441708in}{7.257490in}}%
\pgfpathclose%
\pgfusepath{stroke,fill}%
\end{pgfscope}%
\begin{pgfscope}%
\pgfsetrectcap%
\pgfsetroundjoin%
\pgfsetlinewidth{1.505625pt}%
\definecolor{currentstroke}{rgb}{0.000000,0.000000,1.000000}%
\pgfsetstrokecolor{currentstroke}%
\pgfsetdash{}{0pt}%
\pgfpathmoveto{\pgfqpoint{5.480597in}{8.017323in}}%
\pgfpathlineto{\pgfqpoint{5.675042in}{8.017323in}}%
\pgfpathlineto{\pgfqpoint{5.869486in}{8.017323in}}%
\pgfusepath{stroke}%
\end{pgfscope}%
\begin{pgfscope}%
\definecolor{textcolor}{rgb}{0.000000,0.000000,0.000000}%
\pgfsetstrokecolor{textcolor}%
\pgfsetfillcolor{textcolor}%
\pgftext[x=6.025042in,y=7.949268in,left,base]{\color{textcolor}{\rmfamily\fontsize{14.000000}{16.800000}\selectfont\catcode`\^=\active\def^{\ifmmode\sp\else\^{}\fi}\catcode`\%=\active\def%{\%}Constant Volume}}%
\end{pgfscope}%
\begin{pgfscope}%
\pgfsetrectcap%
\pgfsetroundjoin%
\pgfsetlinewidth{1.505625pt}%
\definecolor{currentstroke}{rgb}{0.000000,0.501961,0.000000}%
\pgfsetstrokecolor{currentstroke}%
\pgfsetdash{}{0pt}%
\pgfpathmoveto{\pgfqpoint{5.480597in}{7.731923in}}%
\pgfpathlineto{\pgfqpoint{5.675042in}{7.731923in}}%
\pgfpathlineto{\pgfqpoint{5.869486in}{7.731923in}}%
\pgfusepath{stroke}%
\end{pgfscope}%
\begin{pgfscope}%
\definecolor{textcolor}{rgb}{0.000000,0.000000,0.000000}%
\pgfsetstrokecolor{textcolor}%
\pgfsetfillcolor{textcolor}%
\pgftext[x=6.025042in,y=7.663868in,left,base]{\color{textcolor}{\rmfamily\fontsize{14.000000}{16.800000}\selectfont\catcode`\^=\active\def^{\ifmmode\sp\else\^{}\fi}\catcode`\%=\active\def%{\%}Constant Temperature}}%
\end{pgfscope}%
\begin{pgfscope}%
\pgfsetrectcap%
\pgfsetroundjoin%
\pgfsetlinewidth{1.505625pt}%
\definecolor{currentstroke}{rgb}{1.000000,0.000000,0.000000}%
\pgfsetstrokecolor{currentstroke}%
\pgfsetdash{}{0pt}%
\pgfpathmoveto{\pgfqpoint{5.480597in}{7.446523in}}%
\pgfpathlineto{\pgfqpoint{5.675042in}{7.446523in}}%
\pgfpathlineto{\pgfqpoint{5.869486in}{7.446523in}}%
\pgfusepath{stroke}%
\end{pgfscope}%
\begin{pgfscope}%
\definecolor{textcolor}{rgb}{0.000000,0.000000,0.000000}%
\pgfsetstrokecolor{textcolor}%
\pgfsetfillcolor{textcolor}%
\pgftext[x=6.025042in,y=7.378467in,left,base]{\color{textcolor}{\rmfamily\fontsize{14.000000}{16.800000}\selectfont\catcode`\^=\active\def^{\ifmmode\sp\else\^{}\fi}\catcode`\%=\active\def%{\%}Adiabatically}}%
\end{pgfscope}%
\end{pgfpicture}%
\makeatother%
\endgroup%
}
    \caption{PV diagram showing the three different processes as
    described in Problem 3.17.}
    \label{fig:s17}
  \end{figure}
\end{solution}

\section*{Problem 3.18}
\addcontentsline{toc}{section}{Problem 3.18}
One mole of an ideal gas with $C_P = \frac{5}{2}R$ and $C_V =
\frac{3}{2}R$ expands from $P_1 = \SI{6}{\bar}$ and $T_1 =
\SI{800}{\kelvin}$ to $P_2 = \SI{1}{\bar}$ by each of the following paths:
\begin{enumerate}[label=(\alph*)]
  \item Constant volume;
  \item Constant temperature;
  \item Adiabatically.
\end{enumerate}
Assuming mechanical reversibility, calculate $W$, $Q$, $\Delta U$,
and $\Delta H$ for each process. Sketch each path on a single $PV$ diagram.

\begin{solution}
  \begin{enumerate}[label=(\alph*)]
    \item
      \begin{gather*}
        \Delta U=Q=C_{V}\Delta T\\
        \Delta H=C_{P}\Delta T\\
        \frac{P_{1}}{T_{1}}=\frac{P_{2}}{T_{2}}\\
        T_{2}=133.33~\unit{ \kelvin }
      \end{gather*}
      \begin{empheq}[box=\widefbox]{gather*}
        W=0\\
        \Delta U=Q=-8.31~\unit{ \kilo\joule\per\mole }\\
        \Delta H=-13.86~\unit{ \kilo\joule\per\mole }
      \end{empheq}
    \item
      \begin{gather*}
        W=-RT\ln\frac{P_{1}}{P_{2}}\\
        Q=-W
      \end{gather*}
      \begin{empheq}[box=\widefbox]{gather*}
        W=-11.92~\unit{ \kilo\joule\per\mole }\\
        Q=11.92~\unit{ \kilo\joule\per\mole }
      \end{empheq}
    \item
      \begin{gather*}
        T_{1}P_{1}^{\left( 1-\gamma  \right)/\gamma
        }=T_{2}P_{2}^{\left( 1-\gamma  \right)/\gamma }\\
        T_{2}=390.69~\unit{ \kelvin }\\
        \Delta U=W=C_{V}\Delta T\\
        \Delta H=C_{P}\Delta T
      \end{gather*}
      \begin{empheq}[box=\widefbox]{gather*}
        Q=0\\
        \Delta U=W=-5.10~\unit{ \kilo\joule\per\mole }\\
        \Delta H=-8.51~\unit{ \kilo\joule\per\mole }
      \end{empheq}
  \end{enumerate}
  \begin{figure}[h!]
    \centering
    \scalebox{0.5}{%% Creator: Matplotlib, PGF backend
%%
%% To include the figure in your LaTeX document, write
%%   \input{<filename>.pgf}
%%
%% Make sure the required packages are loaded in your preamble
%%   \usepackage{pgf}
%%
%% Also ensure that all the required font packages are loaded; for instance,
%% the lmodern package is sometimes necessary when using math font.
%%   \usepackage{lmodern}
%%
%% Figures using additional raster images can only be included by \input if
%% they are in the same directory as the main LaTeX file. For loading figures
%% from other directories you can use the `import` package
%%   \usepackage{import}
%%
%% and then include the figures with
%%   \import{<path to file>}{<filename>.pgf}
%%
%% Matplotlib used the following preamble
%%   \def\mathdefault#1{#1}
%%   \everymath=\expandafter{\the\everymath\displaystyle}
%%   \IfFileExists{scrextend.sty}{
%%     \usepackage[fontsize=10.000000pt]{scrextend}
%%   }{
%%     \renewcommand{\normalsize}{\fontsize{10.000000}{12.000000}\selectfont}
%%     \normalsize
%%   }
%%   
%%   \ifdefined\pdftexversion\else  % non-pdftex case.
%%     \usepackage{fontspec}
%%     \setmainfont{DejaVuSerif.ttf}[Path=\detokenize{C:/Users/Jian/AppData/Local/Programs/Python/Python313/Lib/site-packages/matplotlib/mpl-data/fonts/ttf/}]
%%     \setsansfont{DejaVuSans.ttf}[Path=\detokenize{C:/Users/Jian/AppData/Local/Programs/Python/Python313/Lib/site-packages/matplotlib/mpl-data/fonts/ttf/}]
%%     \setmonofont{DejaVuSansMono.ttf}[Path=\detokenize{C:/Users/Jian/AppData/Local/Programs/Python/Python313/Lib/site-packages/matplotlib/mpl-data/fonts/ttf/}]
%%   \fi
%%   \makeatletter\@ifpackageloaded{underscore}{}{\usepackage[strings]{underscore}}\makeatother
%%
\begingroup%
\makeatletter%
\begin{pgfpicture}%
\pgfpathrectangle{\pgfpointorigin}{\pgfqpoint{6.650000in}{4.830000in}}%
\pgfusepath{use as bounding box, clip}%
\begin{pgfscope}%
\pgfsetbuttcap%
\pgfsetmiterjoin%
\definecolor{currentfill}{rgb}{1.000000,1.000000,1.000000}%
\pgfsetfillcolor{currentfill}%
\pgfsetlinewidth{0.000000pt}%
\definecolor{currentstroke}{rgb}{1.000000,1.000000,1.000000}%
\pgfsetstrokecolor{currentstroke}%
\pgfsetdash{}{0pt}%
\pgfpathmoveto{\pgfqpoint{0.000000in}{0.000000in}}%
\pgfpathlineto{\pgfqpoint{6.650000in}{0.000000in}}%
\pgfpathlineto{\pgfqpoint{6.650000in}{4.830000in}}%
\pgfpathlineto{\pgfqpoint{0.000000in}{4.830000in}}%
\pgfpathlineto{\pgfqpoint{0.000000in}{0.000000in}}%
\pgfpathclose%
\pgfusepath{fill}%
\end{pgfscope}%
\begin{pgfscope}%
\pgfsetbuttcap%
\pgfsetmiterjoin%
\definecolor{currentfill}{rgb}{1.000000,1.000000,1.000000}%
\pgfsetfillcolor{currentfill}%
\pgfsetlinewidth{0.000000pt}%
\definecolor{currentstroke}{rgb}{0.000000,0.000000,0.000000}%
\pgfsetstrokecolor{currentstroke}%
\pgfsetstrokeopacity{0.000000}%
\pgfsetdash{}{0pt}%
\pgfpathmoveto{\pgfqpoint{0.831250in}{0.531300in}}%
\pgfpathlineto{\pgfqpoint{5.985000in}{0.531300in}}%
\pgfpathlineto{\pgfqpoint{5.985000in}{4.250400in}}%
\pgfpathlineto{\pgfqpoint{0.831250in}{4.250400in}}%
\pgfpathlineto{\pgfqpoint{0.831250in}{0.531300in}}%
\pgfpathclose%
\pgfusepath{fill}%
\end{pgfscope}%
\begin{pgfscope}%
\pgfpathrectangle{\pgfqpoint{0.831250in}{0.531300in}}{\pgfqpoint{5.153750in}{3.719100in}}%
\pgfusepath{clip}%
\pgfsetrectcap%
\pgfsetroundjoin%
\pgfsetlinewidth{0.803000pt}%
\definecolor{currentstroke}{rgb}{0.690196,0.690196,0.690196}%
\pgfsetstrokecolor{currentstroke}%
\pgfsetdash{}{0pt}%
\pgfpathmoveto{\pgfqpoint{1.435398in}{0.531300in}}%
\pgfpathlineto{\pgfqpoint{1.435398in}{4.250400in}}%
\pgfusepath{stroke}%
\end{pgfscope}%
\begin{pgfscope}%
\pgfsetbuttcap%
\pgfsetroundjoin%
\definecolor{currentfill}{rgb}{0.000000,0.000000,0.000000}%
\pgfsetfillcolor{currentfill}%
\pgfsetlinewidth{0.803000pt}%
\definecolor{currentstroke}{rgb}{0.000000,0.000000,0.000000}%
\pgfsetstrokecolor{currentstroke}%
\pgfsetdash{}{0pt}%
\pgfsys@defobject{currentmarker}{\pgfqpoint{0.000000in}{-0.048611in}}{\pgfqpoint{0.000000in}{0.000000in}}{%
\pgfpathmoveto{\pgfqpoint{0.000000in}{0.000000in}}%
\pgfpathlineto{\pgfqpoint{0.000000in}{-0.048611in}}%
\pgfusepath{stroke,fill}%
}%
\begin{pgfscope}%
\pgfsys@transformshift{1.435398in}{0.531300in}%
\pgfsys@useobject{currentmarker}{}%
\end{pgfscope}%
\end{pgfscope}%
\begin{pgfscope}%
\definecolor{textcolor}{rgb}{0.000000,0.000000,0.000000}%
\pgfsetstrokecolor{textcolor}%
\pgfsetfillcolor{textcolor}%
\pgftext[x=1.435398in,y=0.434078in,,top]{\color{textcolor}{\rmfamily\fontsize{10.000000}{12.000000}\selectfont\catcode`\^=\active\def^{\ifmmode\sp\else\^{}\fi}\catcode`\%=\active\def%{\%}$\mathdefault{0.25}$}}%
\end{pgfscope}%
\begin{pgfscope}%
\pgfpathrectangle{\pgfqpoint{0.831250in}{0.531300in}}{\pgfqpoint{5.153750in}{3.719100in}}%
\pgfusepath{clip}%
\pgfsetrectcap%
\pgfsetroundjoin%
\pgfsetlinewidth{0.803000pt}%
\definecolor{currentstroke}{rgb}{0.690196,0.690196,0.690196}%
\pgfsetstrokecolor{currentstroke}%
\pgfsetdash{}{0pt}%
\pgfpathmoveto{\pgfqpoint{2.051875in}{0.531300in}}%
\pgfpathlineto{\pgfqpoint{2.051875in}{4.250400in}}%
\pgfusepath{stroke}%
\end{pgfscope}%
\begin{pgfscope}%
\pgfsetbuttcap%
\pgfsetroundjoin%
\definecolor{currentfill}{rgb}{0.000000,0.000000,0.000000}%
\pgfsetfillcolor{currentfill}%
\pgfsetlinewidth{0.803000pt}%
\definecolor{currentstroke}{rgb}{0.000000,0.000000,0.000000}%
\pgfsetstrokecolor{currentstroke}%
\pgfsetdash{}{0pt}%
\pgfsys@defobject{currentmarker}{\pgfqpoint{0.000000in}{-0.048611in}}{\pgfqpoint{0.000000in}{0.000000in}}{%
\pgfpathmoveto{\pgfqpoint{0.000000in}{0.000000in}}%
\pgfpathlineto{\pgfqpoint{0.000000in}{-0.048611in}}%
\pgfusepath{stroke,fill}%
}%
\begin{pgfscope}%
\pgfsys@transformshift{2.051875in}{0.531300in}%
\pgfsys@useobject{currentmarker}{}%
\end{pgfscope}%
\end{pgfscope}%
\begin{pgfscope}%
\definecolor{textcolor}{rgb}{0.000000,0.000000,0.000000}%
\pgfsetstrokecolor{textcolor}%
\pgfsetfillcolor{textcolor}%
\pgftext[x=2.051875in,y=0.434078in,,top]{\color{textcolor}{\rmfamily\fontsize{10.000000}{12.000000}\selectfont\catcode`\^=\active\def^{\ifmmode\sp\else\^{}\fi}\catcode`\%=\active\def%{\%}$\mathdefault{0.50}$}}%
\end{pgfscope}%
\begin{pgfscope}%
\pgfpathrectangle{\pgfqpoint{0.831250in}{0.531300in}}{\pgfqpoint{5.153750in}{3.719100in}}%
\pgfusepath{clip}%
\pgfsetrectcap%
\pgfsetroundjoin%
\pgfsetlinewidth{0.803000pt}%
\definecolor{currentstroke}{rgb}{0.690196,0.690196,0.690196}%
\pgfsetstrokecolor{currentstroke}%
\pgfsetdash{}{0pt}%
\pgfpathmoveto{\pgfqpoint{2.668352in}{0.531300in}}%
\pgfpathlineto{\pgfqpoint{2.668352in}{4.250400in}}%
\pgfusepath{stroke}%
\end{pgfscope}%
\begin{pgfscope}%
\pgfsetbuttcap%
\pgfsetroundjoin%
\definecolor{currentfill}{rgb}{0.000000,0.000000,0.000000}%
\pgfsetfillcolor{currentfill}%
\pgfsetlinewidth{0.803000pt}%
\definecolor{currentstroke}{rgb}{0.000000,0.000000,0.000000}%
\pgfsetstrokecolor{currentstroke}%
\pgfsetdash{}{0pt}%
\pgfsys@defobject{currentmarker}{\pgfqpoint{0.000000in}{-0.048611in}}{\pgfqpoint{0.000000in}{0.000000in}}{%
\pgfpathmoveto{\pgfqpoint{0.000000in}{0.000000in}}%
\pgfpathlineto{\pgfqpoint{0.000000in}{-0.048611in}}%
\pgfusepath{stroke,fill}%
}%
\begin{pgfscope}%
\pgfsys@transformshift{2.668352in}{0.531300in}%
\pgfsys@useobject{currentmarker}{}%
\end{pgfscope}%
\end{pgfscope}%
\begin{pgfscope}%
\definecolor{textcolor}{rgb}{0.000000,0.000000,0.000000}%
\pgfsetstrokecolor{textcolor}%
\pgfsetfillcolor{textcolor}%
\pgftext[x=2.668352in,y=0.434078in,,top]{\color{textcolor}{\rmfamily\fontsize{10.000000}{12.000000}\selectfont\catcode`\^=\active\def^{\ifmmode\sp\else\^{}\fi}\catcode`\%=\active\def%{\%}$\mathdefault{0.75}$}}%
\end{pgfscope}%
\begin{pgfscope}%
\pgfpathrectangle{\pgfqpoint{0.831250in}{0.531300in}}{\pgfqpoint{5.153750in}{3.719100in}}%
\pgfusepath{clip}%
\pgfsetrectcap%
\pgfsetroundjoin%
\pgfsetlinewidth{0.803000pt}%
\definecolor{currentstroke}{rgb}{0.690196,0.690196,0.690196}%
\pgfsetstrokecolor{currentstroke}%
\pgfsetdash{}{0pt}%
\pgfpathmoveto{\pgfqpoint{3.284830in}{0.531300in}}%
\pgfpathlineto{\pgfqpoint{3.284830in}{4.250400in}}%
\pgfusepath{stroke}%
\end{pgfscope}%
\begin{pgfscope}%
\pgfsetbuttcap%
\pgfsetroundjoin%
\definecolor{currentfill}{rgb}{0.000000,0.000000,0.000000}%
\pgfsetfillcolor{currentfill}%
\pgfsetlinewidth{0.803000pt}%
\definecolor{currentstroke}{rgb}{0.000000,0.000000,0.000000}%
\pgfsetstrokecolor{currentstroke}%
\pgfsetdash{}{0pt}%
\pgfsys@defobject{currentmarker}{\pgfqpoint{0.000000in}{-0.048611in}}{\pgfqpoint{0.000000in}{0.000000in}}{%
\pgfpathmoveto{\pgfqpoint{0.000000in}{0.000000in}}%
\pgfpathlineto{\pgfqpoint{0.000000in}{-0.048611in}}%
\pgfusepath{stroke,fill}%
}%
\begin{pgfscope}%
\pgfsys@transformshift{3.284830in}{0.531300in}%
\pgfsys@useobject{currentmarker}{}%
\end{pgfscope}%
\end{pgfscope}%
\begin{pgfscope}%
\definecolor{textcolor}{rgb}{0.000000,0.000000,0.000000}%
\pgfsetstrokecolor{textcolor}%
\pgfsetfillcolor{textcolor}%
\pgftext[x=3.284830in,y=0.434078in,,top]{\color{textcolor}{\rmfamily\fontsize{10.000000}{12.000000}\selectfont\catcode`\^=\active\def^{\ifmmode\sp\else\^{}\fi}\catcode`\%=\active\def%{\%}$\mathdefault{1.00}$}}%
\end{pgfscope}%
\begin{pgfscope}%
\pgfpathrectangle{\pgfqpoint{0.831250in}{0.531300in}}{\pgfqpoint{5.153750in}{3.719100in}}%
\pgfusepath{clip}%
\pgfsetrectcap%
\pgfsetroundjoin%
\pgfsetlinewidth{0.803000pt}%
\definecolor{currentstroke}{rgb}{0.690196,0.690196,0.690196}%
\pgfsetstrokecolor{currentstroke}%
\pgfsetdash{}{0pt}%
\pgfpathmoveto{\pgfqpoint{3.901307in}{0.531300in}}%
\pgfpathlineto{\pgfqpoint{3.901307in}{4.250400in}}%
\pgfusepath{stroke}%
\end{pgfscope}%
\begin{pgfscope}%
\pgfsetbuttcap%
\pgfsetroundjoin%
\definecolor{currentfill}{rgb}{0.000000,0.000000,0.000000}%
\pgfsetfillcolor{currentfill}%
\pgfsetlinewidth{0.803000pt}%
\definecolor{currentstroke}{rgb}{0.000000,0.000000,0.000000}%
\pgfsetstrokecolor{currentstroke}%
\pgfsetdash{}{0pt}%
\pgfsys@defobject{currentmarker}{\pgfqpoint{0.000000in}{-0.048611in}}{\pgfqpoint{0.000000in}{0.000000in}}{%
\pgfpathmoveto{\pgfqpoint{0.000000in}{0.000000in}}%
\pgfpathlineto{\pgfqpoint{0.000000in}{-0.048611in}}%
\pgfusepath{stroke,fill}%
}%
\begin{pgfscope}%
\pgfsys@transformshift{3.901307in}{0.531300in}%
\pgfsys@useobject{currentmarker}{}%
\end{pgfscope}%
\end{pgfscope}%
\begin{pgfscope}%
\definecolor{textcolor}{rgb}{0.000000,0.000000,0.000000}%
\pgfsetstrokecolor{textcolor}%
\pgfsetfillcolor{textcolor}%
\pgftext[x=3.901307in,y=0.434078in,,top]{\color{textcolor}{\rmfamily\fontsize{10.000000}{12.000000}\selectfont\catcode`\^=\active\def^{\ifmmode\sp\else\^{}\fi}\catcode`\%=\active\def%{\%}$\mathdefault{1.25}$}}%
\end{pgfscope}%
\begin{pgfscope}%
\pgfpathrectangle{\pgfqpoint{0.831250in}{0.531300in}}{\pgfqpoint{5.153750in}{3.719100in}}%
\pgfusepath{clip}%
\pgfsetrectcap%
\pgfsetroundjoin%
\pgfsetlinewidth{0.803000pt}%
\definecolor{currentstroke}{rgb}{0.690196,0.690196,0.690196}%
\pgfsetstrokecolor{currentstroke}%
\pgfsetdash{}{0pt}%
\pgfpathmoveto{\pgfqpoint{4.517784in}{0.531300in}}%
\pgfpathlineto{\pgfqpoint{4.517784in}{4.250400in}}%
\pgfusepath{stroke}%
\end{pgfscope}%
\begin{pgfscope}%
\pgfsetbuttcap%
\pgfsetroundjoin%
\definecolor{currentfill}{rgb}{0.000000,0.000000,0.000000}%
\pgfsetfillcolor{currentfill}%
\pgfsetlinewidth{0.803000pt}%
\definecolor{currentstroke}{rgb}{0.000000,0.000000,0.000000}%
\pgfsetstrokecolor{currentstroke}%
\pgfsetdash{}{0pt}%
\pgfsys@defobject{currentmarker}{\pgfqpoint{0.000000in}{-0.048611in}}{\pgfqpoint{0.000000in}{0.000000in}}{%
\pgfpathmoveto{\pgfqpoint{0.000000in}{0.000000in}}%
\pgfpathlineto{\pgfqpoint{0.000000in}{-0.048611in}}%
\pgfusepath{stroke,fill}%
}%
\begin{pgfscope}%
\pgfsys@transformshift{4.517784in}{0.531300in}%
\pgfsys@useobject{currentmarker}{}%
\end{pgfscope}%
\end{pgfscope}%
\begin{pgfscope}%
\definecolor{textcolor}{rgb}{0.000000,0.000000,0.000000}%
\pgfsetstrokecolor{textcolor}%
\pgfsetfillcolor{textcolor}%
\pgftext[x=4.517784in,y=0.434078in,,top]{\color{textcolor}{\rmfamily\fontsize{10.000000}{12.000000}\selectfont\catcode`\^=\active\def^{\ifmmode\sp\else\^{}\fi}\catcode`\%=\active\def%{\%}$\mathdefault{1.50}$}}%
\end{pgfscope}%
\begin{pgfscope}%
\pgfpathrectangle{\pgfqpoint{0.831250in}{0.531300in}}{\pgfqpoint{5.153750in}{3.719100in}}%
\pgfusepath{clip}%
\pgfsetrectcap%
\pgfsetroundjoin%
\pgfsetlinewidth{0.803000pt}%
\definecolor{currentstroke}{rgb}{0.690196,0.690196,0.690196}%
\pgfsetstrokecolor{currentstroke}%
\pgfsetdash{}{0pt}%
\pgfpathmoveto{\pgfqpoint{5.134261in}{0.531300in}}%
\pgfpathlineto{\pgfqpoint{5.134261in}{4.250400in}}%
\pgfusepath{stroke}%
\end{pgfscope}%
\begin{pgfscope}%
\pgfsetbuttcap%
\pgfsetroundjoin%
\definecolor{currentfill}{rgb}{0.000000,0.000000,0.000000}%
\pgfsetfillcolor{currentfill}%
\pgfsetlinewidth{0.803000pt}%
\definecolor{currentstroke}{rgb}{0.000000,0.000000,0.000000}%
\pgfsetstrokecolor{currentstroke}%
\pgfsetdash{}{0pt}%
\pgfsys@defobject{currentmarker}{\pgfqpoint{0.000000in}{-0.048611in}}{\pgfqpoint{0.000000in}{0.000000in}}{%
\pgfpathmoveto{\pgfqpoint{0.000000in}{0.000000in}}%
\pgfpathlineto{\pgfqpoint{0.000000in}{-0.048611in}}%
\pgfusepath{stroke,fill}%
}%
\begin{pgfscope}%
\pgfsys@transformshift{5.134261in}{0.531300in}%
\pgfsys@useobject{currentmarker}{}%
\end{pgfscope}%
\end{pgfscope}%
\begin{pgfscope}%
\definecolor{textcolor}{rgb}{0.000000,0.000000,0.000000}%
\pgfsetstrokecolor{textcolor}%
\pgfsetfillcolor{textcolor}%
\pgftext[x=5.134261in,y=0.434078in,,top]{\color{textcolor}{\rmfamily\fontsize{10.000000}{12.000000}\selectfont\catcode`\^=\active\def^{\ifmmode\sp\else\^{}\fi}\catcode`\%=\active\def%{\%}$\mathdefault{1.75}$}}%
\end{pgfscope}%
\begin{pgfscope}%
\pgfpathrectangle{\pgfqpoint{0.831250in}{0.531300in}}{\pgfqpoint{5.153750in}{3.719100in}}%
\pgfusepath{clip}%
\pgfsetrectcap%
\pgfsetroundjoin%
\pgfsetlinewidth{0.803000pt}%
\definecolor{currentstroke}{rgb}{0.690196,0.690196,0.690196}%
\pgfsetstrokecolor{currentstroke}%
\pgfsetdash{}{0pt}%
\pgfpathmoveto{\pgfqpoint{5.750739in}{0.531300in}}%
\pgfpathlineto{\pgfqpoint{5.750739in}{4.250400in}}%
\pgfusepath{stroke}%
\end{pgfscope}%
\begin{pgfscope}%
\pgfsetbuttcap%
\pgfsetroundjoin%
\definecolor{currentfill}{rgb}{0.000000,0.000000,0.000000}%
\pgfsetfillcolor{currentfill}%
\pgfsetlinewidth{0.803000pt}%
\definecolor{currentstroke}{rgb}{0.000000,0.000000,0.000000}%
\pgfsetstrokecolor{currentstroke}%
\pgfsetdash{}{0pt}%
\pgfsys@defobject{currentmarker}{\pgfqpoint{0.000000in}{-0.048611in}}{\pgfqpoint{0.000000in}{0.000000in}}{%
\pgfpathmoveto{\pgfqpoint{0.000000in}{0.000000in}}%
\pgfpathlineto{\pgfqpoint{0.000000in}{-0.048611in}}%
\pgfusepath{stroke,fill}%
}%
\begin{pgfscope}%
\pgfsys@transformshift{5.750739in}{0.531300in}%
\pgfsys@useobject{currentmarker}{}%
\end{pgfscope}%
\end{pgfscope}%
\begin{pgfscope}%
\definecolor{textcolor}{rgb}{0.000000,0.000000,0.000000}%
\pgfsetstrokecolor{textcolor}%
\pgfsetfillcolor{textcolor}%
\pgftext[x=5.750739in,y=0.434078in,,top]{\color{textcolor}{\rmfamily\fontsize{10.000000}{12.000000}\selectfont\catcode`\^=\active\def^{\ifmmode\sp\else\^{}\fi}\catcode`\%=\active\def%{\%}$\mathdefault{2.00}$}}%
\end{pgfscope}%
\begin{pgfscope}%
\definecolor{textcolor}{rgb}{0.000000,0.000000,0.000000}%
\pgfsetstrokecolor{textcolor}%
\pgfsetfillcolor{textcolor}%
\pgftext[x=3.408125in,y=0.244109in,,top]{\color{textcolor}{\rmfamily\fontsize{14.000000}{16.800000}\selectfont\catcode`\^=\active\def^{\ifmmode\sp\else\^{}\fi}\catcode`\%=\active\def%{\%}$P/\mathrm{Pa}^{-1}$}}%
\end{pgfscope}%
\begin{pgfscope}%
\definecolor{textcolor}{rgb}{0.000000,0.000000,0.000000}%
\pgfsetstrokecolor{textcolor}%
\pgfsetfillcolor{textcolor}%
\pgftext[x=5.985000in,y=0.257998in,right,top]{\color{textcolor}{\rmfamily\fontsize{10.000000}{12.000000}\selectfont\catcode`\^=\active\def^{\ifmmode\sp\else\^{}\fi}\catcode`\%=\active\def%{\%}$\times\mathdefault{10^{6}}\mathdefault{}$}}%
\end{pgfscope}%
\begin{pgfscope}%
\pgfpathrectangle{\pgfqpoint{0.831250in}{0.531300in}}{\pgfqpoint{5.153750in}{3.719100in}}%
\pgfusepath{clip}%
\pgfsetrectcap%
\pgfsetroundjoin%
\pgfsetlinewidth{0.803000pt}%
\definecolor{currentstroke}{rgb}{0.690196,0.690196,0.690196}%
\pgfsetstrokecolor{currentstroke}%
\pgfsetdash{}{0pt}%
\pgfpathmoveto{\pgfqpoint{0.831250in}{1.057456in}}%
\pgfpathlineto{\pgfqpoint{5.985000in}{1.057456in}}%
\pgfusepath{stroke}%
\end{pgfscope}%
\begin{pgfscope}%
\pgfsetbuttcap%
\pgfsetroundjoin%
\definecolor{currentfill}{rgb}{0.000000,0.000000,0.000000}%
\pgfsetfillcolor{currentfill}%
\pgfsetlinewidth{0.803000pt}%
\definecolor{currentstroke}{rgb}{0.000000,0.000000,0.000000}%
\pgfsetstrokecolor{currentstroke}%
\pgfsetdash{}{0pt}%
\pgfsys@defobject{currentmarker}{\pgfqpoint{-0.048611in}{0.000000in}}{\pgfqpoint{-0.000000in}{0.000000in}}{%
\pgfpathmoveto{\pgfqpoint{-0.000000in}{0.000000in}}%
\pgfpathlineto{\pgfqpoint{-0.048611in}{0.000000in}}%
\pgfusepath{stroke,fill}%
}%
\begin{pgfscope}%
\pgfsys@transformshift{0.831250in}{1.057456in}%
\pgfsys@useobject{currentmarker}{}%
\end{pgfscope}%
\end{pgfscope}%
\begin{pgfscope}%
\definecolor{textcolor}{rgb}{0.000000,0.000000,0.000000}%
\pgfsetstrokecolor{textcolor}%
\pgfsetfillcolor{textcolor}%
\pgftext[x=0.487113in, y=1.004695in, left, base]{\color{textcolor}{\rmfamily\fontsize{10.000000}{12.000000}\selectfont\catcode`\^=\active\def^{\ifmmode\sp\else\^{}\fi}\catcode`\%=\active\def%{\%}$\mathdefault{0.01}$}}%
\end{pgfscope}%
\begin{pgfscope}%
\pgfpathrectangle{\pgfqpoint{0.831250in}{0.531300in}}{\pgfqpoint{5.153750in}{3.719100in}}%
\pgfusepath{clip}%
\pgfsetrectcap%
\pgfsetroundjoin%
\pgfsetlinewidth{0.803000pt}%
\definecolor{currentstroke}{rgb}{0.690196,0.690196,0.690196}%
\pgfsetstrokecolor{currentstroke}%
\pgfsetdash{}{0pt}%
\pgfpathmoveto{\pgfqpoint{0.831250in}{1.592510in}}%
\pgfpathlineto{\pgfqpoint{5.985000in}{1.592510in}}%
\pgfusepath{stroke}%
\end{pgfscope}%
\begin{pgfscope}%
\pgfsetbuttcap%
\pgfsetroundjoin%
\definecolor{currentfill}{rgb}{0.000000,0.000000,0.000000}%
\pgfsetfillcolor{currentfill}%
\pgfsetlinewidth{0.803000pt}%
\definecolor{currentstroke}{rgb}{0.000000,0.000000,0.000000}%
\pgfsetstrokecolor{currentstroke}%
\pgfsetdash{}{0pt}%
\pgfsys@defobject{currentmarker}{\pgfqpoint{-0.048611in}{0.000000in}}{\pgfqpoint{-0.000000in}{0.000000in}}{%
\pgfpathmoveto{\pgfqpoint{-0.000000in}{0.000000in}}%
\pgfpathlineto{\pgfqpoint{-0.048611in}{0.000000in}}%
\pgfusepath{stroke,fill}%
}%
\begin{pgfscope}%
\pgfsys@transformshift{0.831250in}{1.592510in}%
\pgfsys@useobject{currentmarker}{}%
\end{pgfscope}%
\end{pgfscope}%
\begin{pgfscope}%
\definecolor{textcolor}{rgb}{0.000000,0.000000,0.000000}%
\pgfsetstrokecolor{textcolor}%
\pgfsetfillcolor{textcolor}%
\pgftext[x=0.487113in, y=1.539749in, left, base]{\color{textcolor}{\rmfamily\fontsize{10.000000}{12.000000}\selectfont\catcode`\^=\active\def^{\ifmmode\sp\else\^{}\fi}\catcode`\%=\active\def%{\%}$\mathdefault{0.02}$}}%
\end{pgfscope}%
\begin{pgfscope}%
\pgfpathrectangle{\pgfqpoint{0.831250in}{0.531300in}}{\pgfqpoint{5.153750in}{3.719100in}}%
\pgfusepath{clip}%
\pgfsetrectcap%
\pgfsetroundjoin%
\pgfsetlinewidth{0.803000pt}%
\definecolor{currentstroke}{rgb}{0.690196,0.690196,0.690196}%
\pgfsetstrokecolor{currentstroke}%
\pgfsetdash{}{0pt}%
\pgfpathmoveto{\pgfqpoint{0.831250in}{2.127564in}}%
\pgfpathlineto{\pgfqpoint{5.985000in}{2.127564in}}%
\pgfusepath{stroke}%
\end{pgfscope}%
\begin{pgfscope}%
\pgfsetbuttcap%
\pgfsetroundjoin%
\definecolor{currentfill}{rgb}{0.000000,0.000000,0.000000}%
\pgfsetfillcolor{currentfill}%
\pgfsetlinewidth{0.803000pt}%
\definecolor{currentstroke}{rgb}{0.000000,0.000000,0.000000}%
\pgfsetstrokecolor{currentstroke}%
\pgfsetdash{}{0pt}%
\pgfsys@defobject{currentmarker}{\pgfqpoint{-0.048611in}{0.000000in}}{\pgfqpoint{-0.000000in}{0.000000in}}{%
\pgfpathmoveto{\pgfqpoint{-0.000000in}{0.000000in}}%
\pgfpathlineto{\pgfqpoint{-0.048611in}{0.000000in}}%
\pgfusepath{stroke,fill}%
}%
\begin{pgfscope}%
\pgfsys@transformshift{0.831250in}{2.127564in}%
\pgfsys@useobject{currentmarker}{}%
\end{pgfscope}%
\end{pgfscope}%
\begin{pgfscope}%
\definecolor{textcolor}{rgb}{0.000000,0.000000,0.000000}%
\pgfsetstrokecolor{textcolor}%
\pgfsetfillcolor{textcolor}%
\pgftext[x=0.487113in, y=2.074802in, left, base]{\color{textcolor}{\rmfamily\fontsize{10.000000}{12.000000}\selectfont\catcode`\^=\active\def^{\ifmmode\sp\else\^{}\fi}\catcode`\%=\active\def%{\%}$\mathdefault{0.03}$}}%
\end{pgfscope}%
\begin{pgfscope}%
\pgfpathrectangle{\pgfqpoint{0.831250in}{0.531300in}}{\pgfqpoint{5.153750in}{3.719100in}}%
\pgfusepath{clip}%
\pgfsetrectcap%
\pgfsetroundjoin%
\pgfsetlinewidth{0.803000pt}%
\definecolor{currentstroke}{rgb}{0.690196,0.690196,0.690196}%
\pgfsetstrokecolor{currentstroke}%
\pgfsetdash{}{0pt}%
\pgfpathmoveto{\pgfqpoint{0.831250in}{2.662618in}}%
\pgfpathlineto{\pgfqpoint{5.985000in}{2.662618in}}%
\pgfusepath{stroke}%
\end{pgfscope}%
\begin{pgfscope}%
\pgfsetbuttcap%
\pgfsetroundjoin%
\definecolor{currentfill}{rgb}{0.000000,0.000000,0.000000}%
\pgfsetfillcolor{currentfill}%
\pgfsetlinewidth{0.803000pt}%
\definecolor{currentstroke}{rgb}{0.000000,0.000000,0.000000}%
\pgfsetstrokecolor{currentstroke}%
\pgfsetdash{}{0pt}%
\pgfsys@defobject{currentmarker}{\pgfqpoint{-0.048611in}{0.000000in}}{\pgfqpoint{-0.000000in}{0.000000in}}{%
\pgfpathmoveto{\pgfqpoint{-0.000000in}{0.000000in}}%
\pgfpathlineto{\pgfqpoint{-0.048611in}{0.000000in}}%
\pgfusepath{stroke,fill}%
}%
\begin{pgfscope}%
\pgfsys@transformshift{0.831250in}{2.662618in}%
\pgfsys@useobject{currentmarker}{}%
\end{pgfscope}%
\end{pgfscope}%
\begin{pgfscope}%
\definecolor{textcolor}{rgb}{0.000000,0.000000,0.000000}%
\pgfsetstrokecolor{textcolor}%
\pgfsetfillcolor{textcolor}%
\pgftext[x=0.487113in, y=2.609856in, left, base]{\color{textcolor}{\rmfamily\fontsize{10.000000}{12.000000}\selectfont\catcode`\^=\active\def^{\ifmmode\sp\else\^{}\fi}\catcode`\%=\active\def%{\%}$\mathdefault{0.04}$}}%
\end{pgfscope}%
\begin{pgfscope}%
\pgfpathrectangle{\pgfqpoint{0.831250in}{0.531300in}}{\pgfqpoint{5.153750in}{3.719100in}}%
\pgfusepath{clip}%
\pgfsetrectcap%
\pgfsetroundjoin%
\pgfsetlinewidth{0.803000pt}%
\definecolor{currentstroke}{rgb}{0.690196,0.690196,0.690196}%
\pgfsetstrokecolor{currentstroke}%
\pgfsetdash{}{0pt}%
\pgfpathmoveto{\pgfqpoint{0.831250in}{3.197671in}}%
\pgfpathlineto{\pgfqpoint{5.985000in}{3.197671in}}%
\pgfusepath{stroke}%
\end{pgfscope}%
\begin{pgfscope}%
\pgfsetbuttcap%
\pgfsetroundjoin%
\definecolor{currentfill}{rgb}{0.000000,0.000000,0.000000}%
\pgfsetfillcolor{currentfill}%
\pgfsetlinewidth{0.803000pt}%
\definecolor{currentstroke}{rgb}{0.000000,0.000000,0.000000}%
\pgfsetstrokecolor{currentstroke}%
\pgfsetdash{}{0pt}%
\pgfsys@defobject{currentmarker}{\pgfqpoint{-0.048611in}{0.000000in}}{\pgfqpoint{-0.000000in}{0.000000in}}{%
\pgfpathmoveto{\pgfqpoint{-0.000000in}{0.000000in}}%
\pgfpathlineto{\pgfqpoint{-0.048611in}{0.000000in}}%
\pgfusepath{stroke,fill}%
}%
\begin{pgfscope}%
\pgfsys@transformshift{0.831250in}{3.197671in}%
\pgfsys@useobject{currentmarker}{}%
\end{pgfscope}%
\end{pgfscope}%
\begin{pgfscope}%
\definecolor{textcolor}{rgb}{0.000000,0.000000,0.000000}%
\pgfsetstrokecolor{textcolor}%
\pgfsetfillcolor{textcolor}%
\pgftext[x=0.487113in, y=3.144910in, left, base]{\color{textcolor}{\rmfamily\fontsize{10.000000}{12.000000}\selectfont\catcode`\^=\active\def^{\ifmmode\sp\else\^{}\fi}\catcode`\%=\active\def%{\%}$\mathdefault{0.05}$}}%
\end{pgfscope}%
\begin{pgfscope}%
\pgfpathrectangle{\pgfqpoint{0.831250in}{0.531300in}}{\pgfqpoint{5.153750in}{3.719100in}}%
\pgfusepath{clip}%
\pgfsetrectcap%
\pgfsetroundjoin%
\pgfsetlinewidth{0.803000pt}%
\definecolor{currentstroke}{rgb}{0.690196,0.690196,0.690196}%
\pgfsetstrokecolor{currentstroke}%
\pgfsetdash{}{0pt}%
\pgfpathmoveto{\pgfqpoint{0.831250in}{3.732725in}}%
\pgfpathlineto{\pgfqpoint{5.985000in}{3.732725in}}%
\pgfusepath{stroke}%
\end{pgfscope}%
\begin{pgfscope}%
\pgfsetbuttcap%
\pgfsetroundjoin%
\definecolor{currentfill}{rgb}{0.000000,0.000000,0.000000}%
\pgfsetfillcolor{currentfill}%
\pgfsetlinewidth{0.803000pt}%
\definecolor{currentstroke}{rgb}{0.000000,0.000000,0.000000}%
\pgfsetstrokecolor{currentstroke}%
\pgfsetdash{}{0pt}%
\pgfsys@defobject{currentmarker}{\pgfqpoint{-0.048611in}{0.000000in}}{\pgfqpoint{-0.000000in}{0.000000in}}{%
\pgfpathmoveto{\pgfqpoint{-0.000000in}{0.000000in}}%
\pgfpathlineto{\pgfqpoint{-0.048611in}{0.000000in}}%
\pgfusepath{stroke,fill}%
}%
\begin{pgfscope}%
\pgfsys@transformshift{0.831250in}{3.732725in}%
\pgfsys@useobject{currentmarker}{}%
\end{pgfscope}%
\end{pgfscope}%
\begin{pgfscope}%
\definecolor{textcolor}{rgb}{0.000000,0.000000,0.000000}%
\pgfsetstrokecolor{textcolor}%
\pgfsetfillcolor{textcolor}%
\pgftext[x=0.487113in, y=3.679963in, left, base]{\color{textcolor}{\rmfamily\fontsize{10.000000}{12.000000}\selectfont\catcode`\^=\active\def^{\ifmmode\sp\else\^{}\fi}\catcode`\%=\active\def%{\%}$\mathdefault{0.06}$}}%
\end{pgfscope}%
\begin{pgfscope}%
\definecolor{textcolor}{rgb}{0.000000,0.000000,0.000000}%
\pgfsetstrokecolor{textcolor}%
\pgfsetfillcolor{textcolor}%
\pgftext[x=0.431558in,y=2.390850in,,bottom,rotate=90.000000]{\color{textcolor}{\rmfamily\fontsize{14.000000}{16.800000}\selectfont\catcode`\^=\active\def^{\ifmmode\sp\else\^{}\fi}\catcode`\%=\active\def%{\%}$V/\mathrm{m}^3\,\mathrm{mol}^{-1}$}}%
\end{pgfscope}%
\begin{pgfscope}%
\pgfpathrectangle{\pgfqpoint{0.831250in}{0.531300in}}{\pgfqpoint{5.153750in}{3.719100in}}%
\pgfusepath{clip}%
\pgfsetrectcap%
\pgfsetroundjoin%
\pgfsetlinewidth{1.505625pt}%
\definecolor{currentstroke}{rgb}{0.000000,0.000000,1.000000}%
\pgfsetstrokecolor{currentstroke}%
\pgfsetdash{}{0pt}%
\pgfpathmoveto{\pgfqpoint{1.065511in}{1.115561in}}%
\pgfpathlineto{\pgfqpoint{5.750739in}{1.115561in}}%
\pgfpathlineto{\pgfqpoint{5.750739in}{1.115561in}}%
\pgfusepath{stroke}%
\end{pgfscope}%
\begin{pgfscope}%
\pgfpathrectangle{\pgfqpoint{0.831250in}{0.531300in}}{\pgfqpoint{5.153750in}{3.719100in}}%
\pgfusepath{clip}%
\pgfsetrectcap%
\pgfsetroundjoin%
\pgfsetlinewidth{1.505625pt}%
\definecolor{currentstroke}{rgb}{0.000000,0.501961,0.000000}%
\pgfsetstrokecolor{currentstroke}%
\pgfsetdash{}{0pt}%
\pgfpathmoveto{\pgfqpoint{1.065511in}{4.081350in}}%
\pgfpathlineto{\pgfqpoint{1.084290in}{3.829506in}}%
\pgfpathlineto{\pgfqpoint{1.103068in}{3.610950in}}%
\pgfpathlineto{\pgfqpoint{1.121847in}{3.419490in}}%
\pgfpathlineto{\pgfqpoint{1.140625in}{3.250382in}}%
\pgfpathlineto{\pgfqpoint{1.159404in}{3.099928in}}%
\pgfpathlineto{\pgfqpoint{1.178182in}{2.965201in}}%
\pgfpathlineto{\pgfqpoint{1.196961in}{2.843860in}}%
\pgfpathlineto{\pgfqpoint{1.215739in}{2.734003in}}%
\pgfpathlineto{\pgfqpoint{1.234518in}{2.634073in}}%
\pgfpathlineto{\pgfqpoint{1.253296in}{2.542783in}}%
\pgfpathlineto{\pgfqpoint{1.272074in}{2.459060in}}%
\pgfpathlineto{\pgfqpoint{1.300242in}{2.345724in}}%
\pgfpathlineto{\pgfqpoint{1.328410in}{2.244919in}}%
\pgfpathlineto{\pgfqpoint{1.356578in}{2.154677in}}%
\pgfpathlineto{\pgfqpoint{1.384745in}{2.073420in}}%
\pgfpathlineto{\pgfqpoint{1.412913in}{1.999869in}}%
\pgfpathlineto{\pgfqpoint{1.441081in}{1.932978in}}%
\pgfpathlineto{\pgfqpoint{1.469248in}{1.871882in}}%
\pgfpathlineto{\pgfqpoint{1.497416in}{1.815858in}}%
\pgfpathlineto{\pgfqpoint{1.525584in}{1.764301in}}%
\pgfpathlineto{\pgfqpoint{1.553751in}{1.716696in}}%
\pgfpathlineto{\pgfqpoint{1.581919in}{1.672606in}}%
\pgfpathlineto{\pgfqpoint{1.619476in}{1.618646in}}%
\pgfpathlineto{\pgfqpoint{1.657033in}{1.569522in}}%
\pgfpathlineto{\pgfqpoint{1.694590in}{1.524612in}}%
\pgfpathlineto{\pgfqpoint{1.732147in}{1.483395in}}%
\pgfpathlineto{\pgfqpoint{1.769704in}{1.445435in}}%
\pgfpathlineto{\pgfqpoint{1.807261in}{1.410360in}}%
\pgfpathlineto{\pgfqpoint{1.844818in}{1.377853in}}%
\pgfpathlineto{\pgfqpoint{1.891764in}{1.340420in}}%
\pgfpathlineto{\pgfqpoint{1.938710in}{1.306125in}}%
\pgfpathlineto{\pgfqpoint{1.985656in}{1.274590in}}%
\pgfpathlineto{\pgfqpoint{2.032602in}{1.245495in}}%
\pgfpathlineto{\pgfqpoint{2.088938in}{1.213420in}}%
\pgfpathlineto{\pgfqpoint{2.145273in}{1.184070in}}%
\pgfpathlineto{\pgfqpoint{2.201609in}{1.157111in}}%
\pgfpathlineto{\pgfqpoint{2.267333in}{1.128310in}}%
\pgfpathlineto{\pgfqpoint{2.333058in}{1.102009in}}%
\pgfpathlineto{\pgfqpoint{2.408172in}{1.074615in}}%
\pgfpathlineto{\pgfqpoint{2.483286in}{1.049693in}}%
\pgfpathlineto{\pgfqpoint{2.567789in}{1.024215in}}%
\pgfpathlineto{\pgfqpoint{2.652292in}{1.001086in}}%
\pgfpathlineto{\pgfqpoint{2.746184in}{0.977765in}}%
\pgfpathlineto{\pgfqpoint{2.849466in}{0.954604in}}%
\pgfpathlineto{\pgfqpoint{2.952747in}{0.933684in}}%
\pgfpathlineto{\pgfqpoint{3.065418in}{0.913057in}}%
\pgfpathlineto{\pgfqpoint{3.187478in}{0.892925in}}%
\pgfpathlineto{\pgfqpoint{3.318927in}{0.873443in}}%
\pgfpathlineto{\pgfqpoint{3.469155in}{0.853545in}}%
\pgfpathlineto{\pgfqpoint{3.628772in}{0.834734in}}%
\pgfpathlineto{\pgfqpoint{3.807167in}{0.816088in}}%
\pgfpathlineto{\pgfqpoint{3.994952in}{0.798724in}}%
\pgfpathlineto{\pgfqpoint{4.201515in}{0.781850in}}%
\pgfpathlineto{\pgfqpoint{4.426857in}{0.765645in}}%
\pgfpathlineto{\pgfqpoint{4.680366in}{0.749676in}}%
\pgfpathlineto{\pgfqpoint{4.962043in}{0.734225in}}%
\pgfpathlineto{\pgfqpoint{5.271888in}{0.719486in}}%
\pgfpathlineto{\pgfqpoint{5.619289in}{0.705223in}}%
\pgfpathlineto{\pgfqpoint{5.750739in}{0.700350in}}%
\pgfpathlineto{\pgfqpoint{5.750739in}{0.700350in}}%
\pgfusepath{stroke}%
\end{pgfscope}%
\begin{pgfscope}%
\pgfpathrectangle{\pgfqpoint{0.831250in}{0.531300in}}{\pgfqpoint{5.153750in}{3.719100in}}%
\pgfusepath{clip}%
\pgfsetrectcap%
\pgfsetroundjoin%
\pgfsetlinewidth{1.505625pt}%
\definecolor{currentstroke}{rgb}{1.000000,0.000000,0.000000}%
\pgfsetstrokecolor{currentstroke}%
\pgfsetdash{}{0pt}%
\pgfpathmoveto{\pgfqpoint{1.065511in}{2.260448in}}%
\pgfpathlineto{\pgfqpoint{1.084290in}{2.185573in}}%
\pgfpathlineto{\pgfqpoint{1.103068in}{2.118725in}}%
\pgfpathlineto{\pgfqpoint{1.121847in}{2.058593in}}%
\pgfpathlineto{\pgfqpoint{1.140625in}{2.004145in}}%
\pgfpathlineto{\pgfqpoint{1.159404in}{1.954557in}}%
\pgfpathlineto{\pgfqpoint{1.187571in}{1.887860in}}%
\pgfpathlineto{\pgfqpoint{1.215739in}{1.828850in}}%
\pgfpathlineto{\pgfqpoint{1.243907in}{1.776185in}}%
\pgfpathlineto{\pgfqpoint{1.272074in}{1.728826in}}%
\pgfpathlineto{\pgfqpoint{1.300242in}{1.685955in}}%
\pgfpathlineto{\pgfqpoint{1.328410in}{1.646920in}}%
\pgfpathlineto{\pgfqpoint{1.356578in}{1.611192in}}%
\pgfpathlineto{\pgfqpoint{1.384745in}{1.578340in}}%
\pgfpathlineto{\pgfqpoint{1.422302in}{1.538399in}}%
\pgfpathlineto{\pgfqpoint{1.459859in}{1.502248in}}%
\pgfpathlineto{\pgfqpoint{1.497416in}{1.469335in}}%
\pgfpathlineto{\pgfqpoint{1.534973in}{1.439215in}}%
\pgfpathlineto{\pgfqpoint{1.572530in}{1.411521in}}%
\pgfpathlineto{\pgfqpoint{1.619476in}{1.379860in}}%
\pgfpathlineto{\pgfqpoint{1.666422in}{1.351037in}}%
\pgfpathlineto{\pgfqpoint{1.713368in}{1.324661in}}%
\pgfpathlineto{\pgfqpoint{1.769704in}{1.295792in}}%
\pgfpathlineto{\pgfqpoint{1.826039in}{1.269537in}}%
\pgfpathlineto{\pgfqpoint{1.882375in}{1.245532in}}%
\pgfpathlineto{\pgfqpoint{1.948099in}{1.219976in}}%
\pgfpathlineto{\pgfqpoint{2.013824in}{1.196694in}}%
\pgfpathlineto{\pgfqpoint{2.088938in}{1.172475in}}%
\pgfpathlineto{\pgfqpoint{2.173441in}{1.147829in}}%
\pgfpathlineto{\pgfqpoint{2.257944in}{1.125527in}}%
\pgfpathlineto{\pgfqpoint{2.351836in}{1.103082in}}%
\pgfpathlineto{\pgfqpoint{2.455118in}{1.080803in}}%
\pgfpathlineto{\pgfqpoint{2.567789in}{1.058931in}}%
\pgfpathlineto{\pgfqpoint{2.689849in}{1.037647in}}%
\pgfpathlineto{\pgfqpoint{2.821298in}{1.017077in}}%
\pgfpathlineto{\pgfqpoint{2.962136in}{0.997309in}}%
\pgfpathlineto{\pgfqpoint{3.121753in}{0.977275in}}%
\pgfpathlineto{\pgfqpoint{3.290760in}{0.958351in}}%
\pgfpathlineto{\pgfqpoint{3.478544in}{0.939613in}}%
\pgfpathlineto{\pgfqpoint{3.685107in}{0.921303in}}%
\pgfpathlineto{\pgfqpoint{3.910449in}{0.903594in}}%
\pgfpathlineto{\pgfqpoint{4.163958in}{0.885988in}}%
\pgfpathlineto{\pgfqpoint{4.445635in}{0.868772in}}%
\pgfpathlineto{\pgfqpoint{4.755480in}{0.852147in}}%
\pgfpathlineto{\pgfqpoint{5.102882in}{0.835832in}}%
\pgfpathlineto{\pgfqpoint{5.487840in}{0.820061in}}%
\pgfpathlineto{\pgfqpoint{5.750739in}{0.810436in}}%
\pgfpathlineto{\pgfqpoint{5.750739in}{0.810436in}}%
\pgfusepath{stroke}%
\end{pgfscope}%
\begin{pgfscope}%
\pgfsetrectcap%
\pgfsetmiterjoin%
\pgfsetlinewidth{0.803000pt}%
\definecolor{currentstroke}{rgb}{0.000000,0.000000,0.000000}%
\pgfsetstrokecolor{currentstroke}%
\pgfsetdash{}{0pt}%
\pgfpathmoveto{\pgfqpoint{0.831250in}{0.531300in}}%
\pgfpathlineto{\pgfqpoint{0.831250in}{4.250400in}}%
\pgfusepath{stroke}%
\end{pgfscope}%
\begin{pgfscope}%
\pgfsetrectcap%
\pgfsetmiterjoin%
\pgfsetlinewidth{0.803000pt}%
\definecolor{currentstroke}{rgb}{0.000000,0.000000,0.000000}%
\pgfsetstrokecolor{currentstroke}%
\pgfsetdash{}{0pt}%
\pgfpathmoveto{\pgfqpoint{5.985000in}{0.531300in}}%
\pgfpathlineto{\pgfqpoint{5.985000in}{4.250400in}}%
\pgfusepath{stroke}%
\end{pgfscope}%
\begin{pgfscope}%
\pgfsetrectcap%
\pgfsetmiterjoin%
\pgfsetlinewidth{0.803000pt}%
\definecolor{currentstroke}{rgb}{0.000000,0.000000,0.000000}%
\pgfsetstrokecolor{currentstroke}%
\pgfsetdash{}{0pt}%
\pgfpathmoveto{\pgfqpoint{0.831250in}{0.531300in}}%
\pgfpathlineto{\pgfqpoint{5.985000in}{0.531300in}}%
\pgfusepath{stroke}%
\end{pgfscope}%
\begin{pgfscope}%
\pgfsetrectcap%
\pgfsetmiterjoin%
\pgfsetlinewidth{0.803000pt}%
\definecolor{currentstroke}{rgb}{0.000000,0.000000,0.000000}%
\pgfsetstrokecolor{currentstroke}%
\pgfsetdash{}{0pt}%
\pgfpathmoveto{\pgfqpoint{0.831250in}{4.250400in}}%
\pgfpathlineto{\pgfqpoint{5.985000in}{4.250400in}}%
\pgfusepath{stroke}%
\end{pgfscope}%
\begin{pgfscope}%
\pgfsetbuttcap%
\pgfsetmiterjoin%
\definecolor{currentfill}{rgb}{1.000000,1.000000,1.000000}%
\pgfsetfillcolor{currentfill}%
\pgfsetfillopacity{0.800000}%
\pgfsetlinewidth{1.003750pt}%
\definecolor{currentstroke}{rgb}{0.800000,0.800000,0.800000}%
\pgfsetstrokecolor{currentstroke}%
\pgfsetstrokeopacity{0.800000}%
\pgfsetdash{}{0pt}%
\pgfpathmoveto{\pgfqpoint{3.002708in}{3.235890in}}%
\pgfpathlineto{\pgfqpoint{5.848889in}{3.235890in}}%
\pgfpathquadraticcurveto{\pgfqpoint{5.887778in}{3.235890in}}{\pgfqpoint{5.887778in}{3.274779in}}%
\pgfpathlineto{\pgfqpoint{5.887778in}{4.114289in}}%
\pgfpathquadraticcurveto{\pgfqpoint{5.887778in}{4.153178in}}{\pgfqpoint{5.848889in}{4.153178in}}%
\pgfpathlineto{\pgfqpoint{3.002708in}{4.153178in}}%
\pgfpathquadraticcurveto{\pgfqpoint{2.963819in}{4.153178in}}{\pgfqpoint{2.963819in}{4.114289in}}%
\pgfpathlineto{\pgfqpoint{2.963819in}{3.274779in}}%
\pgfpathquadraticcurveto{\pgfqpoint{2.963819in}{3.235890in}}{\pgfqpoint{3.002708in}{3.235890in}}%
\pgfpathlineto{\pgfqpoint{3.002708in}{3.235890in}}%
\pgfpathclose%
\pgfusepath{stroke,fill}%
\end{pgfscope}%
\begin{pgfscope}%
\pgfsetrectcap%
\pgfsetroundjoin%
\pgfsetlinewidth{1.505625pt}%
\definecolor{currentstroke}{rgb}{0.000000,0.000000,1.000000}%
\pgfsetstrokecolor{currentstroke}%
\pgfsetdash{}{0pt}%
\pgfpathmoveto{\pgfqpoint{3.041597in}{3.995723in}}%
\pgfpathlineto{\pgfqpoint{3.236042in}{3.995723in}}%
\pgfpathlineto{\pgfqpoint{3.430486in}{3.995723in}}%
\pgfusepath{stroke}%
\end{pgfscope}%
\begin{pgfscope}%
\definecolor{textcolor}{rgb}{0.000000,0.000000,0.000000}%
\pgfsetstrokecolor{textcolor}%
\pgfsetfillcolor{textcolor}%
\pgftext[x=3.586042in,y=3.927668in,left,base]{\color{textcolor}{\rmfamily\fontsize{14.000000}{16.800000}\selectfont\catcode`\^=\active\def^{\ifmmode\sp\else\^{}\fi}\catcode`\%=\active\def%{\%}Constant Volume}}%
\end{pgfscope}%
\begin{pgfscope}%
\pgfsetrectcap%
\pgfsetroundjoin%
\pgfsetlinewidth{1.505625pt}%
\definecolor{currentstroke}{rgb}{0.000000,0.501961,0.000000}%
\pgfsetstrokecolor{currentstroke}%
\pgfsetdash{}{0pt}%
\pgfpathmoveto{\pgfqpoint{3.041597in}{3.710323in}}%
\pgfpathlineto{\pgfqpoint{3.236042in}{3.710323in}}%
\pgfpathlineto{\pgfqpoint{3.430486in}{3.710323in}}%
\pgfusepath{stroke}%
\end{pgfscope}%
\begin{pgfscope}%
\definecolor{textcolor}{rgb}{0.000000,0.000000,0.000000}%
\pgfsetstrokecolor{textcolor}%
\pgfsetfillcolor{textcolor}%
\pgftext[x=3.586042in,y=3.642268in,left,base]{\color{textcolor}{\rmfamily\fontsize{14.000000}{16.800000}\selectfont\catcode`\^=\active\def^{\ifmmode\sp\else\^{}\fi}\catcode`\%=\active\def%{\%}Constant Temperature}}%
\end{pgfscope}%
\begin{pgfscope}%
\pgfsetrectcap%
\pgfsetroundjoin%
\pgfsetlinewidth{1.505625pt}%
\definecolor{currentstroke}{rgb}{1.000000,0.000000,0.000000}%
\pgfsetstrokecolor{currentstroke}%
\pgfsetdash{}{0pt}%
\pgfpathmoveto{\pgfqpoint{3.041597in}{3.424923in}}%
\pgfpathlineto{\pgfqpoint{3.236042in}{3.424923in}}%
\pgfpathlineto{\pgfqpoint{3.430486in}{3.424923in}}%
\pgfusepath{stroke}%
\end{pgfscope}%
\begin{pgfscope}%
\definecolor{textcolor}{rgb}{0.000000,0.000000,0.000000}%
\pgfsetstrokecolor{textcolor}%
\pgfsetfillcolor{textcolor}%
\pgftext[x=3.586042in,y=3.356867in,left,base]{\color{textcolor}{\rmfamily\fontsize{14.000000}{16.800000}\selectfont\catcode`\^=\active\def^{\ifmmode\sp\else\^{}\fi}\catcode`\%=\active\def%{\%}Adiabatically}}%
\end{pgfscope}%
\end{pgfpicture}%
\makeatother%
\endgroup%
}
    \caption{PV diagram showing the three different processes as
    described in Problem 3.18.}
    \label{fig:s18}
  \end{figure}
\end{solution}

\section*{Problem 3.19}
\addcontentsline{toc}{section}{Problem 3.19}
An ideal gas initially at $\SI{600}{\kelvin}$ and $\SI{10}{\bar}$
undergoes a four-step mechanically reversible cycle in a closed
system. In step $12$, pressure decreases isothermally to
$\SI{3}{\bar}$; in step $23$, pressure decreases at constant volume
to $\SI{2}{\bar}$; in step $34$, volume decreases at constant
pressure; and in step $41$, the gas returns adiabatically to its
initial state. Take $C_P = \frac{7}{2}R$ and $C_V = \frac{5}{2}R$.
\begin{enumerate}[label=(\alph*)]
  \item Sketch the cycle on a $PV$ diagram.
  \item Determine (where unknown) both $T$ and $P$ for states $1$, $2$,
    $3$, and $4$.
  \item Calculate $Q$, $W$, $\Delta U$, and $\Delta H$ for each step
    of the cycle.
\end{enumerate}

\begin{solution}
  \begin{enumerate}[label=(\alph*)]
    \item Figure~\ref{fig:s19}
      \begin{figure}[h!]
        \centering
        \scalebox{0.5}{%% Creator: Matplotlib, PGF backend
%%
%% To include the figure in your LaTeX document, write
%%   \input{<filename>.pgf}
%%
%% Make sure the required packages are loaded in your preamble
%%   \usepackage{pgf}
%%
%% Also ensure that all the required font packages are loaded; for instance,
%% the lmodern package is sometimes necessary when using math font.
%%   \usepackage{lmodern}
%%
%% Figures using additional raster images can only be included by \input if
%% they are in the same directory as the main LaTeX file. For loading figures
%% from other directories you can use the `import` package
%%   \usepackage{import}
%%
%% and then include the figures with
%%   \import{<path to file>}{<filename>.pgf}
%%
%% Matplotlib used the following preamble
%%   \def\mathdefault#1{#1}
%%   \everymath=\expandafter{\the\everymath\displaystyle}
%%   \IfFileExists{scrextend.sty}{
%%     \usepackage[fontsize=10.000000pt]{scrextend}
%%   }{
%%     \renewcommand{\normalsize}{\fontsize{10.000000}{12.000000}\selectfont}
%%     \normalsize
%%   }
%%   
%%   \ifdefined\pdftexversion\else  % non-pdftex case.
%%     \usepackage{fontspec}
%%     \setmainfont{DejaVuSerif.ttf}[Path=\detokenize{C:/Users/Jian/AppData/Local/Programs/Python/Python313/Lib/site-packages/matplotlib/mpl-data/fonts/ttf/}]
%%     \setsansfont{DejaVuSans.ttf}[Path=\detokenize{C:/Users/Jian/AppData/Local/Programs/Python/Python313/Lib/site-packages/matplotlib/mpl-data/fonts/ttf/}]
%%     \setmonofont{DejaVuSansMono.ttf}[Path=\detokenize{C:/Users/Jian/AppData/Local/Programs/Python/Python313/Lib/site-packages/matplotlib/mpl-data/fonts/ttf/}]
%%   \fi
%%   \makeatletter\@ifpackageloaded{underscore}{}{\usepackage[strings]{underscore}}\makeatother
%%
\begingroup%
\makeatletter%
\begin{pgfpicture}%
\pgfpathrectangle{\pgfpointorigin}{\pgfqpoint{7.140000in}{4.850000in}}%
\pgfusepath{use as bounding box, clip}%
\begin{pgfscope}%
\pgfsetbuttcap%
\pgfsetmiterjoin%
\definecolor{currentfill}{rgb}{1.000000,1.000000,1.000000}%
\pgfsetfillcolor{currentfill}%
\pgfsetlinewidth{0.000000pt}%
\definecolor{currentstroke}{rgb}{1.000000,1.000000,1.000000}%
\pgfsetstrokecolor{currentstroke}%
\pgfsetdash{}{0pt}%
\pgfpathmoveto{\pgfqpoint{0.000000in}{0.000000in}}%
\pgfpathlineto{\pgfqpoint{7.140000in}{0.000000in}}%
\pgfpathlineto{\pgfqpoint{7.140000in}{4.850000in}}%
\pgfpathlineto{\pgfqpoint{0.000000in}{4.850000in}}%
\pgfpathlineto{\pgfqpoint{0.000000in}{0.000000in}}%
\pgfpathclose%
\pgfusepath{fill}%
\end{pgfscope}%
\begin{pgfscope}%
\pgfsetbuttcap%
\pgfsetmiterjoin%
\definecolor{currentfill}{rgb}{1.000000,1.000000,1.000000}%
\pgfsetfillcolor{currentfill}%
\pgfsetlinewidth{0.000000pt}%
\definecolor{currentstroke}{rgb}{0.000000,0.000000,0.000000}%
\pgfsetstrokecolor{currentstroke}%
\pgfsetstrokeopacity{0.000000}%
\pgfsetdash{}{0pt}%
\pgfpathmoveto{\pgfqpoint{0.892500in}{0.533500in}}%
\pgfpathlineto{\pgfqpoint{6.426000in}{0.533500in}}%
\pgfpathlineto{\pgfqpoint{6.426000in}{4.268000in}}%
\pgfpathlineto{\pgfqpoint{0.892500in}{4.268000in}}%
\pgfpathlineto{\pgfqpoint{0.892500in}{0.533500in}}%
\pgfpathclose%
\pgfusepath{fill}%
\end{pgfscope}%
\begin{pgfscope}%
\pgfpathrectangle{\pgfqpoint{0.892500in}{0.533500in}}{\pgfqpoint{5.533500in}{3.734500in}}%
\pgfusepath{clip}%
\pgfsetbuttcap%
\pgfsetroundjoin%
\pgfsetlinewidth{1.003750pt}%
\definecolor{currentstroke}{rgb}{0.000000,0.000000,0.000000}%
\pgfsetstrokecolor{currentstroke}%
\pgfsetdash{}{0pt}%
\pgfpathmoveto{\pgfqpoint{6.174477in}{0.681290in}}%
\pgfpathcurveto{\pgfqpoint{6.180301in}{0.681290in}}{\pgfqpoint{6.185887in}{0.683604in}}{\pgfqpoint{6.190006in}{0.687722in}}%
\pgfpathcurveto{\pgfqpoint{6.194124in}{0.691840in}}{\pgfqpoint{6.196438in}{0.697426in}}{\pgfqpoint{6.196438in}{0.703250in}}%
\pgfpathcurveto{\pgfqpoint{6.196438in}{0.709074in}}{\pgfqpoint{6.194124in}{0.714660in}}{\pgfqpoint{6.190006in}{0.718778in}}%
\pgfpathcurveto{\pgfqpoint{6.185887in}{0.722896in}}{\pgfqpoint{6.180301in}{0.725210in}}{\pgfqpoint{6.174477in}{0.725210in}}%
\pgfpathcurveto{\pgfqpoint{6.168653in}{0.725210in}}{\pgfqpoint{6.163067in}{0.722896in}}{\pgfqpoint{6.158949in}{0.718778in}}%
\pgfpathcurveto{\pgfqpoint{6.154831in}{0.714660in}}{\pgfqpoint{6.152517in}{0.709074in}}{\pgfqpoint{6.152517in}{0.703250in}}%
\pgfpathcurveto{\pgfqpoint{6.152517in}{0.697426in}}{\pgfqpoint{6.154831in}{0.691840in}}{\pgfqpoint{6.158949in}{0.687722in}}%
\pgfpathcurveto{\pgfqpoint{6.163067in}{0.683604in}}{\pgfqpoint{6.168653in}{0.681290in}}{\pgfqpoint{6.174477in}{0.681290in}}%
\pgfpathlineto{\pgfqpoint{6.174477in}{0.681290in}}%
\pgfpathclose%
\pgfusepath{stroke}%
\end{pgfscope}%
\begin{pgfscope}%
\pgfpathrectangle{\pgfqpoint{0.892500in}{0.533500in}}{\pgfqpoint{5.533500in}{3.734500in}}%
\pgfusepath{clip}%
\pgfsetbuttcap%
\pgfsetroundjoin%
\pgfsetlinewidth{1.003750pt}%
\definecolor{currentstroke}{rgb}{0.000000,0.000000,0.000000}%
\pgfsetstrokecolor{currentstroke}%
\pgfsetdash{}{0pt}%
\pgfpathmoveto{\pgfqpoint{1.772830in}{4.076290in}}%
\pgfpathcurveto{\pgfqpoint{1.778653in}{4.076290in}}{\pgfqpoint{1.784240in}{4.078604in}}{\pgfqpoint{1.788358in}{4.082722in}}%
\pgfpathcurveto{\pgfqpoint{1.792476in}{4.086840in}}{\pgfqpoint{1.794790in}{4.092426in}}{\pgfqpoint{1.794790in}{4.098250in}}%
\pgfpathcurveto{\pgfqpoint{1.794790in}{4.104074in}}{\pgfqpoint{1.792476in}{4.109660in}}{\pgfqpoint{1.788358in}{4.113778in}}%
\pgfpathcurveto{\pgfqpoint{1.784240in}{4.117896in}}{\pgfqpoint{1.778653in}{4.120210in}}{\pgfqpoint{1.772830in}{4.120210in}}%
\pgfpathcurveto{\pgfqpoint{1.767006in}{4.120210in}}{\pgfqpoint{1.761419in}{4.117896in}}{\pgfqpoint{1.757301in}{4.113778in}}%
\pgfpathcurveto{\pgfqpoint{1.753183in}{4.109660in}}{\pgfqpoint{1.750869in}{4.104074in}}{\pgfqpoint{1.750869in}{4.098250in}}%
\pgfpathcurveto{\pgfqpoint{1.750869in}{4.092426in}}{\pgfqpoint{1.753183in}{4.086840in}}{\pgfqpoint{1.757301in}{4.082722in}}%
\pgfpathcurveto{\pgfqpoint{1.761419in}{4.078604in}}{\pgfqpoint{1.767006in}{4.076290in}}{\pgfqpoint{1.772830in}{4.076290in}}%
\pgfpathlineto{\pgfqpoint{1.772830in}{4.076290in}}%
\pgfpathclose%
\pgfusepath{stroke}%
\end{pgfscope}%
\begin{pgfscope}%
\pgfpathrectangle{\pgfqpoint{0.892500in}{0.533500in}}{\pgfqpoint{5.533500in}{3.734500in}}%
\pgfusepath{clip}%
\pgfsetbuttcap%
\pgfsetroundjoin%
\pgfsetlinewidth{1.003750pt}%
\definecolor{currentstroke}{rgb}{0.000000,0.000000,0.000000}%
\pgfsetstrokecolor{currentstroke}%
\pgfsetdash{}{0pt}%
\pgfpathmoveto{\pgfqpoint{1.144023in}{4.076290in}}%
\pgfpathcurveto{\pgfqpoint{1.149847in}{4.076290in}}{\pgfqpoint{1.155433in}{4.078604in}}{\pgfqpoint{1.159551in}{4.082722in}}%
\pgfpathcurveto{\pgfqpoint{1.163669in}{4.086840in}}{\pgfqpoint{1.165983in}{4.092426in}}{\pgfqpoint{1.165983in}{4.098250in}}%
\pgfpathcurveto{\pgfqpoint{1.165983in}{4.104074in}}{\pgfqpoint{1.163669in}{4.109660in}}{\pgfqpoint{1.159551in}{4.113778in}}%
\pgfpathcurveto{\pgfqpoint{1.155433in}{4.117896in}}{\pgfqpoint{1.149847in}{4.120210in}}{\pgfqpoint{1.144023in}{4.120210in}}%
\pgfpathcurveto{\pgfqpoint{1.138199in}{4.120210in}}{\pgfqpoint{1.132613in}{4.117896in}}{\pgfqpoint{1.128494in}{4.113778in}}%
\pgfpathcurveto{\pgfqpoint{1.124376in}{4.109660in}}{\pgfqpoint{1.122062in}{4.104074in}}{\pgfqpoint{1.122062in}{4.098250in}}%
\pgfpathcurveto{\pgfqpoint{1.122062in}{4.092426in}}{\pgfqpoint{1.124376in}{4.086840in}}{\pgfqpoint{1.128494in}{4.082722in}}%
\pgfpathcurveto{\pgfqpoint{1.132613in}{4.078604in}}{\pgfqpoint{1.138199in}{4.076290in}}{\pgfqpoint{1.144023in}{4.076290in}}%
\pgfpathlineto{\pgfqpoint{1.144023in}{4.076290in}}%
\pgfpathclose%
\pgfusepath{stroke}%
\end{pgfscope}%
\begin{pgfscope}%
\pgfpathrectangle{\pgfqpoint{0.892500in}{0.533500in}}{\pgfqpoint{5.533500in}{3.734500in}}%
\pgfusepath{clip}%
\pgfsetbuttcap%
\pgfsetroundjoin%
\pgfsetlinewidth{1.003750pt}%
\definecolor{currentstroke}{rgb}{0.000000,0.000000,0.000000}%
\pgfsetstrokecolor{currentstroke}%
\pgfsetdash{}{0pt}%
\pgfpathmoveto{\pgfqpoint{1.144023in}{3.819616in}}%
\pgfpathcurveto{\pgfqpoint{1.149847in}{3.819616in}}{\pgfqpoint{1.155433in}{3.821930in}}{\pgfqpoint{1.159551in}{3.826048in}}%
\pgfpathcurveto{\pgfqpoint{1.163669in}{3.830166in}}{\pgfqpoint{1.165983in}{3.835752in}}{\pgfqpoint{1.165983in}{3.841576in}}%
\pgfpathcurveto{\pgfqpoint{1.165983in}{3.847400in}}{\pgfqpoint{1.163669in}{3.852986in}}{\pgfqpoint{1.159551in}{3.857104in}}%
\pgfpathcurveto{\pgfqpoint{1.155433in}{3.861223in}}{\pgfqpoint{1.149847in}{3.863536in}}{\pgfqpoint{1.144023in}{3.863536in}}%
\pgfpathcurveto{\pgfqpoint{1.138199in}{3.863536in}}{\pgfqpoint{1.132613in}{3.861223in}}{\pgfqpoint{1.128494in}{3.857104in}}%
\pgfpathcurveto{\pgfqpoint{1.124376in}{3.852986in}}{\pgfqpoint{1.122062in}{3.847400in}}{\pgfqpoint{1.122062in}{3.841576in}}%
\pgfpathcurveto{\pgfqpoint{1.122062in}{3.835752in}}{\pgfqpoint{1.124376in}{3.830166in}}{\pgfqpoint{1.128494in}{3.826048in}}%
\pgfpathcurveto{\pgfqpoint{1.132613in}{3.821930in}}{\pgfqpoint{1.138199in}{3.819616in}}{\pgfqpoint{1.144023in}{3.819616in}}%
\pgfpathlineto{\pgfqpoint{1.144023in}{3.819616in}}%
\pgfpathclose%
\pgfusepath{stroke}%
\end{pgfscope}%
\begin{pgfscope}%
\pgfpathrectangle{\pgfqpoint{0.892500in}{0.533500in}}{\pgfqpoint{5.533500in}{3.734500in}}%
\pgfusepath{clip}%
\pgfsetrectcap%
\pgfsetroundjoin%
\pgfsetlinewidth{0.803000pt}%
\definecolor{currentstroke}{rgb}{0.690196,0.690196,0.690196}%
\pgfsetstrokecolor{currentstroke}%
\pgfsetdash{}{0pt}%
\pgfpathmoveto{\pgfqpoint{1.144023in}{0.533500in}}%
\pgfpathlineto{\pgfqpoint{1.144023in}{4.268000in}}%
\pgfusepath{stroke}%
\end{pgfscope}%
\begin{pgfscope}%
\pgfsetbuttcap%
\pgfsetroundjoin%
\definecolor{currentfill}{rgb}{0.000000,0.000000,0.000000}%
\pgfsetfillcolor{currentfill}%
\pgfsetlinewidth{0.803000pt}%
\definecolor{currentstroke}{rgb}{0.000000,0.000000,0.000000}%
\pgfsetstrokecolor{currentstroke}%
\pgfsetdash{}{0pt}%
\pgfsys@defobject{currentmarker}{\pgfqpoint{0.000000in}{-0.048611in}}{\pgfqpoint{0.000000in}{0.000000in}}{%
\pgfpathmoveto{\pgfqpoint{0.000000in}{0.000000in}}%
\pgfpathlineto{\pgfqpoint{0.000000in}{-0.048611in}}%
\pgfusepath{stroke,fill}%
}%
\begin{pgfscope}%
\pgfsys@transformshift{1.144023in}{0.533500in}%
\pgfsys@useobject{currentmarker}{}%
\end{pgfscope}%
\end{pgfscope}%
\begin{pgfscope}%
\definecolor{textcolor}{rgb}{0.000000,0.000000,0.000000}%
\pgfsetstrokecolor{textcolor}%
\pgfsetfillcolor{textcolor}%
\pgftext[x=1.144023in,y=0.436278in,,top]{\color{textcolor}{\rmfamily\fontsize{10.000000}{12.000000}\selectfont\catcode`\^=\active\def^{\ifmmode\sp\else\^{}\fi}\catcode`\%=\active\def%{\%}$\mathdefault{2}$}}%
\end{pgfscope}%
\begin{pgfscope}%
\pgfpathrectangle{\pgfqpoint{0.892500in}{0.533500in}}{\pgfqpoint{5.533500in}{3.734500in}}%
\pgfusepath{clip}%
\pgfsetrectcap%
\pgfsetroundjoin%
\pgfsetlinewidth{0.803000pt}%
\definecolor{currentstroke}{rgb}{0.690196,0.690196,0.690196}%
\pgfsetstrokecolor{currentstroke}%
\pgfsetdash{}{0pt}%
\pgfpathmoveto{\pgfqpoint{1.772830in}{0.533500in}}%
\pgfpathlineto{\pgfqpoint{1.772830in}{4.268000in}}%
\pgfusepath{stroke}%
\end{pgfscope}%
\begin{pgfscope}%
\pgfsetbuttcap%
\pgfsetroundjoin%
\definecolor{currentfill}{rgb}{0.000000,0.000000,0.000000}%
\pgfsetfillcolor{currentfill}%
\pgfsetlinewidth{0.803000pt}%
\definecolor{currentstroke}{rgb}{0.000000,0.000000,0.000000}%
\pgfsetstrokecolor{currentstroke}%
\pgfsetdash{}{0pt}%
\pgfsys@defobject{currentmarker}{\pgfqpoint{0.000000in}{-0.048611in}}{\pgfqpoint{0.000000in}{0.000000in}}{%
\pgfpathmoveto{\pgfqpoint{0.000000in}{0.000000in}}%
\pgfpathlineto{\pgfqpoint{0.000000in}{-0.048611in}}%
\pgfusepath{stroke,fill}%
}%
\begin{pgfscope}%
\pgfsys@transformshift{1.772830in}{0.533500in}%
\pgfsys@useobject{currentmarker}{}%
\end{pgfscope}%
\end{pgfscope}%
\begin{pgfscope}%
\definecolor{textcolor}{rgb}{0.000000,0.000000,0.000000}%
\pgfsetstrokecolor{textcolor}%
\pgfsetfillcolor{textcolor}%
\pgftext[x=1.772830in,y=0.436278in,,top]{\color{textcolor}{\rmfamily\fontsize{10.000000}{12.000000}\selectfont\catcode`\^=\active\def^{\ifmmode\sp\else\^{}\fi}\catcode`\%=\active\def%{\%}$\mathdefault{3}$}}%
\end{pgfscope}%
\begin{pgfscope}%
\pgfpathrectangle{\pgfqpoint{0.892500in}{0.533500in}}{\pgfqpoint{5.533500in}{3.734500in}}%
\pgfusepath{clip}%
\pgfsetrectcap%
\pgfsetroundjoin%
\pgfsetlinewidth{0.803000pt}%
\definecolor{currentstroke}{rgb}{0.690196,0.690196,0.690196}%
\pgfsetstrokecolor{currentstroke}%
\pgfsetdash{}{0pt}%
\pgfpathmoveto{\pgfqpoint{2.401636in}{0.533500in}}%
\pgfpathlineto{\pgfqpoint{2.401636in}{4.268000in}}%
\pgfusepath{stroke}%
\end{pgfscope}%
\begin{pgfscope}%
\pgfsetbuttcap%
\pgfsetroundjoin%
\definecolor{currentfill}{rgb}{0.000000,0.000000,0.000000}%
\pgfsetfillcolor{currentfill}%
\pgfsetlinewidth{0.803000pt}%
\definecolor{currentstroke}{rgb}{0.000000,0.000000,0.000000}%
\pgfsetstrokecolor{currentstroke}%
\pgfsetdash{}{0pt}%
\pgfsys@defobject{currentmarker}{\pgfqpoint{0.000000in}{-0.048611in}}{\pgfqpoint{0.000000in}{0.000000in}}{%
\pgfpathmoveto{\pgfqpoint{0.000000in}{0.000000in}}%
\pgfpathlineto{\pgfqpoint{0.000000in}{-0.048611in}}%
\pgfusepath{stroke,fill}%
}%
\begin{pgfscope}%
\pgfsys@transformshift{2.401636in}{0.533500in}%
\pgfsys@useobject{currentmarker}{}%
\end{pgfscope}%
\end{pgfscope}%
\begin{pgfscope}%
\definecolor{textcolor}{rgb}{0.000000,0.000000,0.000000}%
\pgfsetstrokecolor{textcolor}%
\pgfsetfillcolor{textcolor}%
\pgftext[x=2.401636in,y=0.436278in,,top]{\color{textcolor}{\rmfamily\fontsize{10.000000}{12.000000}\selectfont\catcode`\^=\active\def^{\ifmmode\sp\else\^{}\fi}\catcode`\%=\active\def%{\%}$\mathdefault{4}$}}%
\end{pgfscope}%
\begin{pgfscope}%
\pgfpathrectangle{\pgfqpoint{0.892500in}{0.533500in}}{\pgfqpoint{5.533500in}{3.734500in}}%
\pgfusepath{clip}%
\pgfsetrectcap%
\pgfsetroundjoin%
\pgfsetlinewidth{0.803000pt}%
\definecolor{currentstroke}{rgb}{0.690196,0.690196,0.690196}%
\pgfsetstrokecolor{currentstroke}%
\pgfsetdash{}{0pt}%
\pgfpathmoveto{\pgfqpoint{3.030443in}{0.533500in}}%
\pgfpathlineto{\pgfqpoint{3.030443in}{4.268000in}}%
\pgfusepath{stroke}%
\end{pgfscope}%
\begin{pgfscope}%
\pgfsetbuttcap%
\pgfsetroundjoin%
\definecolor{currentfill}{rgb}{0.000000,0.000000,0.000000}%
\pgfsetfillcolor{currentfill}%
\pgfsetlinewidth{0.803000pt}%
\definecolor{currentstroke}{rgb}{0.000000,0.000000,0.000000}%
\pgfsetstrokecolor{currentstroke}%
\pgfsetdash{}{0pt}%
\pgfsys@defobject{currentmarker}{\pgfqpoint{0.000000in}{-0.048611in}}{\pgfqpoint{0.000000in}{0.000000in}}{%
\pgfpathmoveto{\pgfqpoint{0.000000in}{0.000000in}}%
\pgfpathlineto{\pgfqpoint{0.000000in}{-0.048611in}}%
\pgfusepath{stroke,fill}%
}%
\begin{pgfscope}%
\pgfsys@transformshift{3.030443in}{0.533500in}%
\pgfsys@useobject{currentmarker}{}%
\end{pgfscope}%
\end{pgfscope}%
\begin{pgfscope}%
\definecolor{textcolor}{rgb}{0.000000,0.000000,0.000000}%
\pgfsetstrokecolor{textcolor}%
\pgfsetfillcolor{textcolor}%
\pgftext[x=3.030443in,y=0.436278in,,top]{\color{textcolor}{\rmfamily\fontsize{10.000000}{12.000000}\selectfont\catcode`\^=\active\def^{\ifmmode\sp\else\^{}\fi}\catcode`\%=\active\def%{\%}$\mathdefault{5}$}}%
\end{pgfscope}%
\begin{pgfscope}%
\pgfpathrectangle{\pgfqpoint{0.892500in}{0.533500in}}{\pgfqpoint{5.533500in}{3.734500in}}%
\pgfusepath{clip}%
\pgfsetrectcap%
\pgfsetroundjoin%
\pgfsetlinewidth{0.803000pt}%
\definecolor{currentstroke}{rgb}{0.690196,0.690196,0.690196}%
\pgfsetstrokecolor{currentstroke}%
\pgfsetdash{}{0pt}%
\pgfpathmoveto{\pgfqpoint{3.659250in}{0.533500in}}%
\pgfpathlineto{\pgfqpoint{3.659250in}{4.268000in}}%
\pgfusepath{stroke}%
\end{pgfscope}%
\begin{pgfscope}%
\pgfsetbuttcap%
\pgfsetroundjoin%
\definecolor{currentfill}{rgb}{0.000000,0.000000,0.000000}%
\pgfsetfillcolor{currentfill}%
\pgfsetlinewidth{0.803000pt}%
\definecolor{currentstroke}{rgb}{0.000000,0.000000,0.000000}%
\pgfsetstrokecolor{currentstroke}%
\pgfsetdash{}{0pt}%
\pgfsys@defobject{currentmarker}{\pgfqpoint{0.000000in}{-0.048611in}}{\pgfqpoint{0.000000in}{0.000000in}}{%
\pgfpathmoveto{\pgfqpoint{0.000000in}{0.000000in}}%
\pgfpathlineto{\pgfqpoint{0.000000in}{-0.048611in}}%
\pgfusepath{stroke,fill}%
}%
\begin{pgfscope}%
\pgfsys@transformshift{3.659250in}{0.533500in}%
\pgfsys@useobject{currentmarker}{}%
\end{pgfscope}%
\end{pgfscope}%
\begin{pgfscope}%
\definecolor{textcolor}{rgb}{0.000000,0.000000,0.000000}%
\pgfsetstrokecolor{textcolor}%
\pgfsetfillcolor{textcolor}%
\pgftext[x=3.659250in,y=0.436278in,,top]{\color{textcolor}{\rmfamily\fontsize{10.000000}{12.000000}\selectfont\catcode`\^=\active\def^{\ifmmode\sp\else\^{}\fi}\catcode`\%=\active\def%{\%}$\mathdefault{6}$}}%
\end{pgfscope}%
\begin{pgfscope}%
\pgfpathrectangle{\pgfqpoint{0.892500in}{0.533500in}}{\pgfqpoint{5.533500in}{3.734500in}}%
\pgfusepath{clip}%
\pgfsetrectcap%
\pgfsetroundjoin%
\pgfsetlinewidth{0.803000pt}%
\definecolor{currentstroke}{rgb}{0.690196,0.690196,0.690196}%
\pgfsetstrokecolor{currentstroke}%
\pgfsetdash{}{0pt}%
\pgfpathmoveto{\pgfqpoint{4.288057in}{0.533500in}}%
\pgfpathlineto{\pgfqpoint{4.288057in}{4.268000in}}%
\pgfusepath{stroke}%
\end{pgfscope}%
\begin{pgfscope}%
\pgfsetbuttcap%
\pgfsetroundjoin%
\definecolor{currentfill}{rgb}{0.000000,0.000000,0.000000}%
\pgfsetfillcolor{currentfill}%
\pgfsetlinewidth{0.803000pt}%
\definecolor{currentstroke}{rgb}{0.000000,0.000000,0.000000}%
\pgfsetstrokecolor{currentstroke}%
\pgfsetdash{}{0pt}%
\pgfsys@defobject{currentmarker}{\pgfqpoint{0.000000in}{-0.048611in}}{\pgfqpoint{0.000000in}{0.000000in}}{%
\pgfpathmoveto{\pgfqpoint{0.000000in}{0.000000in}}%
\pgfpathlineto{\pgfqpoint{0.000000in}{-0.048611in}}%
\pgfusepath{stroke,fill}%
}%
\begin{pgfscope}%
\pgfsys@transformshift{4.288057in}{0.533500in}%
\pgfsys@useobject{currentmarker}{}%
\end{pgfscope}%
\end{pgfscope}%
\begin{pgfscope}%
\definecolor{textcolor}{rgb}{0.000000,0.000000,0.000000}%
\pgfsetstrokecolor{textcolor}%
\pgfsetfillcolor{textcolor}%
\pgftext[x=4.288057in,y=0.436278in,,top]{\color{textcolor}{\rmfamily\fontsize{10.000000}{12.000000}\selectfont\catcode`\^=\active\def^{\ifmmode\sp\else\^{}\fi}\catcode`\%=\active\def%{\%}$\mathdefault{7}$}}%
\end{pgfscope}%
\begin{pgfscope}%
\pgfpathrectangle{\pgfqpoint{0.892500in}{0.533500in}}{\pgfqpoint{5.533500in}{3.734500in}}%
\pgfusepath{clip}%
\pgfsetrectcap%
\pgfsetroundjoin%
\pgfsetlinewidth{0.803000pt}%
\definecolor{currentstroke}{rgb}{0.690196,0.690196,0.690196}%
\pgfsetstrokecolor{currentstroke}%
\pgfsetdash{}{0pt}%
\pgfpathmoveto{\pgfqpoint{4.916864in}{0.533500in}}%
\pgfpathlineto{\pgfqpoint{4.916864in}{4.268000in}}%
\pgfusepath{stroke}%
\end{pgfscope}%
\begin{pgfscope}%
\pgfsetbuttcap%
\pgfsetroundjoin%
\definecolor{currentfill}{rgb}{0.000000,0.000000,0.000000}%
\pgfsetfillcolor{currentfill}%
\pgfsetlinewidth{0.803000pt}%
\definecolor{currentstroke}{rgb}{0.000000,0.000000,0.000000}%
\pgfsetstrokecolor{currentstroke}%
\pgfsetdash{}{0pt}%
\pgfsys@defobject{currentmarker}{\pgfqpoint{0.000000in}{-0.048611in}}{\pgfqpoint{0.000000in}{0.000000in}}{%
\pgfpathmoveto{\pgfqpoint{0.000000in}{0.000000in}}%
\pgfpathlineto{\pgfqpoint{0.000000in}{-0.048611in}}%
\pgfusepath{stroke,fill}%
}%
\begin{pgfscope}%
\pgfsys@transformshift{4.916864in}{0.533500in}%
\pgfsys@useobject{currentmarker}{}%
\end{pgfscope}%
\end{pgfscope}%
\begin{pgfscope}%
\definecolor{textcolor}{rgb}{0.000000,0.000000,0.000000}%
\pgfsetstrokecolor{textcolor}%
\pgfsetfillcolor{textcolor}%
\pgftext[x=4.916864in,y=0.436278in,,top]{\color{textcolor}{\rmfamily\fontsize{10.000000}{12.000000}\selectfont\catcode`\^=\active\def^{\ifmmode\sp\else\^{}\fi}\catcode`\%=\active\def%{\%}$\mathdefault{8}$}}%
\end{pgfscope}%
\begin{pgfscope}%
\pgfpathrectangle{\pgfqpoint{0.892500in}{0.533500in}}{\pgfqpoint{5.533500in}{3.734500in}}%
\pgfusepath{clip}%
\pgfsetrectcap%
\pgfsetroundjoin%
\pgfsetlinewidth{0.803000pt}%
\definecolor{currentstroke}{rgb}{0.690196,0.690196,0.690196}%
\pgfsetstrokecolor{currentstroke}%
\pgfsetdash{}{0pt}%
\pgfpathmoveto{\pgfqpoint{5.545670in}{0.533500in}}%
\pgfpathlineto{\pgfqpoint{5.545670in}{4.268000in}}%
\pgfusepath{stroke}%
\end{pgfscope}%
\begin{pgfscope}%
\pgfsetbuttcap%
\pgfsetroundjoin%
\definecolor{currentfill}{rgb}{0.000000,0.000000,0.000000}%
\pgfsetfillcolor{currentfill}%
\pgfsetlinewidth{0.803000pt}%
\definecolor{currentstroke}{rgb}{0.000000,0.000000,0.000000}%
\pgfsetstrokecolor{currentstroke}%
\pgfsetdash{}{0pt}%
\pgfsys@defobject{currentmarker}{\pgfqpoint{0.000000in}{-0.048611in}}{\pgfqpoint{0.000000in}{0.000000in}}{%
\pgfpathmoveto{\pgfqpoint{0.000000in}{0.000000in}}%
\pgfpathlineto{\pgfqpoint{0.000000in}{-0.048611in}}%
\pgfusepath{stroke,fill}%
}%
\begin{pgfscope}%
\pgfsys@transformshift{5.545670in}{0.533500in}%
\pgfsys@useobject{currentmarker}{}%
\end{pgfscope}%
\end{pgfscope}%
\begin{pgfscope}%
\definecolor{textcolor}{rgb}{0.000000,0.000000,0.000000}%
\pgfsetstrokecolor{textcolor}%
\pgfsetfillcolor{textcolor}%
\pgftext[x=5.545670in,y=0.436278in,,top]{\color{textcolor}{\rmfamily\fontsize{10.000000}{12.000000}\selectfont\catcode`\^=\active\def^{\ifmmode\sp\else\^{}\fi}\catcode`\%=\active\def%{\%}$\mathdefault{9}$}}%
\end{pgfscope}%
\begin{pgfscope}%
\pgfpathrectangle{\pgfqpoint{0.892500in}{0.533500in}}{\pgfqpoint{5.533500in}{3.734500in}}%
\pgfusepath{clip}%
\pgfsetrectcap%
\pgfsetroundjoin%
\pgfsetlinewidth{0.803000pt}%
\definecolor{currentstroke}{rgb}{0.690196,0.690196,0.690196}%
\pgfsetstrokecolor{currentstroke}%
\pgfsetdash{}{0pt}%
\pgfpathmoveto{\pgfqpoint{6.174477in}{0.533500in}}%
\pgfpathlineto{\pgfqpoint{6.174477in}{4.268000in}}%
\pgfusepath{stroke}%
\end{pgfscope}%
\begin{pgfscope}%
\pgfsetbuttcap%
\pgfsetroundjoin%
\definecolor{currentfill}{rgb}{0.000000,0.000000,0.000000}%
\pgfsetfillcolor{currentfill}%
\pgfsetlinewidth{0.803000pt}%
\definecolor{currentstroke}{rgb}{0.000000,0.000000,0.000000}%
\pgfsetstrokecolor{currentstroke}%
\pgfsetdash{}{0pt}%
\pgfsys@defobject{currentmarker}{\pgfqpoint{0.000000in}{-0.048611in}}{\pgfqpoint{0.000000in}{0.000000in}}{%
\pgfpathmoveto{\pgfqpoint{0.000000in}{0.000000in}}%
\pgfpathlineto{\pgfqpoint{0.000000in}{-0.048611in}}%
\pgfusepath{stroke,fill}%
}%
\begin{pgfscope}%
\pgfsys@transformshift{6.174477in}{0.533500in}%
\pgfsys@useobject{currentmarker}{}%
\end{pgfscope}%
\end{pgfscope}%
\begin{pgfscope}%
\definecolor{textcolor}{rgb}{0.000000,0.000000,0.000000}%
\pgfsetstrokecolor{textcolor}%
\pgfsetfillcolor{textcolor}%
\pgftext[x=6.174477in,y=0.436278in,,top]{\color{textcolor}{\rmfamily\fontsize{10.000000}{12.000000}\selectfont\catcode`\^=\active\def^{\ifmmode\sp\else\^{}\fi}\catcode`\%=\active\def%{\%}$\mathdefault{10}$}}%
\end{pgfscope}%
\begin{pgfscope}%
\definecolor{textcolor}{rgb}{0.000000,0.000000,0.000000}%
\pgfsetstrokecolor{textcolor}%
\pgfsetfillcolor{textcolor}%
\pgftext[x=3.659250in,y=0.246309in,,top]{\color{textcolor}{\rmfamily\fontsize{14.000000}{16.800000}\selectfont\catcode`\^=\active\def^{\ifmmode\sp\else\^{}\fi}\catcode`\%=\active\def%{\%}Pressure/$\mathrm{bar}$}}%
\end{pgfscope}%
\begin{pgfscope}%
\pgfpathrectangle{\pgfqpoint{0.892500in}{0.533500in}}{\pgfqpoint{5.533500in}{3.734500in}}%
\pgfusepath{clip}%
\pgfsetrectcap%
\pgfsetroundjoin%
\pgfsetlinewidth{0.803000pt}%
\definecolor{currentstroke}{rgb}{0.690196,0.690196,0.690196}%
\pgfsetstrokecolor{currentstroke}%
\pgfsetdash{}{0pt}%
\pgfpathmoveto{\pgfqpoint{0.892500in}{0.998213in}}%
\pgfpathlineto{\pgfqpoint{6.426000in}{0.998213in}}%
\pgfusepath{stroke}%
\end{pgfscope}%
\begin{pgfscope}%
\pgfsetbuttcap%
\pgfsetroundjoin%
\definecolor{currentfill}{rgb}{0.000000,0.000000,0.000000}%
\pgfsetfillcolor{currentfill}%
\pgfsetlinewidth{0.803000pt}%
\definecolor{currentstroke}{rgb}{0.000000,0.000000,0.000000}%
\pgfsetstrokecolor{currentstroke}%
\pgfsetdash{}{0pt}%
\pgfsys@defobject{currentmarker}{\pgfqpoint{-0.048611in}{0.000000in}}{\pgfqpoint{-0.000000in}{0.000000in}}{%
\pgfpathmoveto{\pgfqpoint{-0.000000in}{0.000000in}}%
\pgfpathlineto{\pgfqpoint{-0.048611in}{0.000000in}}%
\pgfusepath{stroke,fill}%
}%
\begin{pgfscope}%
\pgfsys@transformshift{0.892500in}{0.998213in}%
\pgfsys@useobject{currentmarker}{}%
\end{pgfscope}%
\end{pgfscope}%
\begin{pgfscope}%
\definecolor{textcolor}{rgb}{0.000000,0.000000,0.000000}%
\pgfsetstrokecolor{textcolor}%
\pgfsetfillcolor{textcolor}%
\pgftext[x=0.478919in, y=0.945451in, left, base]{\color{textcolor}{\rmfamily\fontsize{10.000000}{12.000000}\selectfont\catcode`\^=\active\def^{\ifmmode\sp\else\^{}\fi}\catcode`\%=\active\def%{\%}$\mathdefault{0.006}$}}%
\end{pgfscope}%
\begin{pgfscope}%
\pgfpathrectangle{\pgfqpoint{0.892500in}{0.533500in}}{\pgfqpoint{5.533500in}{3.734500in}}%
\pgfusepath{clip}%
\pgfsetrectcap%
\pgfsetroundjoin%
\pgfsetlinewidth{0.803000pt}%
\definecolor{currentstroke}{rgb}{0.690196,0.690196,0.690196}%
\pgfsetstrokecolor{currentstroke}%
\pgfsetdash{}{0pt}%
\pgfpathmoveto{\pgfqpoint{0.892500in}{1.581534in}}%
\pgfpathlineto{\pgfqpoint{6.426000in}{1.581534in}}%
\pgfusepath{stroke}%
\end{pgfscope}%
\begin{pgfscope}%
\pgfsetbuttcap%
\pgfsetroundjoin%
\definecolor{currentfill}{rgb}{0.000000,0.000000,0.000000}%
\pgfsetfillcolor{currentfill}%
\pgfsetlinewidth{0.803000pt}%
\definecolor{currentstroke}{rgb}{0.000000,0.000000,0.000000}%
\pgfsetstrokecolor{currentstroke}%
\pgfsetdash{}{0pt}%
\pgfsys@defobject{currentmarker}{\pgfqpoint{-0.048611in}{0.000000in}}{\pgfqpoint{-0.000000in}{0.000000in}}{%
\pgfpathmoveto{\pgfqpoint{-0.000000in}{0.000000in}}%
\pgfpathlineto{\pgfqpoint{-0.048611in}{0.000000in}}%
\pgfusepath{stroke,fill}%
}%
\begin{pgfscope}%
\pgfsys@transformshift{0.892500in}{1.581534in}%
\pgfsys@useobject{currentmarker}{}%
\end{pgfscope}%
\end{pgfscope}%
\begin{pgfscope}%
\definecolor{textcolor}{rgb}{0.000000,0.000000,0.000000}%
\pgfsetstrokecolor{textcolor}%
\pgfsetfillcolor{textcolor}%
\pgftext[x=0.478919in, y=1.528772in, left, base]{\color{textcolor}{\rmfamily\fontsize{10.000000}{12.000000}\selectfont\catcode`\^=\active\def^{\ifmmode\sp\else\^{}\fi}\catcode`\%=\active\def%{\%}$\mathdefault{0.008}$}}%
\end{pgfscope}%
\begin{pgfscope}%
\pgfpathrectangle{\pgfqpoint{0.892500in}{0.533500in}}{\pgfqpoint{5.533500in}{3.734500in}}%
\pgfusepath{clip}%
\pgfsetrectcap%
\pgfsetroundjoin%
\pgfsetlinewidth{0.803000pt}%
\definecolor{currentstroke}{rgb}{0.690196,0.690196,0.690196}%
\pgfsetstrokecolor{currentstroke}%
\pgfsetdash{}{0pt}%
\pgfpathmoveto{\pgfqpoint{0.892500in}{2.164855in}}%
\pgfpathlineto{\pgfqpoint{6.426000in}{2.164855in}}%
\pgfusepath{stroke}%
\end{pgfscope}%
\begin{pgfscope}%
\pgfsetbuttcap%
\pgfsetroundjoin%
\definecolor{currentfill}{rgb}{0.000000,0.000000,0.000000}%
\pgfsetfillcolor{currentfill}%
\pgfsetlinewidth{0.803000pt}%
\definecolor{currentstroke}{rgb}{0.000000,0.000000,0.000000}%
\pgfsetstrokecolor{currentstroke}%
\pgfsetdash{}{0pt}%
\pgfsys@defobject{currentmarker}{\pgfqpoint{-0.048611in}{0.000000in}}{\pgfqpoint{-0.000000in}{0.000000in}}{%
\pgfpathmoveto{\pgfqpoint{-0.000000in}{0.000000in}}%
\pgfpathlineto{\pgfqpoint{-0.048611in}{0.000000in}}%
\pgfusepath{stroke,fill}%
}%
\begin{pgfscope}%
\pgfsys@transformshift{0.892500in}{2.164855in}%
\pgfsys@useobject{currentmarker}{}%
\end{pgfscope}%
\end{pgfscope}%
\begin{pgfscope}%
\definecolor{textcolor}{rgb}{0.000000,0.000000,0.000000}%
\pgfsetstrokecolor{textcolor}%
\pgfsetfillcolor{textcolor}%
\pgftext[x=0.478919in, y=2.112093in, left, base]{\color{textcolor}{\rmfamily\fontsize{10.000000}{12.000000}\selectfont\catcode`\^=\active\def^{\ifmmode\sp\else\^{}\fi}\catcode`\%=\active\def%{\%}$\mathdefault{0.010}$}}%
\end{pgfscope}%
\begin{pgfscope}%
\pgfpathrectangle{\pgfqpoint{0.892500in}{0.533500in}}{\pgfqpoint{5.533500in}{3.734500in}}%
\pgfusepath{clip}%
\pgfsetrectcap%
\pgfsetroundjoin%
\pgfsetlinewidth{0.803000pt}%
\definecolor{currentstroke}{rgb}{0.690196,0.690196,0.690196}%
\pgfsetstrokecolor{currentstroke}%
\pgfsetdash{}{0pt}%
\pgfpathmoveto{\pgfqpoint{0.892500in}{2.748176in}}%
\pgfpathlineto{\pgfqpoint{6.426000in}{2.748176in}}%
\pgfusepath{stroke}%
\end{pgfscope}%
\begin{pgfscope}%
\pgfsetbuttcap%
\pgfsetroundjoin%
\definecolor{currentfill}{rgb}{0.000000,0.000000,0.000000}%
\pgfsetfillcolor{currentfill}%
\pgfsetlinewidth{0.803000pt}%
\definecolor{currentstroke}{rgb}{0.000000,0.000000,0.000000}%
\pgfsetstrokecolor{currentstroke}%
\pgfsetdash{}{0pt}%
\pgfsys@defobject{currentmarker}{\pgfqpoint{-0.048611in}{0.000000in}}{\pgfqpoint{-0.000000in}{0.000000in}}{%
\pgfpathmoveto{\pgfqpoint{-0.000000in}{0.000000in}}%
\pgfpathlineto{\pgfqpoint{-0.048611in}{0.000000in}}%
\pgfusepath{stroke,fill}%
}%
\begin{pgfscope}%
\pgfsys@transformshift{0.892500in}{2.748176in}%
\pgfsys@useobject{currentmarker}{}%
\end{pgfscope}%
\end{pgfscope}%
\begin{pgfscope}%
\definecolor{textcolor}{rgb}{0.000000,0.000000,0.000000}%
\pgfsetstrokecolor{textcolor}%
\pgfsetfillcolor{textcolor}%
\pgftext[x=0.478919in, y=2.695414in, left, base]{\color{textcolor}{\rmfamily\fontsize{10.000000}{12.000000}\selectfont\catcode`\^=\active\def^{\ifmmode\sp\else\^{}\fi}\catcode`\%=\active\def%{\%}$\mathdefault{0.012}$}}%
\end{pgfscope}%
\begin{pgfscope}%
\pgfpathrectangle{\pgfqpoint{0.892500in}{0.533500in}}{\pgfqpoint{5.533500in}{3.734500in}}%
\pgfusepath{clip}%
\pgfsetrectcap%
\pgfsetroundjoin%
\pgfsetlinewidth{0.803000pt}%
\definecolor{currentstroke}{rgb}{0.690196,0.690196,0.690196}%
\pgfsetstrokecolor{currentstroke}%
\pgfsetdash{}{0pt}%
\pgfpathmoveto{\pgfqpoint{0.892500in}{3.331496in}}%
\pgfpathlineto{\pgfqpoint{6.426000in}{3.331496in}}%
\pgfusepath{stroke}%
\end{pgfscope}%
\begin{pgfscope}%
\pgfsetbuttcap%
\pgfsetroundjoin%
\definecolor{currentfill}{rgb}{0.000000,0.000000,0.000000}%
\pgfsetfillcolor{currentfill}%
\pgfsetlinewidth{0.803000pt}%
\definecolor{currentstroke}{rgb}{0.000000,0.000000,0.000000}%
\pgfsetstrokecolor{currentstroke}%
\pgfsetdash{}{0pt}%
\pgfsys@defobject{currentmarker}{\pgfqpoint{-0.048611in}{0.000000in}}{\pgfqpoint{-0.000000in}{0.000000in}}{%
\pgfpathmoveto{\pgfqpoint{-0.000000in}{0.000000in}}%
\pgfpathlineto{\pgfqpoint{-0.048611in}{0.000000in}}%
\pgfusepath{stroke,fill}%
}%
\begin{pgfscope}%
\pgfsys@transformshift{0.892500in}{3.331496in}%
\pgfsys@useobject{currentmarker}{}%
\end{pgfscope}%
\end{pgfscope}%
\begin{pgfscope}%
\definecolor{textcolor}{rgb}{0.000000,0.000000,0.000000}%
\pgfsetstrokecolor{textcolor}%
\pgfsetfillcolor{textcolor}%
\pgftext[x=0.478919in, y=3.278735in, left, base]{\color{textcolor}{\rmfamily\fontsize{10.000000}{12.000000}\selectfont\catcode`\^=\active\def^{\ifmmode\sp\else\^{}\fi}\catcode`\%=\active\def%{\%}$\mathdefault{0.014}$}}%
\end{pgfscope}%
\begin{pgfscope}%
\pgfpathrectangle{\pgfqpoint{0.892500in}{0.533500in}}{\pgfqpoint{5.533500in}{3.734500in}}%
\pgfusepath{clip}%
\pgfsetrectcap%
\pgfsetroundjoin%
\pgfsetlinewidth{0.803000pt}%
\definecolor{currentstroke}{rgb}{0.690196,0.690196,0.690196}%
\pgfsetstrokecolor{currentstroke}%
\pgfsetdash{}{0pt}%
\pgfpathmoveto{\pgfqpoint{0.892500in}{3.914817in}}%
\pgfpathlineto{\pgfqpoint{6.426000in}{3.914817in}}%
\pgfusepath{stroke}%
\end{pgfscope}%
\begin{pgfscope}%
\pgfsetbuttcap%
\pgfsetroundjoin%
\definecolor{currentfill}{rgb}{0.000000,0.000000,0.000000}%
\pgfsetfillcolor{currentfill}%
\pgfsetlinewidth{0.803000pt}%
\definecolor{currentstroke}{rgb}{0.000000,0.000000,0.000000}%
\pgfsetstrokecolor{currentstroke}%
\pgfsetdash{}{0pt}%
\pgfsys@defobject{currentmarker}{\pgfqpoint{-0.048611in}{0.000000in}}{\pgfqpoint{-0.000000in}{0.000000in}}{%
\pgfpathmoveto{\pgfqpoint{-0.000000in}{0.000000in}}%
\pgfpathlineto{\pgfqpoint{-0.048611in}{0.000000in}}%
\pgfusepath{stroke,fill}%
}%
\begin{pgfscope}%
\pgfsys@transformshift{0.892500in}{3.914817in}%
\pgfsys@useobject{currentmarker}{}%
\end{pgfscope}%
\end{pgfscope}%
\begin{pgfscope}%
\definecolor{textcolor}{rgb}{0.000000,0.000000,0.000000}%
\pgfsetstrokecolor{textcolor}%
\pgfsetfillcolor{textcolor}%
\pgftext[x=0.478919in, y=3.862056in, left, base]{\color{textcolor}{\rmfamily\fontsize{10.000000}{12.000000}\selectfont\catcode`\^=\active\def^{\ifmmode\sp\else\^{}\fi}\catcode`\%=\active\def%{\%}$\mathdefault{0.016}$}}%
\end{pgfscope}%
\begin{pgfscope}%
\definecolor{textcolor}{rgb}{0.000000,0.000000,0.000000}%
\pgfsetstrokecolor{textcolor}%
\pgfsetfillcolor{textcolor}%
\pgftext[x=0.423363in,y=2.400750in,,bottom,rotate=90.000000]{\color{textcolor}{\rmfamily\fontsize{14.000000}{16.800000}\selectfont\catcode`\^=\active\def^{\ifmmode\sp\else\^{}\fi}\catcode`\%=\active\def%{\%}Volume/$\mathrm{m}^3\mathrm{mol}^{-1}$}}%
\end{pgfscope}%
\begin{pgfscope}%
\pgfpathrectangle{\pgfqpoint{0.892500in}{0.533500in}}{\pgfqpoint{5.533500in}{3.734500in}}%
\pgfusepath{clip}%
\pgfsetrectcap%
\pgfsetroundjoin%
\pgfsetlinewidth{1.505625pt}%
\definecolor{currentstroke}{rgb}{0.501961,0.000000,0.501961}%
\pgfsetstrokecolor{currentstroke}%
\pgfsetdash{}{0pt}%
\pgfpathmoveto{\pgfqpoint{6.174477in}{0.703250in}}%
\pgfpathlineto{\pgfqpoint{6.130016in}{0.713611in}}%
\pgfpathlineto{\pgfqpoint{6.085555in}{0.724121in}}%
\pgfpathlineto{\pgfqpoint{6.041094in}{0.734783in}}%
\pgfpathlineto{\pgfqpoint{5.996633in}{0.745599in}}%
\pgfpathlineto{\pgfqpoint{5.952172in}{0.756575in}}%
\pgfpathlineto{\pgfqpoint{5.907711in}{0.767712in}}%
\pgfpathlineto{\pgfqpoint{5.863250in}{0.779015in}}%
\pgfpathlineto{\pgfqpoint{5.818789in}{0.790488in}}%
\pgfpathlineto{\pgfqpoint{5.774327in}{0.802133in}}%
\pgfpathlineto{\pgfqpoint{5.729866in}{0.813957in}}%
\pgfpathlineto{\pgfqpoint{5.685405in}{0.825961in}}%
\pgfpathlineto{\pgfqpoint{5.640944in}{0.838151in}}%
\pgfpathlineto{\pgfqpoint{5.596483in}{0.850530in}}%
\pgfpathlineto{\pgfqpoint{5.552022in}{0.863104in}}%
\pgfpathlineto{\pgfqpoint{5.507561in}{0.875877in}}%
\pgfpathlineto{\pgfqpoint{5.463100in}{0.888854in}}%
\pgfpathlineto{\pgfqpoint{5.418639in}{0.902039in}}%
\pgfpathlineto{\pgfqpoint{5.374178in}{0.915438in}}%
\pgfpathlineto{\pgfqpoint{5.329717in}{0.929055in}}%
\pgfpathlineto{\pgfqpoint{5.285256in}{0.942897in}}%
\pgfpathlineto{\pgfqpoint{5.240794in}{0.956969in}}%
\pgfpathlineto{\pgfqpoint{5.196333in}{0.971276in}}%
\pgfpathlineto{\pgfqpoint{5.151872in}{0.985825in}}%
\pgfpathlineto{\pgfqpoint{5.107411in}{1.000622in}}%
\pgfpathlineto{\pgfqpoint{5.062950in}{1.015673in}}%
\pgfpathlineto{\pgfqpoint{5.018489in}{1.030985in}}%
\pgfpathlineto{\pgfqpoint{4.974028in}{1.046565in}}%
\pgfpathlineto{\pgfqpoint{4.929567in}{1.062419in}}%
\pgfpathlineto{\pgfqpoint{4.885106in}{1.078555in}}%
\pgfpathlineto{\pgfqpoint{4.840645in}{1.094981in}}%
\pgfpathlineto{\pgfqpoint{4.796184in}{1.111704in}}%
\pgfpathlineto{\pgfqpoint{4.751722in}{1.128733in}}%
\pgfpathlineto{\pgfqpoint{4.707261in}{1.146076in}}%
\pgfpathlineto{\pgfqpoint{4.662800in}{1.163742in}}%
\pgfpathlineto{\pgfqpoint{4.618339in}{1.181740in}}%
\pgfpathlineto{\pgfqpoint{4.573878in}{1.200079in}}%
\pgfpathlineto{\pgfqpoint{4.529417in}{1.218770in}}%
\pgfpathlineto{\pgfqpoint{4.484956in}{1.237822in}}%
\pgfpathlineto{\pgfqpoint{4.440495in}{1.257246in}}%
\pgfpathlineto{\pgfqpoint{4.396034in}{1.277053in}}%
\pgfpathlineto{\pgfqpoint{4.351573in}{1.297254in}}%
\pgfpathlineto{\pgfqpoint{4.307112in}{1.317862in}}%
\pgfpathlineto{\pgfqpoint{4.262650in}{1.338889in}}%
\pgfpathlineto{\pgfqpoint{4.218189in}{1.360347in}}%
\pgfpathlineto{\pgfqpoint{4.173728in}{1.382250in}}%
\pgfpathlineto{\pgfqpoint{4.129267in}{1.404612in}}%
\pgfpathlineto{\pgfqpoint{4.084806in}{1.427448in}}%
\pgfpathlineto{\pgfqpoint{4.040345in}{1.450773in}}%
\pgfpathlineto{\pgfqpoint{3.995884in}{1.474602in}}%
\pgfpathlineto{\pgfqpoint{3.951423in}{1.498953in}}%
\pgfpathlineto{\pgfqpoint{3.906962in}{1.523842in}}%
\pgfpathlineto{\pgfqpoint{3.862501in}{1.549288in}}%
\pgfpathlineto{\pgfqpoint{3.818040in}{1.575310in}}%
\pgfpathlineto{\pgfqpoint{3.773579in}{1.601926in}}%
\pgfpathlineto{\pgfqpoint{3.729117in}{1.629159in}}%
\pgfpathlineto{\pgfqpoint{3.684656in}{1.657029in}}%
\pgfpathlineto{\pgfqpoint{3.640195in}{1.685560in}}%
\pgfpathlineto{\pgfqpoint{3.595734in}{1.714774in}}%
\pgfpathlineto{\pgfqpoint{3.551273in}{1.744697in}}%
\pgfpathlineto{\pgfqpoint{3.506812in}{1.775355in}}%
\pgfpathlineto{\pgfqpoint{3.462351in}{1.806776in}}%
\pgfpathlineto{\pgfqpoint{3.417890in}{1.838987in}}%
\pgfpathlineto{\pgfqpoint{3.373429in}{1.872020in}}%
\pgfpathlineto{\pgfqpoint{3.328968in}{1.905907in}}%
\pgfpathlineto{\pgfqpoint{3.284507in}{1.940680in}}%
\pgfpathlineto{\pgfqpoint{3.240045in}{1.976375in}}%
\pgfpathlineto{\pgfqpoint{3.195584in}{2.013029in}}%
\pgfpathlineto{\pgfqpoint{3.151123in}{2.050682in}}%
\pgfpathlineto{\pgfqpoint{3.106662in}{2.089374in}}%
\pgfpathlineto{\pgfqpoint{3.062201in}{2.129150in}}%
\pgfpathlineto{\pgfqpoint{3.017740in}{2.170055in}}%
\pgfpathlineto{\pgfqpoint{2.973279in}{2.212139in}}%
\pgfpathlineto{\pgfqpoint{2.928818in}{2.255453in}}%
\pgfpathlineto{\pgfqpoint{2.884357in}{2.300051in}}%
\pgfpathlineto{\pgfqpoint{2.839896in}{2.345992in}}%
\pgfpathlineto{\pgfqpoint{2.795435in}{2.393337in}}%
\pgfpathlineto{\pgfqpoint{2.750973in}{2.442152in}}%
\pgfpathlineto{\pgfqpoint{2.706512in}{2.492507in}}%
\pgfpathlineto{\pgfqpoint{2.662051in}{2.544474in}}%
\pgfpathlineto{\pgfqpoint{2.617590in}{2.598134in}}%
\pgfpathlineto{\pgfqpoint{2.573129in}{2.653569in}}%
\pgfpathlineto{\pgfqpoint{2.528668in}{2.710870in}}%
\pgfpathlineto{\pgfqpoint{2.484207in}{2.770133in}}%
\pgfpathlineto{\pgfqpoint{2.439746in}{2.831459in}}%
\pgfpathlineto{\pgfqpoint{2.395285in}{2.894959in}}%
\pgfpathlineto{\pgfqpoint{2.350824in}{2.960750in}}%
\pgfpathlineto{\pgfqpoint{2.306363in}{3.028959in}}%
\pgfpathlineto{\pgfqpoint{2.261902in}{3.099721in}}%
\pgfpathlineto{\pgfqpoint{2.217440in}{3.173182in}}%
\pgfpathlineto{\pgfqpoint{2.172979in}{3.249500in}}%
\pgfpathlineto{\pgfqpoint{2.128518in}{3.328845in}}%
\pgfpathlineto{\pgfqpoint{2.084057in}{3.411400in}}%
\pgfpathlineto{\pgfqpoint{2.039596in}{3.497365in}}%
\pgfpathlineto{\pgfqpoint{1.995135in}{3.586955in}}%
\pgfpathlineto{\pgfqpoint{1.950674in}{3.680404in}}%
\pgfpathlineto{\pgfqpoint{1.906213in}{3.777967in}}%
\pgfpathlineto{\pgfqpoint{1.861752in}{3.879922in}}%
\pgfpathlineto{\pgfqpoint{1.817291in}{3.986572in}}%
\pgfpathlineto{\pgfqpoint{1.772830in}{4.098250in}}%
\pgfusepath{stroke}%
\end{pgfscope}%
\begin{pgfscope}%
\pgfpathrectangle{\pgfqpoint{0.892500in}{0.533500in}}{\pgfqpoint{5.533500in}{3.734500in}}%
\pgfusepath{clip}%
\pgfsetrectcap%
\pgfsetroundjoin%
\pgfsetlinewidth{1.505625pt}%
\definecolor{currentstroke}{rgb}{0.000000,0.501961,0.000000}%
\pgfsetstrokecolor{currentstroke}%
\pgfsetdash{}{0pt}%
\pgfpathmoveto{\pgfqpoint{1.772830in}{4.098250in}}%
\pgfpathlineto{\pgfqpoint{1.144023in}{4.098250in}}%
\pgfusepath{stroke}%
\end{pgfscope}%
\begin{pgfscope}%
\pgfpathrectangle{\pgfqpoint{0.892500in}{0.533500in}}{\pgfqpoint{5.533500in}{3.734500in}}%
\pgfusepath{clip}%
\pgfsetrectcap%
\pgfsetroundjoin%
\pgfsetlinewidth{1.505625pt}%
\definecolor{currentstroke}{rgb}{0.000000,0.000000,1.000000}%
\pgfsetstrokecolor{currentstroke}%
\pgfsetdash{}{0pt}%
\pgfpathmoveto{\pgfqpoint{1.144023in}{4.098250in}}%
\pgfpathlineto{\pgfqpoint{1.144023in}{3.841576in}}%
\pgfusepath{stroke}%
\end{pgfscope}%
\begin{pgfscope}%
\pgfpathrectangle{\pgfqpoint{0.892500in}{0.533500in}}{\pgfqpoint{5.533500in}{3.734500in}}%
\pgfusepath{clip}%
\pgfsetrectcap%
\pgfsetroundjoin%
\pgfsetlinewidth{1.505625pt}%
\definecolor{currentstroke}{rgb}{1.000000,0.000000,0.000000}%
\pgfsetstrokecolor{currentstroke}%
\pgfsetdash{}{0pt}%
\pgfpathmoveto{\pgfqpoint{1.144023in}{3.841576in}}%
\pgfpathlineto{\pgfqpoint{1.194835in}{3.713442in}}%
\pgfpathlineto{\pgfqpoint{1.245648in}{3.593564in}}%
\pgfpathlineto{\pgfqpoint{1.296461in}{3.481130in}}%
\pgfpathlineto{\pgfqpoint{1.347273in}{3.375435in}}%
\pgfpathlineto{\pgfqpoint{1.398086in}{3.275860in}}%
\pgfpathlineto{\pgfqpoint{1.448899in}{3.181863in}}%
\pgfpathlineto{\pgfqpoint{1.499711in}{3.092964in}}%
\pgfpathlineto{\pgfqpoint{1.550524in}{3.008739in}}%
\pgfpathlineto{\pgfqpoint{1.601337in}{2.928811in}}%
\pgfpathlineto{\pgfqpoint{1.652149in}{2.852842in}}%
\pgfpathlineto{\pgfqpoint{1.702962in}{2.780531in}}%
\pgfpathlineto{\pgfqpoint{1.753775in}{2.711606in}}%
\pgfpathlineto{\pgfqpoint{1.804587in}{2.645824in}}%
\pgfpathlineto{\pgfqpoint{1.855400in}{2.582962in}}%
\pgfpathlineto{\pgfqpoint{1.906213in}{2.522821in}}%
\pgfpathlineto{\pgfqpoint{1.957025in}{2.465220in}}%
\pgfpathlineto{\pgfqpoint{2.007838in}{2.409992in}}%
\pgfpathlineto{\pgfqpoint{2.058651in}{2.356985in}}%
\pgfpathlineto{\pgfqpoint{2.109463in}{2.306063in}}%
\pgfpathlineto{\pgfqpoint{2.160276in}{2.257098in}}%
\pgfpathlineto{\pgfqpoint{2.211089in}{2.209973in}}%
\pgfpathlineto{\pgfqpoint{2.261902in}{2.164582in}}%
\pgfpathlineto{\pgfqpoint{2.312714in}{2.120825in}}%
\pgfpathlineto{\pgfqpoint{2.363527in}{2.078612in}}%
\pgfpathlineto{\pgfqpoint{2.414340in}{2.037857in}}%
\pgfpathlineto{\pgfqpoint{2.465152in}{1.998483in}}%
\pgfpathlineto{\pgfqpoint{2.515965in}{1.960417in}}%
\pgfpathlineto{\pgfqpoint{2.566778in}{1.923592in}}%
\pgfpathlineto{\pgfqpoint{2.617590in}{1.887944in}}%
\pgfpathlineto{\pgfqpoint{2.668403in}{1.853415in}}%
\pgfpathlineto{\pgfqpoint{2.719216in}{1.819950in}}%
\pgfpathlineto{\pgfqpoint{2.770028in}{1.787500in}}%
\pgfpathlineto{\pgfqpoint{2.820841in}{1.756014in}}%
\pgfpathlineto{\pgfqpoint{2.871654in}{1.725450in}}%
\pgfpathlineto{\pgfqpoint{2.922466in}{1.695765in}}%
\pgfpathlineto{\pgfqpoint{2.973279in}{1.666919in}}%
\pgfpathlineto{\pgfqpoint{3.024092in}{1.638877in}}%
\pgfpathlineto{\pgfqpoint{3.074904in}{1.611602in}}%
\pgfpathlineto{\pgfqpoint{3.125717in}{1.585062in}}%
\pgfpathlineto{\pgfqpoint{3.176530in}{1.559226in}}%
\pgfpathlineto{\pgfqpoint{3.227342in}{1.534065in}}%
\pgfpathlineto{\pgfqpoint{3.278155in}{1.509553in}}%
\pgfpathlineto{\pgfqpoint{3.328968in}{1.485661in}}%
\pgfpathlineto{\pgfqpoint{3.379780in}{1.462367in}}%
\pgfpathlineto{\pgfqpoint{3.430593in}{1.439646in}}%
\pgfpathlineto{\pgfqpoint{3.481406in}{1.417477in}}%
\pgfpathlineto{\pgfqpoint{3.532218in}{1.395838in}}%
\pgfpathlineto{\pgfqpoint{3.583031in}{1.374711in}}%
\pgfpathlineto{\pgfqpoint{3.633844in}{1.354076in}}%
\pgfpathlineto{\pgfqpoint{3.684656in}{1.333914in}}%
\pgfpathlineto{\pgfqpoint{3.735469in}{1.314210in}}%
\pgfpathlineto{\pgfqpoint{3.786282in}{1.294947in}}%
\pgfpathlineto{\pgfqpoint{3.837094in}{1.276109in}}%
\pgfpathlineto{\pgfqpoint{3.887907in}{1.257683in}}%
\pgfpathlineto{\pgfqpoint{3.938720in}{1.239653in}}%
\pgfpathlineto{\pgfqpoint{3.989532in}{1.222006in}}%
\pgfpathlineto{\pgfqpoint{4.040345in}{1.204730in}}%
\pgfpathlineto{\pgfqpoint{4.091158in}{1.187813in}}%
\pgfpathlineto{\pgfqpoint{4.141970in}{1.171242in}}%
\pgfpathlineto{\pgfqpoint{4.192783in}{1.155008in}}%
\pgfpathlineto{\pgfqpoint{4.243596in}{1.139098in}}%
\pgfpathlineto{\pgfqpoint{4.294408in}{1.123503in}}%
\pgfpathlineto{\pgfqpoint{4.345221in}{1.108214in}}%
\pgfpathlineto{\pgfqpoint{4.396034in}{1.093220in}}%
\pgfpathlineto{\pgfqpoint{4.446846in}{1.078513in}}%
\pgfpathlineto{\pgfqpoint{4.497659in}{1.064084in}}%
\pgfpathlineto{\pgfqpoint{4.548472in}{1.049926in}}%
\pgfpathlineto{\pgfqpoint{4.599284in}{1.036029in}}%
\pgfpathlineto{\pgfqpoint{4.650097in}{1.022387in}}%
\pgfpathlineto{\pgfqpoint{4.700910in}{1.008992in}}%
\pgfpathlineto{\pgfqpoint{4.751722in}{0.995838in}}%
\pgfpathlineto{\pgfqpoint{4.802535in}{0.982917in}}%
\pgfpathlineto{\pgfqpoint{4.853348in}{0.970222in}}%
\pgfpathlineto{\pgfqpoint{4.904160in}{0.957749in}}%
\pgfpathlineto{\pgfqpoint{4.954973in}{0.945490in}}%
\pgfpathlineto{\pgfqpoint{5.005786in}{0.933440in}}%
\pgfpathlineto{\pgfqpoint{5.056598in}{0.921593in}}%
\pgfpathlineto{\pgfqpoint{5.107411in}{0.909944in}}%
\pgfpathlineto{\pgfqpoint{5.158224in}{0.898488in}}%
\pgfpathlineto{\pgfqpoint{5.209037in}{0.887220in}}%
\pgfpathlineto{\pgfqpoint{5.259849in}{0.876134in}}%
\pgfpathlineto{\pgfqpoint{5.310662in}{0.865227in}}%
\pgfpathlineto{\pgfqpoint{5.361475in}{0.854494in}}%
\pgfpathlineto{\pgfqpoint{5.412287in}{0.843930in}}%
\pgfpathlineto{\pgfqpoint{5.463100in}{0.833531in}}%
\pgfpathlineto{\pgfqpoint{5.513913in}{0.823294in}}%
\pgfpathlineto{\pgfqpoint{5.564725in}{0.813213in}}%
\pgfpathlineto{\pgfqpoint{5.615538in}{0.803287in}}%
\pgfpathlineto{\pgfqpoint{5.666351in}{0.793510in}}%
\pgfpathlineto{\pgfqpoint{5.717163in}{0.783879in}}%
\pgfpathlineto{\pgfqpoint{5.767976in}{0.774391in}}%
\pgfpathlineto{\pgfqpoint{5.818789in}{0.765042in}}%
\pgfpathlineto{\pgfqpoint{5.869601in}{0.755830in}}%
\pgfpathlineto{\pgfqpoint{5.920414in}{0.746751in}}%
\pgfpathlineto{\pgfqpoint{5.971227in}{0.737802in}}%
\pgfpathlineto{\pgfqpoint{6.022039in}{0.728980in}}%
\pgfpathlineto{\pgfqpoint{6.072852in}{0.720283in}}%
\pgfpathlineto{\pgfqpoint{6.123665in}{0.711707in}}%
\pgfpathlineto{\pgfqpoint{6.174477in}{0.703250in}}%
\pgfusepath{stroke}%
\end{pgfscope}%
\begin{pgfscope}%
\pgfsetrectcap%
\pgfsetmiterjoin%
\pgfsetlinewidth{0.803000pt}%
\definecolor{currentstroke}{rgb}{0.000000,0.000000,0.000000}%
\pgfsetstrokecolor{currentstroke}%
\pgfsetdash{}{0pt}%
\pgfpathmoveto{\pgfqpoint{0.892500in}{0.533500in}}%
\pgfpathlineto{\pgfqpoint{0.892500in}{4.268000in}}%
\pgfusepath{stroke}%
\end{pgfscope}%
\begin{pgfscope}%
\pgfsetrectcap%
\pgfsetmiterjoin%
\pgfsetlinewidth{0.803000pt}%
\definecolor{currentstroke}{rgb}{0.000000,0.000000,0.000000}%
\pgfsetstrokecolor{currentstroke}%
\pgfsetdash{}{0pt}%
\pgfpathmoveto{\pgfqpoint{6.426000in}{0.533500in}}%
\pgfpathlineto{\pgfqpoint{6.426000in}{4.268000in}}%
\pgfusepath{stroke}%
\end{pgfscope}%
\begin{pgfscope}%
\pgfsetrectcap%
\pgfsetmiterjoin%
\pgfsetlinewidth{0.803000pt}%
\definecolor{currentstroke}{rgb}{0.000000,0.000000,0.000000}%
\pgfsetstrokecolor{currentstroke}%
\pgfsetdash{}{0pt}%
\pgfpathmoveto{\pgfqpoint{0.892500in}{0.533500in}}%
\pgfpathlineto{\pgfqpoint{6.426000in}{0.533500in}}%
\pgfusepath{stroke}%
\end{pgfscope}%
\begin{pgfscope}%
\pgfsetrectcap%
\pgfsetmiterjoin%
\pgfsetlinewidth{0.803000pt}%
\definecolor{currentstroke}{rgb}{0.000000,0.000000,0.000000}%
\pgfsetstrokecolor{currentstroke}%
\pgfsetdash{}{0pt}%
\pgfpathmoveto{\pgfqpoint{0.892500in}{4.268000in}}%
\pgfpathlineto{\pgfqpoint{6.426000in}{4.268000in}}%
\pgfusepath{stroke}%
\end{pgfscope}%
\begin{pgfscope}%
\definecolor{textcolor}{rgb}{0.000000,0.000000,0.000000}%
\pgfsetstrokecolor{textcolor}%
\pgfsetfillcolor{textcolor}%
\pgftext[x=6.237358in,y=0.703250in,left,base]{\color{textcolor}{\rmfamily\fontsize{10.000000}{12.000000}\selectfont\catcode`\^=\active\def^{\ifmmode\sp\else\^{}\fi}\catcode`\%=\active\def%{\%}State 1}}%
\end{pgfscope}%
\begin{pgfscope}%
\definecolor{textcolor}{rgb}{0.000000,0.000000,0.000000}%
\pgfsetstrokecolor{textcolor}%
\pgfsetfillcolor{textcolor}%
\pgftext[x=1.772830in,y=4.127416in,left,base]{\color{textcolor}{\rmfamily\fontsize{10.000000}{12.000000}\selectfont\catcode`\^=\active\def^{\ifmmode\sp\else\^{}\fi}\catcode`\%=\active\def%{\%}State 2}}%
\end{pgfscope}%
\begin{pgfscope}%
\definecolor{textcolor}{rgb}{0.000000,0.000000,0.000000}%
\pgfsetstrokecolor{textcolor}%
\pgfsetfillcolor{textcolor}%
\pgftext[x=1.144023in,y=4.127416in,left,base]{\color{textcolor}{\rmfamily\fontsize{10.000000}{12.000000}\selectfont\catcode`\^=\active\def^{\ifmmode\sp\else\^{}\fi}\catcode`\%=\active\def%{\%}State 3}}%
\end{pgfscope}%
\begin{pgfscope}%
\definecolor{textcolor}{rgb}{0.000000,0.000000,0.000000}%
\pgfsetstrokecolor{textcolor}%
\pgfsetfillcolor{textcolor}%
\pgftext[x=1.206903in,y=3.841576in,left,base]{\color{textcolor}{\rmfamily\fontsize{10.000000}{12.000000}\selectfont\catcode`\^=\active\def^{\ifmmode\sp\else\^{}\fi}\catcode`\%=\active\def%{\%}State 4}}%
\end{pgfscope}%
\begin{pgfscope}%
\pgfsetbuttcap%
\pgfsetmiterjoin%
\definecolor{currentfill}{rgb}{1.000000,1.000000,1.000000}%
\pgfsetfillcolor{currentfill}%
\pgfsetfillopacity{0.800000}%
\pgfsetlinewidth{1.003750pt}%
\definecolor{currentstroke}{rgb}{0.800000,0.800000,0.800000}%
\pgfsetstrokecolor{currentstroke}%
\pgfsetstrokeopacity{0.800000}%
\pgfsetdash{}{0pt}%
\pgfpathmoveto{\pgfqpoint{5.121801in}{3.341460in}}%
\pgfpathlineto{\pgfqpoint{6.328778in}{3.341460in}}%
\pgfpathquadraticcurveto{\pgfqpoint{6.356556in}{3.341460in}}{\pgfqpoint{6.356556in}{3.369238in}}%
\pgfpathlineto{\pgfqpoint{6.356556in}{4.170778in}}%
\pgfpathquadraticcurveto{\pgfqpoint{6.356556in}{4.198556in}}{\pgfqpoint{6.328778in}{4.198556in}}%
\pgfpathlineto{\pgfqpoint{5.121801in}{4.198556in}}%
\pgfpathquadraticcurveto{\pgfqpoint{5.094023in}{4.198556in}}{\pgfqpoint{5.094023in}{4.170778in}}%
\pgfpathlineto{\pgfqpoint{5.094023in}{3.369238in}}%
\pgfpathquadraticcurveto{\pgfqpoint{5.094023in}{3.341460in}}{\pgfqpoint{5.121801in}{3.341460in}}%
\pgfpathlineto{\pgfqpoint{5.121801in}{3.341460in}}%
\pgfpathclose%
\pgfusepath{stroke,fill}%
\end{pgfscope}%
\begin{pgfscope}%
\pgfsetrectcap%
\pgfsetroundjoin%
\pgfsetlinewidth{1.505625pt}%
\definecolor{currentstroke}{rgb}{0.501961,0.000000,0.501961}%
\pgfsetstrokecolor{currentstroke}%
\pgfsetdash{}{0pt}%
\pgfpathmoveto{\pgfqpoint{5.149579in}{4.086088in}}%
\pgfpathlineto{\pgfqpoint{5.288468in}{4.086088in}}%
\pgfpathlineto{\pgfqpoint{5.427356in}{4.086088in}}%
\pgfusepath{stroke}%
\end{pgfscope}%
\begin{pgfscope}%
\definecolor{textcolor}{rgb}{0.000000,0.000000,0.000000}%
\pgfsetstrokecolor{textcolor}%
\pgfsetfillcolor{textcolor}%
\pgftext[x=5.538468in,y=4.037477in,left,base]{\color{textcolor}{\rmfamily\fontsize{10.000000}{12.000000}\selectfont\catcode`\^=\active\def^{\ifmmode\sp\else\^{}\fi}\catcode`\%=\active\def%{\%}Isothermal}}%
\end{pgfscope}%
\begin{pgfscope}%
\pgfsetrectcap%
\pgfsetroundjoin%
\pgfsetlinewidth{1.505625pt}%
\definecolor{currentstroke}{rgb}{0.000000,0.501961,0.000000}%
\pgfsetstrokecolor{currentstroke}%
\pgfsetdash{}{0pt}%
\pgfpathmoveto{\pgfqpoint{5.149579in}{3.882231in}}%
\pgfpathlineto{\pgfqpoint{5.288468in}{3.882231in}}%
\pgfpathlineto{\pgfqpoint{5.427356in}{3.882231in}}%
\pgfusepath{stroke}%
\end{pgfscope}%
\begin{pgfscope}%
\definecolor{textcolor}{rgb}{0.000000,0.000000,0.000000}%
\pgfsetstrokecolor{textcolor}%
\pgfsetfillcolor{textcolor}%
\pgftext[x=5.538468in,y=3.833620in,left,base]{\color{textcolor}{\rmfamily\fontsize{10.000000}{12.000000}\selectfont\catcode`\^=\active\def^{\ifmmode\sp\else\^{}\fi}\catcode`\%=\active\def%{\%}Isochoric}}%
\end{pgfscope}%
\begin{pgfscope}%
\pgfsetrectcap%
\pgfsetroundjoin%
\pgfsetlinewidth{1.505625pt}%
\definecolor{currentstroke}{rgb}{0.000000,0.000000,1.000000}%
\pgfsetstrokecolor{currentstroke}%
\pgfsetdash{}{0pt}%
\pgfpathmoveto{\pgfqpoint{5.149579in}{3.678374in}}%
\pgfpathlineto{\pgfqpoint{5.288468in}{3.678374in}}%
\pgfpathlineto{\pgfqpoint{5.427356in}{3.678374in}}%
\pgfusepath{stroke}%
\end{pgfscope}%
\begin{pgfscope}%
\definecolor{textcolor}{rgb}{0.000000,0.000000,0.000000}%
\pgfsetstrokecolor{textcolor}%
\pgfsetfillcolor{textcolor}%
\pgftext[x=5.538468in,y=3.629762in,left,base]{\color{textcolor}{\rmfamily\fontsize{10.000000}{12.000000}\selectfont\catcode`\^=\active\def^{\ifmmode\sp\else\^{}\fi}\catcode`\%=\active\def%{\%}Isobaric}}%
\end{pgfscope}%
\begin{pgfscope}%
\pgfsetrectcap%
\pgfsetroundjoin%
\pgfsetlinewidth{1.505625pt}%
\definecolor{currentstroke}{rgb}{1.000000,0.000000,0.000000}%
\pgfsetstrokecolor{currentstroke}%
\pgfsetdash{}{0pt}%
\pgfpathmoveto{\pgfqpoint{5.149579in}{3.474516in}}%
\pgfpathlineto{\pgfqpoint{5.288468in}{3.474516in}}%
\pgfpathlineto{\pgfqpoint{5.427356in}{3.474516in}}%
\pgfusepath{stroke}%
\end{pgfscope}%
\begin{pgfscope}%
\definecolor{textcolor}{rgb}{0.000000,0.000000,0.000000}%
\pgfsetstrokecolor{textcolor}%
\pgfsetfillcolor{textcolor}%
\pgftext[x=5.538468in,y=3.425905in,left,base]{\color{textcolor}{\rmfamily\fontsize{10.000000}{12.000000}\selectfont\catcode`\^=\active\def^{\ifmmode\sp\else\^{}\fi}\catcode`\%=\active\def%{\%}Adiabatic}}%
\end{pgfscope}%
\end{pgfpicture}%
\makeatother%
\endgroup%
}
        \caption{
          PV graph of the four-step cycle as described in
          Problem 3.19.
        }
        \label{fig:s19}
      \end{figure}
    \item ~
      \begin{enumerate}[label=\roman*.]
        \item State 2:
          The path from state 1 to state 2 is isothermal so
          $T_{2}=600~\unit{ \kelvin }$
        \item State 3:
          Solving for $T_{3}$ using the ideal gas equation yields
          $400~\unit{ \kelvin }$
        \item State 4:
          The path from state 3 to state 4 is isobaric so
          $P_{2}=2~\unit{ \bar }$. Using the fact that state 4 to
          state 1 is adiabatic, we can use
          $$TP^{(1-\gamma)/\gamma}=\text{constant}$$ to solve for
          $T_{4}$ which is 378.83~\unit{ \kelvin }
      \end{enumerate}
    \item ~
      \begin{enumerate}[label=\roman*.]
        \item Path 1
          \begin{gather*}
            Q=-W\\
            Q=P\,dV\\
            Q=RT\ln\frac{P_{1}}{P_{2}}
          \end{gather*}
          \begin{empheq}[box=\widefbox]{gather*}
            \Delta H=\Delta U=0\\
            Q=6.01~\unit{ \kilo\joule\per\mole }\\
            W=-6.01~\unit{ \kilo\joule\per\mole }
          \end{empheq}
        \item Path 2
          \begin{gather*}
            Q=\Delta U\\
            Q=C_{V}\Delta T\\
            \Delta H=C_{P}\Delta T
          \end{gather*}
          \begin{empheq}[box=\widefbox]{gather*}
            W=0\\
            Q=\Delta U=-4.16~\unit{ \kilo\joule\per\mole }\\
            \Delta H=-5.82~\unit{ \kilo\joule\per\mole }
          \end{empheq}
        \item Path 3
          \begin{gather*}
            Q=\Delta H\\
            Q=C_{P}\Delta T\\
            \Delta U=C_{V}\Delta T\\
            W=\left( C_{V}-C_{P} \right)\left( \Delta T \right)
          \end{gather*}
          \begin{empheq}[box=\widefbox]{gather*}
            Q=\Delta H=-0.62~\unit{ \kilo\joule\per\mole }\\
            \Delta U=-0.44~\unit{ \kilo\joule\per\mole }\\
            W=-0.18~\unit{ \kilo\joule\per\mole }
          \end{empheq}
      \end{enumerate}
  \end{enumerate}
\end{solution}

\section*{Problem 3.20}
\addcontentsline{toc}{section}{Problem 3.20}
An ideal gas initially at $\SI{300}{\kelvin}$ and $\SI{1}{\bar}$
undergoes a three-step mechanically reversible cycle in a closed
system. In step $12$, pressure increases isothermally to
$\SI{5}{\bar}$; in step $23$, pressure increases at constant volume;
and in step $31$, the gas returns adiabatically to its initial state.
Take $C_P = \frac{7}{2}R$ and $C_V = \frac{5}{2}R$.
\begin{enumerate}[label=(\alph*)]
  \item Sketch the cycle on a $PV$ diagram.
  \item Determine (where unknown) $V$, $T$, and $P$ for states $1$,
    $2$, and $3$.
  \item Calculate $Q$, $W$, $\Delta U$, and $\Delta H$ for each step
    of the cycle.
\end{enumerate}

\begin{solution}
  \begin{enumerate}[label=(\alph*)]
    \item Figure~\ref{fig:s20}
      \begin{figure}[h!]
        \centering
        \scalebox{0.5}{%% Creator: Matplotlib, PGF backend
%%
%% To include the figure in your LaTeX document, write
%%   \input{<filename>.pgf}
%%
%% Make sure the required packages are loaded in your preamble
%%   \usepackage{pgf}
%%
%% Also ensure that all the required font packages are loaded; for instance,
%% the lmodern package is sometimes necessary when using math font.
%%   \usepackage{lmodern}
%%
%% Figures using additional raster images can only be included by \input if
%% they are in the same directory as the main LaTeX file. For loading figures
%% from other directories you can use the `import` package
%%   \usepackage{import}
%%
%% and then include the figures with
%%   \import{<path to file>}{<filename>.pgf}
%%
%% Matplotlib used the following preamble
%%   \def\mathdefault#1{#1}
%%   \everymath=\expandafter{\the\everymath\displaystyle}
%%   \IfFileExists{scrextend.sty}{
%%     \usepackage[fontsize=10.000000pt]{scrextend}
%%   }{
%%     \renewcommand{\normalsize}{\fontsize{10.000000}{12.000000}\selectfont}
%%     \normalsize
%%   }
%%   
%%   \ifdefined\pdftexversion\else  % non-pdftex case.
%%     \usepackage{fontspec}
%%     \setmainfont{DejaVuSerif.ttf}[Path=\detokenize{C:/Users/Jian/AppData/Local/Programs/Python/Python313/Lib/site-packages/matplotlib/mpl-data/fonts/ttf/}]
%%     \setsansfont{DejaVuSans.ttf}[Path=\detokenize{C:/Users/Jian/AppData/Local/Programs/Python/Python313/Lib/site-packages/matplotlib/mpl-data/fonts/ttf/}]
%%     \setmonofont{DejaVuSansMono.ttf}[Path=\detokenize{C:/Users/Jian/AppData/Local/Programs/Python/Python313/Lib/site-packages/matplotlib/mpl-data/fonts/ttf/}]
%%   \fi
%%   \makeatletter\@ifpackageloaded{underscore}{}{\usepackage[strings]{underscore}}\makeatother
%%
\begingroup%
\makeatletter%
\begin{pgfpicture}%
\pgfpathrectangle{\pgfpointorigin}{\pgfqpoint{7.320000in}{5.720000in}}%
\pgfusepath{use as bounding box, clip}%
\begin{pgfscope}%
\pgfsetbuttcap%
\pgfsetmiterjoin%
\definecolor{currentfill}{rgb}{1.000000,1.000000,1.000000}%
\pgfsetfillcolor{currentfill}%
\pgfsetlinewidth{0.000000pt}%
\definecolor{currentstroke}{rgb}{1.000000,1.000000,1.000000}%
\pgfsetstrokecolor{currentstroke}%
\pgfsetdash{}{0pt}%
\pgfpathmoveto{\pgfqpoint{0.000000in}{0.000000in}}%
\pgfpathlineto{\pgfqpoint{7.320000in}{0.000000in}}%
\pgfpathlineto{\pgfqpoint{7.320000in}{5.720000in}}%
\pgfpathlineto{\pgfqpoint{0.000000in}{5.720000in}}%
\pgfpathlineto{\pgfqpoint{0.000000in}{0.000000in}}%
\pgfpathclose%
\pgfusepath{fill}%
\end{pgfscope}%
\begin{pgfscope}%
\pgfsetbuttcap%
\pgfsetmiterjoin%
\definecolor{currentfill}{rgb}{1.000000,1.000000,1.000000}%
\pgfsetfillcolor{currentfill}%
\pgfsetlinewidth{0.000000pt}%
\definecolor{currentstroke}{rgb}{0.000000,0.000000,0.000000}%
\pgfsetstrokecolor{currentstroke}%
\pgfsetstrokeopacity{0.000000}%
\pgfsetdash{}{0pt}%
\pgfpathmoveto{\pgfqpoint{0.915000in}{0.629200in}}%
\pgfpathlineto{\pgfqpoint{6.588000in}{0.629200in}}%
\pgfpathlineto{\pgfqpoint{6.588000in}{5.033600in}}%
\pgfpathlineto{\pgfqpoint{0.915000in}{5.033600in}}%
\pgfpathlineto{\pgfqpoint{0.915000in}{0.629200in}}%
\pgfpathclose%
\pgfusepath{fill}%
\end{pgfscope}%
\begin{pgfscope}%
\pgfpathrectangle{\pgfqpoint{0.915000in}{0.629200in}}{\pgfqpoint{5.673000in}{4.404400in}}%
\pgfusepath{clip}%
\pgfsetbuttcap%
\pgfsetroundjoin%
\pgfsetlinewidth{1.003750pt}%
\definecolor{currentstroke}{rgb}{0.000000,0.000000,0.000000}%
\pgfsetstrokecolor{currentstroke}%
\pgfsetdash{}{0pt}%
\pgfpathmoveto{\pgfqpoint{1.172864in}{4.811440in}}%
\pgfpathcurveto{\pgfqpoint{1.178688in}{4.811440in}}{\pgfqpoint{1.184274in}{4.813754in}}{\pgfqpoint{1.188392in}{4.817872in}}%
\pgfpathcurveto{\pgfqpoint{1.192510in}{4.821990in}}{\pgfqpoint{1.194824in}{4.827576in}}{\pgfqpoint{1.194824in}{4.833400in}}%
\pgfpathcurveto{\pgfqpoint{1.194824in}{4.839224in}}{\pgfqpoint{1.192510in}{4.844810in}}{\pgfqpoint{1.188392in}{4.848928in}}%
\pgfpathcurveto{\pgfqpoint{1.184274in}{4.853046in}}{\pgfqpoint{1.178688in}{4.855360in}}{\pgfqpoint{1.172864in}{4.855360in}}%
\pgfpathcurveto{\pgfqpoint{1.167040in}{4.855360in}}{\pgfqpoint{1.161454in}{4.853046in}}{\pgfqpoint{1.157335in}{4.848928in}}%
\pgfpathcurveto{\pgfqpoint{1.153217in}{4.844810in}}{\pgfqpoint{1.150903in}{4.839224in}}{\pgfqpoint{1.150903in}{4.833400in}}%
\pgfpathcurveto{\pgfqpoint{1.150903in}{4.827576in}}{\pgfqpoint{1.153217in}{4.821990in}}{\pgfqpoint{1.157335in}{4.817872in}}%
\pgfpathcurveto{\pgfqpoint{1.161454in}{4.813754in}}{\pgfqpoint{1.167040in}{4.811440in}}{\pgfqpoint{1.172864in}{4.811440in}}%
\pgfpathlineto{\pgfqpoint{1.172864in}{4.811440in}}%
\pgfpathclose%
\pgfusepath{stroke}%
\end{pgfscope}%
\begin{pgfscope}%
\pgfpathrectangle{\pgfqpoint{0.915000in}{0.629200in}}{\pgfqpoint{5.673000in}{4.404400in}}%
\pgfusepath{clip}%
\pgfsetbuttcap%
\pgfsetroundjoin%
\pgfsetlinewidth{1.003750pt}%
\definecolor{currentstroke}{rgb}{0.000000,0.000000,0.000000}%
\pgfsetstrokecolor{currentstroke}%
\pgfsetdash{}{0pt}%
\pgfpathmoveto{\pgfqpoint{3.594610in}{0.807440in}}%
\pgfpathcurveto{\pgfqpoint{3.600434in}{0.807440in}}{\pgfqpoint{3.606020in}{0.809754in}}{\pgfqpoint{3.610139in}{0.813872in}}%
\pgfpathcurveto{\pgfqpoint{3.614257in}{0.817990in}}{\pgfqpoint{3.616571in}{0.823576in}}{\pgfqpoint{3.616571in}{0.829400in}}%
\pgfpathcurveto{\pgfqpoint{3.616571in}{0.835224in}}{\pgfqpoint{3.614257in}{0.840810in}}{\pgfqpoint{3.610139in}{0.844928in}}%
\pgfpathcurveto{\pgfqpoint{3.606020in}{0.849046in}}{\pgfqpoint{3.600434in}{0.851360in}}{\pgfqpoint{3.594610in}{0.851360in}}%
\pgfpathcurveto{\pgfqpoint{3.588786in}{0.851360in}}{\pgfqpoint{3.583200in}{0.849046in}}{\pgfqpoint{3.579082in}{0.844928in}}%
\pgfpathcurveto{\pgfqpoint{3.574964in}{0.840810in}}{\pgfqpoint{3.572650in}{0.835224in}}{\pgfqpoint{3.572650in}{0.829400in}}%
\pgfpathcurveto{\pgfqpoint{3.572650in}{0.823576in}}{\pgfqpoint{3.574964in}{0.817990in}}{\pgfqpoint{3.579082in}{0.813872in}}%
\pgfpathcurveto{\pgfqpoint{3.583200in}{0.809754in}}{\pgfqpoint{3.588786in}{0.807440in}}{\pgfqpoint{3.594610in}{0.807440in}}%
\pgfpathlineto{\pgfqpoint{3.594610in}{0.807440in}}%
\pgfpathclose%
\pgfusepath{stroke}%
\end{pgfscope}%
\begin{pgfscope}%
\pgfpathrectangle{\pgfqpoint{0.915000in}{0.629200in}}{\pgfqpoint{5.673000in}{4.404400in}}%
\pgfusepath{clip}%
\pgfsetbuttcap%
\pgfsetroundjoin%
\pgfsetlinewidth{1.003750pt}%
\definecolor{currentstroke}{rgb}{0.000000,0.000000,0.000000}%
\pgfsetstrokecolor{currentstroke}%
\pgfsetdash{}{0pt}%
\pgfpathmoveto{\pgfqpoint{6.330136in}{0.807440in}}%
\pgfpathcurveto{\pgfqpoint{6.335960in}{0.807440in}}{\pgfqpoint{6.341546in}{0.809754in}}{\pgfqpoint{6.345665in}{0.813872in}}%
\pgfpathcurveto{\pgfqpoint{6.349783in}{0.817990in}}{\pgfqpoint{6.352097in}{0.823576in}}{\pgfqpoint{6.352097in}{0.829400in}}%
\pgfpathcurveto{\pgfqpoint{6.352097in}{0.835224in}}{\pgfqpoint{6.349783in}{0.840810in}}{\pgfqpoint{6.345665in}{0.844928in}}%
\pgfpathcurveto{\pgfqpoint{6.341546in}{0.849046in}}{\pgfqpoint{6.335960in}{0.851360in}}{\pgfqpoint{6.330136in}{0.851360in}}%
\pgfpathcurveto{\pgfqpoint{6.324312in}{0.851360in}}{\pgfqpoint{6.318726in}{0.849046in}}{\pgfqpoint{6.314608in}{0.844928in}}%
\pgfpathcurveto{\pgfqpoint{6.310490in}{0.840810in}}{\pgfqpoint{6.308176in}{0.835224in}}{\pgfqpoint{6.308176in}{0.829400in}}%
\pgfpathcurveto{\pgfqpoint{6.308176in}{0.823576in}}{\pgfqpoint{6.310490in}{0.817990in}}{\pgfqpoint{6.314608in}{0.813872in}}%
\pgfpathcurveto{\pgfqpoint{6.318726in}{0.809754in}}{\pgfqpoint{6.324312in}{0.807440in}}{\pgfqpoint{6.330136in}{0.807440in}}%
\pgfpathlineto{\pgfqpoint{6.330136in}{0.807440in}}%
\pgfpathclose%
\pgfusepath{stroke}%
\end{pgfscope}%
\begin{pgfscope}%
\pgfpathrectangle{\pgfqpoint{0.915000in}{0.629200in}}{\pgfqpoint{5.673000in}{4.404400in}}%
\pgfusepath{clip}%
\pgfsetrectcap%
\pgfsetroundjoin%
\pgfsetlinewidth{0.803000pt}%
\definecolor{currentstroke}{rgb}{0.690196,0.690196,0.690196}%
\pgfsetstrokecolor{currentstroke}%
\pgfsetdash{}{0pt}%
\pgfpathmoveto{\pgfqpoint{1.778300in}{0.629200in}}%
\pgfpathlineto{\pgfqpoint{1.778300in}{5.033600in}}%
\pgfusepath{stroke}%
\end{pgfscope}%
\begin{pgfscope}%
\pgfsetbuttcap%
\pgfsetroundjoin%
\definecolor{currentfill}{rgb}{0.000000,0.000000,0.000000}%
\pgfsetfillcolor{currentfill}%
\pgfsetlinewidth{0.803000pt}%
\definecolor{currentstroke}{rgb}{0.000000,0.000000,0.000000}%
\pgfsetstrokecolor{currentstroke}%
\pgfsetdash{}{0pt}%
\pgfsys@defobject{currentmarker}{\pgfqpoint{0.000000in}{-0.048611in}}{\pgfqpoint{0.000000in}{0.000000in}}{%
\pgfpathmoveto{\pgfqpoint{0.000000in}{0.000000in}}%
\pgfpathlineto{\pgfqpoint{0.000000in}{-0.048611in}}%
\pgfusepath{stroke,fill}%
}%
\begin{pgfscope}%
\pgfsys@transformshift{1.778300in}{0.629200in}%
\pgfsys@useobject{currentmarker}{}%
\end{pgfscope}%
\end{pgfscope}%
\begin{pgfscope}%
\definecolor{textcolor}{rgb}{0.000000,0.000000,0.000000}%
\pgfsetstrokecolor{textcolor}%
\pgfsetfillcolor{textcolor}%
\pgftext[x=1.778300in,y=0.531978in,,top]{\color{textcolor}{\sffamily\fontsize{10.000000}{12.000000}\selectfont\catcode`\^=\active\def^{\ifmmode\sp\else\^{}\fi}\catcode`\%=\active\def%{\%}2}}%
\end{pgfscope}%
\begin{pgfscope}%
\pgfpathrectangle{\pgfqpoint{0.915000in}{0.629200in}}{\pgfqpoint{5.673000in}{4.404400in}}%
\pgfusepath{clip}%
\pgfsetrectcap%
\pgfsetroundjoin%
\pgfsetlinewidth{0.803000pt}%
\definecolor{currentstroke}{rgb}{0.690196,0.690196,0.690196}%
\pgfsetstrokecolor{currentstroke}%
\pgfsetdash{}{0pt}%
\pgfpathmoveto{\pgfqpoint{2.989174in}{0.629200in}}%
\pgfpathlineto{\pgfqpoint{2.989174in}{5.033600in}}%
\pgfusepath{stroke}%
\end{pgfscope}%
\begin{pgfscope}%
\pgfsetbuttcap%
\pgfsetroundjoin%
\definecolor{currentfill}{rgb}{0.000000,0.000000,0.000000}%
\pgfsetfillcolor{currentfill}%
\pgfsetlinewidth{0.803000pt}%
\definecolor{currentstroke}{rgb}{0.000000,0.000000,0.000000}%
\pgfsetstrokecolor{currentstroke}%
\pgfsetdash{}{0pt}%
\pgfsys@defobject{currentmarker}{\pgfqpoint{0.000000in}{-0.048611in}}{\pgfqpoint{0.000000in}{0.000000in}}{%
\pgfpathmoveto{\pgfqpoint{0.000000in}{0.000000in}}%
\pgfpathlineto{\pgfqpoint{0.000000in}{-0.048611in}}%
\pgfusepath{stroke,fill}%
}%
\begin{pgfscope}%
\pgfsys@transformshift{2.989174in}{0.629200in}%
\pgfsys@useobject{currentmarker}{}%
\end{pgfscope}%
\end{pgfscope}%
\begin{pgfscope}%
\definecolor{textcolor}{rgb}{0.000000,0.000000,0.000000}%
\pgfsetstrokecolor{textcolor}%
\pgfsetfillcolor{textcolor}%
\pgftext[x=2.989174in,y=0.531978in,,top]{\color{textcolor}{\sffamily\fontsize{10.000000}{12.000000}\selectfont\catcode`\^=\active\def^{\ifmmode\sp\else\^{}\fi}\catcode`\%=\active\def%{\%}4}}%
\end{pgfscope}%
\begin{pgfscope}%
\pgfpathrectangle{\pgfqpoint{0.915000in}{0.629200in}}{\pgfqpoint{5.673000in}{4.404400in}}%
\pgfusepath{clip}%
\pgfsetrectcap%
\pgfsetroundjoin%
\pgfsetlinewidth{0.803000pt}%
\definecolor{currentstroke}{rgb}{0.690196,0.690196,0.690196}%
\pgfsetstrokecolor{currentstroke}%
\pgfsetdash{}{0pt}%
\pgfpathmoveto{\pgfqpoint{4.200047in}{0.629200in}}%
\pgfpathlineto{\pgfqpoint{4.200047in}{5.033600in}}%
\pgfusepath{stroke}%
\end{pgfscope}%
\begin{pgfscope}%
\pgfsetbuttcap%
\pgfsetroundjoin%
\definecolor{currentfill}{rgb}{0.000000,0.000000,0.000000}%
\pgfsetfillcolor{currentfill}%
\pgfsetlinewidth{0.803000pt}%
\definecolor{currentstroke}{rgb}{0.000000,0.000000,0.000000}%
\pgfsetstrokecolor{currentstroke}%
\pgfsetdash{}{0pt}%
\pgfsys@defobject{currentmarker}{\pgfqpoint{0.000000in}{-0.048611in}}{\pgfqpoint{0.000000in}{0.000000in}}{%
\pgfpathmoveto{\pgfqpoint{0.000000in}{0.000000in}}%
\pgfpathlineto{\pgfqpoint{0.000000in}{-0.048611in}}%
\pgfusepath{stroke,fill}%
}%
\begin{pgfscope}%
\pgfsys@transformshift{4.200047in}{0.629200in}%
\pgfsys@useobject{currentmarker}{}%
\end{pgfscope}%
\end{pgfscope}%
\begin{pgfscope}%
\definecolor{textcolor}{rgb}{0.000000,0.000000,0.000000}%
\pgfsetstrokecolor{textcolor}%
\pgfsetfillcolor{textcolor}%
\pgftext[x=4.200047in,y=0.531978in,,top]{\color{textcolor}{\sffamily\fontsize{10.000000}{12.000000}\selectfont\catcode`\^=\active\def^{\ifmmode\sp\else\^{}\fi}\catcode`\%=\active\def%{\%}6}}%
\end{pgfscope}%
\begin{pgfscope}%
\pgfpathrectangle{\pgfqpoint{0.915000in}{0.629200in}}{\pgfqpoint{5.673000in}{4.404400in}}%
\pgfusepath{clip}%
\pgfsetrectcap%
\pgfsetroundjoin%
\pgfsetlinewidth{0.803000pt}%
\definecolor{currentstroke}{rgb}{0.690196,0.690196,0.690196}%
\pgfsetstrokecolor{currentstroke}%
\pgfsetdash{}{0pt}%
\pgfpathmoveto{\pgfqpoint{5.410920in}{0.629200in}}%
\pgfpathlineto{\pgfqpoint{5.410920in}{5.033600in}}%
\pgfusepath{stroke}%
\end{pgfscope}%
\begin{pgfscope}%
\pgfsetbuttcap%
\pgfsetroundjoin%
\definecolor{currentfill}{rgb}{0.000000,0.000000,0.000000}%
\pgfsetfillcolor{currentfill}%
\pgfsetlinewidth{0.803000pt}%
\definecolor{currentstroke}{rgb}{0.000000,0.000000,0.000000}%
\pgfsetstrokecolor{currentstroke}%
\pgfsetdash{}{0pt}%
\pgfsys@defobject{currentmarker}{\pgfqpoint{0.000000in}{-0.048611in}}{\pgfqpoint{0.000000in}{0.000000in}}{%
\pgfpathmoveto{\pgfqpoint{0.000000in}{0.000000in}}%
\pgfpathlineto{\pgfqpoint{0.000000in}{-0.048611in}}%
\pgfusepath{stroke,fill}%
}%
\begin{pgfscope}%
\pgfsys@transformshift{5.410920in}{0.629200in}%
\pgfsys@useobject{currentmarker}{}%
\end{pgfscope}%
\end{pgfscope}%
\begin{pgfscope}%
\definecolor{textcolor}{rgb}{0.000000,0.000000,0.000000}%
\pgfsetstrokecolor{textcolor}%
\pgfsetfillcolor{textcolor}%
\pgftext[x=5.410920in,y=0.531978in,,top]{\color{textcolor}{\sffamily\fontsize{10.000000}{12.000000}\selectfont\catcode`\^=\active\def^{\ifmmode\sp\else\^{}\fi}\catcode`\%=\active\def%{\%}8}}%
\end{pgfscope}%
\begin{pgfscope}%
\definecolor{textcolor}{rgb}{0.000000,0.000000,0.000000}%
\pgfsetstrokecolor{textcolor}%
\pgfsetfillcolor{textcolor}%
\pgftext[x=3.751500in,y=0.342009in,,top]{\color{textcolor}{\sffamily\fontsize{14.000000}{16.800000}\selectfont\catcode`\^=\active\def^{\ifmmode\sp\else\^{}\fi}\catcode`\%=\active\def%{\%}Pressure/$\mathrm{bar}$}}%
\end{pgfscope}%
\begin{pgfscope}%
\pgfpathrectangle{\pgfqpoint{0.915000in}{0.629200in}}{\pgfqpoint{5.673000in}{4.404400in}}%
\pgfusepath{clip}%
\pgfsetrectcap%
\pgfsetroundjoin%
\pgfsetlinewidth{0.803000pt}%
\definecolor{currentstroke}{rgb}{0.690196,0.690196,0.690196}%
\pgfsetstrokecolor{currentstroke}%
\pgfsetdash{}{0pt}%
\pgfpathmoveto{\pgfqpoint{0.915000in}{0.831672in}}%
\pgfpathlineto{\pgfqpoint{6.588000in}{0.831672in}}%
\pgfusepath{stroke}%
\end{pgfscope}%
\begin{pgfscope}%
\pgfsetbuttcap%
\pgfsetroundjoin%
\definecolor{currentfill}{rgb}{0.000000,0.000000,0.000000}%
\pgfsetfillcolor{currentfill}%
\pgfsetlinewidth{0.803000pt}%
\definecolor{currentstroke}{rgb}{0.000000,0.000000,0.000000}%
\pgfsetstrokecolor{currentstroke}%
\pgfsetdash{}{0pt}%
\pgfsys@defobject{currentmarker}{\pgfqpoint{-0.048611in}{0.000000in}}{\pgfqpoint{-0.000000in}{0.000000in}}{%
\pgfpathmoveto{\pgfqpoint{-0.000000in}{0.000000in}}%
\pgfpathlineto{\pgfqpoint{-0.048611in}{0.000000in}}%
\pgfusepath{stroke,fill}%
}%
\begin{pgfscope}%
\pgfsys@transformshift{0.915000in}{0.831672in}%
\pgfsys@useobject{currentmarker}{}%
\end{pgfscope}%
\end{pgfscope}%
\begin{pgfscope}%
\definecolor{textcolor}{rgb}{0.000000,0.000000,0.000000}%
\pgfsetstrokecolor{textcolor}%
\pgfsetfillcolor{textcolor}%
\pgftext[x=0.331802in, y=0.778910in, left, base]{\color{textcolor}{\sffamily\fontsize{10.000000}{12.000000}\selectfont\catcode`\^=\active\def^{\ifmmode\sp\else\^{}\fi}\catcode`\%=\active\def%{\%}0.0050}}%
\end{pgfscope}%
\begin{pgfscope}%
\pgfpathrectangle{\pgfqpoint{0.915000in}{0.629200in}}{\pgfqpoint{5.673000in}{4.404400in}}%
\pgfusepath{clip}%
\pgfsetrectcap%
\pgfsetroundjoin%
\pgfsetlinewidth{0.803000pt}%
\definecolor{currentstroke}{rgb}{0.690196,0.690196,0.690196}%
\pgfsetstrokecolor{currentstroke}%
\pgfsetdash{}{0pt}%
\pgfpathmoveto{\pgfqpoint{0.915000in}{1.333308in}}%
\pgfpathlineto{\pgfqpoint{6.588000in}{1.333308in}}%
\pgfusepath{stroke}%
\end{pgfscope}%
\begin{pgfscope}%
\pgfsetbuttcap%
\pgfsetroundjoin%
\definecolor{currentfill}{rgb}{0.000000,0.000000,0.000000}%
\pgfsetfillcolor{currentfill}%
\pgfsetlinewidth{0.803000pt}%
\definecolor{currentstroke}{rgb}{0.000000,0.000000,0.000000}%
\pgfsetstrokecolor{currentstroke}%
\pgfsetdash{}{0pt}%
\pgfsys@defobject{currentmarker}{\pgfqpoint{-0.048611in}{0.000000in}}{\pgfqpoint{-0.000000in}{0.000000in}}{%
\pgfpathmoveto{\pgfqpoint{-0.000000in}{0.000000in}}%
\pgfpathlineto{\pgfqpoint{-0.048611in}{0.000000in}}%
\pgfusepath{stroke,fill}%
}%
\begin{pgfscope}%
\pgfsys@transformshift{0.915000in}{1.333308in}%
\pgfsys@useobject{currentmarker}{}%
\end{pgfscope}%
\end{pgfscope}%
\begin{pgfscope}%
\definecolor{textcolor}{rgb}{0.000000,0.000000,0.000000}%
\pgfsetstrokecolor{textcolor}%
\pgfsetfillcolor{textcolor}%
\pgftext[x=0.331802in, y=1.280546in, left, base]{\color{textcolor}{\sffamily\fontsize{10.000000}{12.000000}\selectfont\catcode`\^=\active\def^{\ifmmode\sp\else\^{}\fi}\catcode`\%=\active\def%{\%}0.0075}}%
\end{pgfscope}%
\begin{pgfscope}%
\pgfpathrectangle{\pgfqpoint{0.915000in}{0.629200in}}{\pgfqpoint{5.673000in}{4.404400in}}%
\pgfusepath{clip}%
\pgfsetrectcap%
\pgfsetroundjoin%
\pgfsetlinewidth{0.803000pt}%
\definecolor{currentstroke}{rgb}{0.690196,0.690196,0.690196}%
\pgfsetstrokecolor{currentstroke}%
\pgfsetdash{}{0pt}%
\pgfpathmoveto{\pgfqpoint{0.915000in}{1.834944in}}%
\pgfpathlineto{\pgfqpoint{6.588000in}{1.834944in}}%
\pgfusepath{stroke}%
\end{pgfscope}%
\begin{pgfscope}%
\pgfsetbuttcap%
\pgfsetroundjoin%
\definecolor{currentfill}{rgb}{0.000000,0.000000,0.000000}%
\pgfsetfillcolor{currentfill}%
\pgfsetlinewidth{0.803000pt}%
\definecolor{currentstroke}{rgb}{0.000000,0.000000,0.000000}%
\pgfsetstrokecolor{currentstroke}%
\pgfsetdash{}{0pt}%
\pgfsys@defobject{currentmarker}{\pgfqpoint{-0.048611in}{0.000000in}}{\pgfqpoint{-0.000000in}{0.000000in}}{%
\pgfpathmoveto{\pgfqpoint{-0.000000in}{0.000000in}}%
\pgfpathlineto{\pgfqpoint{-0.048611in}{0.000000in}}%
\pgfusepath{stroke,fill}%
}%
\begin{pgfscope}%
\pgfsys@transformshift{0.915000in}{1.834944in}%
\pgfsys@useobject{currentmarker}{}%
\end{pgfscope}%
\end{pgfscope}%
\begin{pgfscope}%
\definecolor{textcolor}{rgb}{0.000000,0.000000,0.000000}%
\pgfsetstrokecolor{textcolor}%
\pgfsetfillcolor{textcolor}%
\pgftext[x=0.331802in, y=1.782182in, left, base]{\color{textcolor}{\sffamily\fontsize{10.000000}{12.000000}\selectfont\catcode`\^=\active\def^{\ifmmode\sp\else\^{}\fi}\catcode`\%=\active\def%{\%}0.0100}}%
\end{pgfscope}%
\begin{pgfscope}%
\pgfpathrectangle{\pgfqpoint{0.915000in}{0.629200in}}{\pgfqpoint{5.673000in}{4.404400in}}%
\pgfusepath{clip}%
\pgfsetrectcap%
\pgfsetroundjoin%
\pgfsetlinewidth{0.803000pt}%
\definecolor{currentstroke}{rgb}{0.690196,0.690196,0.690196}%
\pgfsetstrokecolor{currentstroke}%
\pgfsetdash{}{0pt}%
\pgfpathmoveto{\pgfqpoint{0.915000in}{2.336580in}}%
\pgfpathlineto{\pgfqpoint{6.588000in}{2.336580in}}%
\pgfusepath{stroke}%
\end{pgfscope}%
\begin{pgfscope}%
\pgfsetbuttcap%
\pgfsetroundjoin%
\definecolor{currentfill}{rgb}{0.000000,0.000000,0.000000}%
\pgfsetfillcolor{currentfill}%
\pgfsetlinewidth{0.803000pt}%
\definecolor{currentstroke}{rgb}{0.000000,0.000000,0.000000}%
\pgfsetstrokecolor{currentstroke}%
\pgfsetdash{}{0pt}%
\pgfsys@defobject{currentmarker}{\pgfqpoint{-0.048611in}{0.000000in}}{\pgfqpoint{-0.000000in}{0.000000in}}{%
\pgfpathmoveto{\pgfqpoint{-0.000000in}{0.000000in}}%
\pgfpathlineto{\pgfqpoint{-0.048611in}{0.000000in}}%
\pgfusepath{stroke,fill}%
}%
\begin{pgfscope}%
\pgfsys@transformshift{0.915000in}{2.336580in}%
\pgfsys@useobject{currentmarker}{}%
\end{pgfscope}%
\end{pgfscope}%
\begin{pgfscope}%
\definecolor{textcolor}{rgb}{0.000000,0.000000,0.000000}%
\pgfsetstrokecolor{textcolor}%
\pgfsetfillcolor{textcolor}%
\pgftext[x=0.331802in, y=2.283818in, left, base]{\color{textcolor}{\sffamily\fontsize{10.000000}{12.000000}\selectfont\catcode`\^=\active\def^{\ifmmode\sp\else\^{}\fi}\catcode`\%=\active\def%{\%}0.0125}}%
\end{pgfscope}%
\begin{pgfscope}%
\pgfpathrectangle{\pgfqpoint{0.915000in}{0.629200in}}{\pgfqpoint{5.673000in}{4.404400in}}%
\pgfusepath{clip}%
\pgfsetrectcap%
\pgfsetroundjoin%
\pgfsetlinewidth{0.803000pt}%
\definecolor{currentstroke}{rgb}{0.690196,0.690196,0.690196}%
\pgfsetstrokecolor{currentstroke}%
\pgfsetdash{}{0pt}%
\pgfpathmoveto{\pgfqpoint{0.915000in}{2.838216in}}%
\pgfpathlineto{\pgfqpoint{6.588000in}{2.838216in}}%
\pgfusepath{stroke}%
\end{pgfscope}%
\begin{pgfscope}%
\pgfsetbuttcap%
\pgfsetroundjoin%
\definecolor{currentfill}{rgb}{0.000000,0.000000,0.000000}%
\pgfsetfillcolor{currentfill}%
\pgfsetlinewidth{0.803000pt}%
\definecolor{currentstroke}{rgb}{0.000000,0.000000,0.000000}%
\pgfsetstrokecolor{currentstroke}%
\pgfsetdash{}{0pt}%
\pgfsys@defobject{currentmarker}{\pgfqpoint{-0.048611in}{0.000000in}}{\pgfqpoint{-0.000000in}{0.000000in}}{%
\pgfpathmoveto{\pgfqpoint{-0.000000in}{0.000000in}}%
\pgfpathlineto{\pgfqpoint{-0.048611in}{0.000000in}}%
\pgfusepath{stroke,fill}%
}%
\begin{pgfscope}%
\pgfsys@transformshift{0.915000in}{2.838216in}%
\pgfsys@useobject{currentmarker}{}%
\end{pgfscope}%
\end{pgfscope}%
\begin{pgfscope}%
\definecolor{textcolor}{rgb}{0.000000,0.000000,0.000000}%
\pgfsetstrokecolor{textcolor}%
\pgfsetfillcolor{textcolor}%
\pgftext[x=0.331802in, y=2.785454in, left, base]{\color{textcolor}{\sffamily\fontsize{10.000000}{12.000000}\selectfont\catcode`\^=\active\def^{\ifmmode\sp\else\^{}\fi}\catcode`\%=\active\def%{\%}0.0150}}%
\end{pgfscope}%
\begin{pgfscope}%
\pgfpathrectangle{\pgfqpoint{0.915000in}{0.629200in}}{\pgfqpoint{5.673000in}{4.404400in}}%
\pgfusepath{clip}%
\pgfsetrectcap%
\pgfsetroundjoin%
\pgfsetlinewidth{0.803000pt}%
\definecolor{currentstroke}{rgb}{0.690196,0.690196,0.690196}%
\pgfsetstrokecolor{currentstroke}%
\pgfsetdash{}{0pt}%
\pgfpathmoveto{\pgfqpoint{0.915000in}{3.339852in}}%
\pgfpathlineto{\pgfqpoint{6.588000in}{3.339852in}}%
\pgfusepath{stroke}%
\end{pgfscope}%
\begin{pgfscope}%
\pgfsetbuttcap%
\pgfsetroundjoin%
\definecolor{currentfill}{rgb}{0.000000,0.000000,0.000000}%
\pgfsetfillcolor{currentfill}%
\pgfsetlinewidth{0.803000pt}%
\definecolor{currentstroke}{rgb}{0.000000,0.000000,0.000000}%
\pgfsetstrokecolor{currentstroke}%
\pgfsetdash{}{0pt}%
\pgfsys@defobject{currentmarker}{\pgfqpoint{-0.048611in}{0.000000in}}{\pgfqpoint{-0.000000in}{0.000000in}}{%
\pgfpathmoveto{\pgfqpoint{-0.000000in}{0.000000in}}%
\pgfpathlineto{\pgfqpoint{-0.048611in}{0.000000in}}%
\pgfusepath{stroke,fill}%
}%
\begin{pgfscope}%
\pgfsys@transformshift{0.915000in}{3.339852in}%
\pgfsys@useobject{currentmarker}{}%
\end{pgfscope}%
\end{pgfscope}%
\begin{pgfscope}%
\definecolor{textcolor}{rgb}{0.000000,0.000000,0.000000}%
\pgfsetstrokecolor{textcolor}%
\pgfsetfillcolor{textcolor}%
\pgftext[x=0.331802in, y=3.287090in, left, base]{\color{textcolor}{\sffamily\fontsize{10.000000}{12.000000}\selectfont\catcode`\^=\active\def^{\ifmmode\sp\else\^{}\fi}\catcode`\%=\active\def%{\%}0.0175}}%
\end{pgfscope}%
\begin{pgfscope}%
\pgfpathrectangle{\pgfqpoint{0.915000in}{0.629200in}}{\pgfqpoint{5.673000in}{4.404400in}}%
\pgfusepath{clip}%
\pgfsetrectcap%
\pgfsetroundjoin%
\pgfsetlinewidth{0.803000pt}%
\definecolor{currentstroke}{rgb}{0.690196,0.690196,0.690196}%
\pgfsetstrokecolor{currentstroke}%
\pgfsetdash{}{0pt}%
\pgfpathmoveto{\pgfqpoint{0.915000in}{3.841488in}}%
\pgfpathlineto{\pgfqpoint{6.588000in}{3.841488in}}%
\pgfusepath{stroke}%
\end{pgfscope}%
\begin{pgfscope}%
\pgfsetbuttcap%
\pgfsetroundjoin%
\definecolor{currentfill}{rgb}{0.000000,0.000000,0.000000}%
\pgfsetfillcolor{currentfill}%
\pgfsetlinewidth{0.803000pt}%
\definecolor{currentstroke}{rgb}{0.000000,0.000000,0.000000}%
\pgfsetstrokecolor{currentstroke}%
\pgfsetdash{}{0pt}%
\pgfsys@defobject{currentmarker}{\pgfqpoint{-0.048611in}{0.000000in}}{\pgfqpoint{-0.000000in}{0.000000in}}{%
\pgfpathmoveto{\pgfqpoint{-0.000000in}{0.000000in}}%
\pgfpathlineto{\pgfqpoint{-0.048611in}{0.000000in}}%
\pgfusepath{stroke,fill}%
}%
\begin{pgfscope}%
\pgfsys@transformshift{0.915000in}{3.841488in}%
\pgfsys@useobject{currentmarker}{}%
\end{pgfscope}%
\end{pgfscope}%
\begin{pgfscope}%
\definecolor{textcolor}{rgb}{0.000000,0.000000,0.000000}%
\pgfsetstrokecolor{textcolor}%
\pgfsetfillcolor{textcolor}%
\pgftext[x=0.331802in, y=3.788726in, left, base]{\color{textcolor}{\sffamily\fontsize{10.000000}{12.000000}\selectfont\catcode`\^=\active\def^{\ifmmode\sp\else\^{}\fi}\catcode`\%=\active\def%{\%}0.0200}}%
\end{pgfscope}%
\begin{pgfscope}%
\pgfpathrectangle{\pgfqpoint{0.915000in}{0.629200in}}{\pgfqpoint{5.673000in}{4.404400in}}%
\pgfusepath{clip}%
\pgfsetrectcap%
\pgfsetroundjoin%
\pgfsetlinewidth{0.803000pt}%
\definecolor{currentstroke}{rgb}{0.690196,0.690196,0.690196}%
\pgfsetstrokecolor{currentstroke}%
\pgfsetdash{}{0pt}%
\pgfpathmoveto{\pgfqpoint{0.915000in}{4.343124in}}%
\pgfpathlineto{\pgfqpoint{6.588000in}{4.343124in}}%
\pgfusepath{stroke}%
\end{pgfscope}%
\begin{pgfscope}%
\pgfsetbuttcap%
\pgfsetroundjoin%
\definecolor{currentfill}{rgb}{0.000000,0.000000,0.000000}%
\pgfsetfillcolor{currentfill}%
\pgfsetlinewidth{0.803000pt}%
\definecolor{currentstroke}{rgb}{0.000000,0.000000,0.000000}%
\pgfsetstrokecolor{currentstroke}%
\pgfsetdash{}{0pt}%
\pgfsys@defobject{currentmarker}{\pgfqpoint{-0.048611in}{0.000000in}}{\pgfqpoint{-0.000000in}{0.000000in}}{%
\pgfpathmoveto{\pgfqpoint{-0.000000in}{0.000000in}}%
\pgfpathlineto{\pgfqpoint{-0.048611in}{0.000000in}}%
\pgfusepath{stroke,fill}%
}%
\begin{pgfscope}%
\pgfsys@transformshift{0.915000in}{4.343124in}%
\pgfsys@useobject{currentmarker}{}%
\end{pgfscope}%
\end{pgfscope}%
\begin{pgfscope}%
\definecolor{textcolor}{rgb}{0.000000,0.000000,0.000000}%
\pgfsetstrokecolor{textcolor}%
\pgfsetfillcolor{textcolor}%
\pgftext[x=0.331802in, y=4.290362in, left, base]{\color{textcolor}{\sffamily\fontsize{10.000000}{12.000000}\selectfont\catcode`\^=\active\def^{\ifmmode\sp\else\^{}\fi}\catcode`\%=\active\def%{\%}0.0225}}%
\end{pgfscope}%
\begin{pgfscope}%
\pgfpathrectangle{\pgfqpoint{0.915000in}{0.629200in}}{\pgfqpoint{5.673000in}{4.404400in}}%
\pgfusepath{clip}%
\pgfsetrectcap%
\pgfsetroundjoin%
\pgfsetlinewidth{0.803000pt}%
\definecolor{currentstroke}{rgb}{0.690196,0.690196,0.690196}%
\pgfsetstrokecolor{currentstroke}%
\pgfsetdash{}{0pt}%
\pgfpathmoveto{\pgfqpoint{0.915000in}{4.844759in}}%
\pgfpathlineto{\pgfqpoint{6.588000in}{4.844759in}}%
\pgfusepath{stroke}%
\end{pgfscope}%
\begin{pgfscope}%
\pgfsetbuttcap%
\pgfsetroundjoin%
\definecolor{currentfill}{rgb}{0.000000,0.000000,0.000000}%
\pgfsetfillcolor{currentfill}%
\pgfsetlinewidth{0.803000pt}%
\definecolor{currentstroke}{rgb}{0.000000,0.000000,0.000000}%
\pgfsetstrokecolor{currentstroke}%
\pgfsetdash{}{0pt}%
\pgfsys@defobject{currentmarker}{\pgfqpoint{-0.048611in}{0.000000in}}{\pgfqpoint{-0.000000in}{0.000000in}}{%
\pgfpathmoveto{\pgfqpoint{-0.000000in}{0.000000in}}%
\pgfpathlineto{\pgfqpoint{-0.048611in}{0.000000in}}%
\pgfusepath{stroke,fill}%
}%
\begin{pgfscope}%
\pgfsys@transformshift{0.915000in}{4.844759in}%
\pgfsys@useobject{currentmarker}{}%
\end{pgfscope}%
\end{pgfscope}%
\begin{pgfscope}%
\definecolor{textcolor}{rgb}{0.000000,0.000000,0.000000}%
\pgfsetstrokecolor{textcolor}%
\pgfsetfillcolor{textcolor}%
\pgftext[x=0.331802in, y=4.791998in, left, base]{\color{textcolor}{\sffamily\fontsize{10.000000}{12.000000}\selectfont\catcode`\^=\active\def^{\ifmmode\sp\else\^{}\fi}\catcode`\%=\active\def%{\%}0.0250}}%
\end{pgfscope}%
\begin{pgfscope}%
\definecolor{textcolor}{rgb}{0.000000,0.000000,0.000000}%
\pgfsetstrokecolor{textcolor}%
\pgfsetfillcolor{textcolor}%
\pgftext[x=0.276247in,y=2.831400in,,bottom,rotate=90.000000]{\color{textcolor}{\sffamily\fontsize{14.000000}{16.800000}\selectfont\catcode`\^=\active\def^{\ifmmode\sp\else\^{}\fi}\catcode`\%=\active\def%{\%}Volume/$\mathrm{m}^3\mathrm{mol}^{-1}$}}%
\end{pgfscope}%
\begin{pgfscope}%
\pgfpathrectangle{\pgfqpoint{0.915000in}{0.629200in}}{\pgfqpoint{5.673000in}{4.404400in}}%
\pgfusepath{clip}%
\pgfsetrectcap%
\pgfsetroundjoin%
\pgfsetlinewidth{1.505625pt}%
\definecolor{currentstroke}{rgb}{0.121569,0.466667,0.705882}%
\pgfsetstrokecolor{currentstroke}%
\pgfsetdash{}{0pt}%
\pgfpathmoveto{\pgfqpoint{1.172864in}{4.833400in}}%
\pgfpathlineto{\pgfqpoint{1.197326in}{4.639031in}}%
\pgfpathlineto{\pgfqpoint{1.221788in}{4.459194in}}%
\pgfpathlineto{\pgfqpoint{1.246250in}{4.292319in}}%
\pgfpathlineto{\pgfqpoint{1.270712in}{4.137052in}}%
\pgfpathlineto{\pgfqpoint{1.295174in}{3.992224in}}%
\pgfpathlineto{\pgfqpoint{1.319636in}{3.856815in}}%
\pgfpathlineto{\pgfqpoint{1.344098in}{3.729935in}}%
\pgfpathlineto{\pgfqpoint{1.368560in}{3.610805in}}%
\pgfpathlineto{\pgfqpoint{1.393022in}{3.498733in}}%
\pgfpathlineto{\pgfqpoint{1.417485in}{3.393112in}}%
\pgfpathlineto{\pgfqpoint{1.441947in}{3.293400in}}%
\pgfpathlineto{\pgfqpoint{1.466409in}{3.199114in}}%
\pgfpathlineto{\pgfqpoint{1.490871in}{3.109824in}}%
\pgfpathlineto{\pgfqpoint{1.515333in}{3.025142in}}%
\pgfpathlineto{\pgfqpoint{1.539795in}{2.944721in}}%
\pgfpathlineto{\pgfqpoint{1.564257in}{2.868247in}}%
\pgfpathlineto{\pgfqpoint{1.588719in}{2.795436in}}%
\pgfpathlineto{\pgfqpoint{1.613181in}{2.726032in}}%
\pgfpathlineto{\pgfqpoint{1.637643in}{2.659800in}}%
\pgfpathlineto{\pgfqpoint{1.662105in}{2.596528in}}%
\pgfpathlineto{\pgfqpoint{1.686567in}{2.536023in}}%
\pgfpathlineto{\pgfqpoint{1.711030in}{2.478106in}}%
\pgfpathlineto{\pgfqpoint{1.735492in}{2.422615in}}%
\pgfpathlineto{\pgfqpoint{1.759954in}{2.369400in}}%
\pgfpathlineto{\pgfqpoint{1.784416in}{2.318325in}}%
\pgfpathlineto{\pgfqpoint{1.808878in}{2.269262in}}%
\pgfpathlineto{\pgfqpoint{1.833340in}{2.222096in}}%
\pgfpathlineto{\pgfqpoint{1.857802in}{2.176718in}}%
\pgfpathlineto{\pgfqpoint{1.882264in}{2.133028in}}%
\pgfpathlineto{\pgfqpoint{1.906726in}{2.090934in}}%
\pgfpathlineto{\pgfqpoint{1.931188in}{2.050351in}}%
\pgfpathlineto{\pgfqpoint{1.955650in}{2.011197in}}%
\pgfpathlineto{\pgfqpoint{1.980113in}{1.973400in}}%
\pgfpathlineto{\pgfqpoint{2.004575in}{1.936889in}}%
\pgfpathlineto{\pgfqpoint{2.029037in}{1.901601in}}%
\pgfpathlineto{\pgfqpoint{2.053499in}{1.867474in}}%
\pgfpathlineto{\pgfqpoint{2.077961in}{1.834453in}}%
\pgfpathlineto{\pgfqpoint{2.102423in}{1.802484in}}%
\pgfpathlineto{\pgfqpoint{2.126885in}{1.771518in}}%
\pgfpathlineto{\pgfqpoint{2.151347in}{1.741508in}}%
\pgfpathlineto{\pgfqpoint{2.175809in}{1.712411in}}%
\pgfpathlineto{\pgfqpoint{2.200271in}{1.684187in}}%
\pgfpathlineto{\pgfqpoint{2.224733in}{1.656795in}}%
\pgfpathlineto{\pgfqpoint{2.249195in}{1.630200in}}%
\pgfpathlineto{\pgfqpoint{2.273658in}{1.604368in}}%
\pgfpathlineto{\pgfqpoint{2.298120in}{1.579266in}}%
\pgfpathlineto{\pgfqpoint{2.322582in}{1.554863in}}%
\pgfpathlineto{\pgfqpoint{2.347044in}{1.531132in}}%
\pgfpathlineto{\pgfqpoint{2.371506in}{1.508044in}}%
\pgfpathlineto{\pgfqpoint{2.395968in}{1.485574in}}%
\pgfpathlineto{\pgfqpoint{2.420430in}{1.463697in}}%
\pgfpathlineto{\pgfqpoint{2.444892in}{1.442390in}}%
\pgfpathlineto{\pgfqpoint{2.469354in}{1.421632in}}%
\pgfpathlineto{\pgfqpoint{2.493816in}{1.401400in}}%
\pgfpathlineto{\pgfqpoint{2.518278in}{1.381676in}}%
\pgfpathlineto{\pgfqpoint{2.542741in}{1.362440in}}%
\pgfpathlineto{\pgfqpoint{2.567203in}{1.343675in}}%
\pgfpathlineto{\pgfqpoint{2.591665in}{1.325364in}}%
\pgfpathlineto{\pgfqpoint{2.616127in}{1.307490in}}%
\pgfpathlineto{\pgfqpoint{2.640589in}{1.290037in}}%
\pgfpathlineto{\pgfqpoint{2.665051in}{1.272992in}}%
\pgfpathlineto{\pgfqpoint{2.689513in}{1.256339in}}%
\pgfpathlineto{\pgfqpoint{2.713975in}{1.240067in}}%
\pgfpathlineto{\pgfqpoint{2.738437in}{1.224161in}}%
\pgfpathlineto{\pgfqpoint{2.762899in}{1.208609in}}%
\pgfpathlineto{\pgfqpoint{2.787361in}{1.193400in}}%
\pgfpathlineto{\pgfqpoint{2.811823in}{1.178523in}}%
\pgfpathlineto{\pgfqpoint{2.836286in}{1.163966in}}%
\pgfpathlineto{\pgfqpoint{2.860748in}{1.149720in}}%
\pgfpathlineto{\pgfqpoint{2.885210in}{1.135775in}}%
\pgfpathlineto{\pgfqpoint{2.909672in}{1.122121in}}%
\pgfpathlineto{\pgfqpoint{2.934134in}{1.108749in}}%
\pgfpathlineto{\pgfqpoint{2.958596in}{1.095651in}}%
\pgfpathlineto{\pgfqpoint{2.983058in}{1.082818in}}%
\pgfpathlineto{\pgfqpoint{3.007520in}{1.070242in}}%
\pgfpathlineto{\pgfqpoint{3.031982in}{1.057916in}}%
\pgfpathlineto{\pgfqpoint{3.056444in}{1.045832in}}%
\pgfpathlineto{\pgfqpoint{3.080906in}{1.033984in}}%
\pgfpathlineto{\pgfqpoint{3.105369in}{1.022364in}}%
\pgfpathlineto{\pgfqpoint{3.129831in}{1.010966in}}%
\pgfpathlineto{\pgfqpoint{3.154293in}{0.999783in}}%
\pgfpathlineto{\pgfqpoint{3.178755in}{0.988810in}}%
\pgfpathlineto{\pgfqpoint{3.203217in}{0.978040in}}%
\pgfpathlineto{\pgfqpoint{3.227679in}{0.967469in}}%
\pgfpathlineto{\pgfqpoint{3.252141in}{0.957090in}}%
\pgfpathlineto{\pgfqpoint{3.276603in}{0.946899in}}%
\pgfpathlineto{\pgfqpoint{3.301065in}{0.936890in}}%
\pgfpathlineto{\pgfqpoint{3.325527in}{0.927059in}}%
\pgfpathlineto{\pgfqpoint{3.349989in}{0.917400in}}%
\pgfpathlineto{\pgfqpoint{3.374451in}{0.907910in}}%
\pgfpathlineto{\pgfqpoint{3.398914in}{0.898584in}}%
\pgfpathlineto{\pgfqpoint{3.423376in}{0.889417in}}%
\pgfpathlineto{\pgfqpoint{3.447838in}{0.880406in}}%
\pgfpathlineto{\pgfqpoint{3.472300in}{0.871547in}}%
\pgfpathlineto{\pgfqpoint{3.496762in}{0.862836in}}%
\pgfpathlineto{\pgfqpoint{3.521224in}{0.854270in}}%
\pgfpathlineto{\pgfqpoint{3.545686in}{0.845844in}}%
\pgfpathlineto{\pgfqpoint{3.570148in}{0.837555in}}%
\pgfpathlineto{\pgfqpoint{3.594610in}{0.829400in}}%
\pgfusepath{stroke}%
\end{pgfscope}%
\begin{pgfscope}%
\pgfpathrectangle{\pgfqpoint{0.915000in}{0.629200in}}{\pgfqpoint{5.673000in}{4.404400in}}%
\pgfusepath{clip}%
\pgfsetrectcap%
\pgfsetroundjoin%
\pgfsetlinewidth{1.505625pt}%
\definecolor{currentstroke}{rgb}{1.000000,0.498039,0.054902}%
\pgfsetstrokecolor{currentstroke}%
\pgfsetdash{}{0pt}%
\pgfpathmoveto{\pgfqpoint{3.594610in}{0.829400in}}%
\pgfpathlineto{\pgfqpoint{6.330136in}{0.829400in}}%
\pgfusepath{stroke}%
\end{pgfscope}%
\begin{pgfscope}%
\pgfpathrectangle{\pgfqpoint{0.915000in}{0.629200in}}{\pgfqpoint{5.673000in}{4.404400in}}%
\pgfusepath{clip}%
\pgfsetrectcap%
\pgfsetroundjoin%
\pgfsetlinewidth{1.505625pt}%
\definecolor{currentstroke}{rgb}{0.172549,0.627451,0.172549}%
\pgfsetstrokecolor{currentstroke}%
\pgfsetdash{}{0pt}%
\pgfpathmoveto{\pgfqpoint{6.330136in}{0.829400in}}%
\pgfpathlineto{\pgfqpoint{6.278043in}{0.835914in}}%
\pgfpathlineto{\pgfqpoint{6.225949in}{0.842531in}}%
\pgfpathlineto{\pgfqpoint{6.173855in}{0.849252in}}%
\pgfpathlineto{\pgfqpoint{6.121762in}{0.856082in}}%
\pgfpathlineto{\pgfqpoint{6.069668in}{0.863023in}}%
\pgfpathlineto{\pgfqpoint{6.017574in}{0.870077in}}%
\pgfpathlineto{\pgfqpoint{5.965481in}{0.877247in}}%
\pgfpathlineto{\pgfqpoint{5.913387in}{0.884538in}}%
\pgfpathlineto{\pgfqpoint{5.861293in}{0.891951in}}%
\pgfpathlineto{\pgfqpoint{5.809200in}{0.899490in}}%
\pgfpathlineto{\pgfqpoint{5.757106in}{0.907159in}}%
\pgfpathlineto{\pgfqpoint{5.705012in}{0.914960in}}%
\pgfpathlineto{\pgfqpoint{5.652919in}{0.922899in}}%
\pgfpathlineto{\pgfqpoint{5.600825in}{0.930978in}}%
\pgfpathlineto{\pgfqpoint{5.548731in}{0.939202in}}%
\pgfpathlineto{\pgfqpoint{5.496638in}{0.947575in}}%
\pgfpathlineto{\pgfqpoint{5.444544in}{0.956100in}}%
\pgfpathlineto{\pgfqpoint{5.392450in}{0.964784in}}%
\pgfpathlineto{\pgfqpoint{5.340357in}{0.973629in}}%
\pgfpathlineto{\pgfqpoint{5.288263in}{0.982642in}}%
\pgfpathlineto{\pgfqpoint{5.236169in}{0.991826in}}%
\pgfpathlineto{\pgfqpoint{5.184076in}{1.001189in}}%
\pgfpathlineto{\pgfqpoint{5.131982in}{1.010733in}}%
\pgfpathlineto{\pgfqpoint{5.079888in}{1.020467in}}%
\pgfpathlineto{\pgfqpoint{5.027795in}{1.030395in}}%
\pgfpathlineto{\pgfqpoint{4.975701in}{1.040524in}}%
\pgfpathlineto{\pgfqpoint{4.923607in}{1.050860in}}%
\pgfpathlineto{\pgfqpoint{4.871514in}{1.061410in}}%
\pgfpathlineto{\pgfqpoint{4.819420in}{1.072182in}}%
\pgfpathlineto{\pgfqpoint{4.767326in}{1.083182in}}%
\pgfpathlineto{\pgfqpoint{4.715233in}{1.094418in}}%
\pgfpathlineto{\pgfqpoint{4.663139in}{1.105899in}}%
\pgfpathlineto{\pgfqpoint{4.611045in}{1.117634in}}%
\pgfpathlineto{\pgfqpoint{4.558952in}{1.129630in}}%
\pgfpathlineto{\pgfqpoint{4.506858in}{1.141897in}}%
\pgfpathlineto{\pgfqpoint{4.454764in}{1.154446in}}%
\pgfpathlineto{\pgfqpoint{4.402671in}{1.167287in}}%
\pgfpathlineto{\pgfqpoint{4.350577in}{1.180430in}}%
\pgfpathlineto{\pgfqpoint{4.298483in}{1.193887in}}%
\pgfpathlineto{\pgfqpoint{4.246390in}{1.207670in}}%
\pgfpathlineto{\pgfqpoint{4.194296in}{1.221792in}}%
\pgfpathlineto{\pgfqpoint{4.142202in}{1.236266in}}%
\pgfpathlineto{\pgfqpoint{4.090109in}{1.251105in}}%
\pgfpathlineto{\pgfqpoint{4.038015in}{1.266326in}}%
\pgfpathlineto{\pgfqpoint{3.985921in}{1.281944in}}%
\pgfpathlineto{\pgfqpoint{3.933828in}{1.297975in}}%
\pgfpathlineto{\pgfqpoint{3.881734in}{1.314437in}}%
\pgfpathlineto{\pgfqpoint{3.829640in}{1.331349in}}%
\pgfpathlineto{\pgfqpoint{3.777547in}{1.348730in}}%
\pgfpathlineto{\pgfqpoint{3.725453in}{1.366602in}}%
\pgfpathlineto{\pgfqpoint{3.673360in}{1.384986in}}%
\pgfpathlineto{\pgfqpoint{3.621266in}{1.403907in}}%
\pgfpathlineto{\pgfqpoint{3.569172in}{1.423389in}}%
\pgfpathlineto{\pgfqpoint{3.517079in}{1.443459in}}%
\pgfpathlineto{\pgfqpoint{3.464985in}{1.464146in}}%
\pgfpathlineto{\pgfqpoint{3.412891in}{1.485481in}}%
\pgfpathlineto{\pgfqpoint{3.360798in}{1.507496in}}%
\pgfpathlineto{\pgfqpoint{3.308704in}{1.530227in}}%
\pgfpathlineto{\pgfqpoint{3.256610in}{1.553710in}}%
\pgfpathlineto{\pgfqpoint{3.204517in}{1.577986in}}%
\pgfpathlineto{\pgfqpoint{3.152423in}{1.603099in}}%
\pgfpathlineto{\pgfqpoint{3.100329in}{1.629094in}}%
\pgfpathlineto{\pgfqpoint{3.048236in}{1.656023in}}%
\pgfpathlineto{\pgfqpoint{2.996142in}{1.683938in}}%
\pgfpathlineto{\pgfqpoint{2.944048in}{1.712900in}}%
\pgfpathlineto{\pgfqpoint{2.891955in}{1.742970in}}%
\pgfpathlineto{\pgfqpoint{2.839861in}{1.774218in}}%
\pgfpathlineto{\pgfqpoint{2.787767in}{1.806719in}}%
\pgfpathlineto{\pgfqpoint{2.735674in}{1.840554in}}%
\pgfpathlineto{\pgfqpoint{2.683580in}{1.875812in}}%
\pgfpathlineto{\pgfqpoint{2.631486in}{1.912590in}}%
\pgfpathlineto{\pgfqpoint{2.579393in}{1.950995in}}%
\pgfpathlineto{\pgfqpoint{2.527299in}{1.991142in}}%
\pgfpathlineto{\pgfqpoint{2.475205in}{2.033162in}}%
\pgfpathlineto{\pgfqpoint{2.423112in}{2.077197in}}%
\pgfpathlineto{\pgfqpoint{2.371018in}{2.123402in}}%
\pgfpathlineto{\pgfqpoint{2.318924in}{2.171954in}}%
\pgfpathlineto{\pgfqpoint{2.266831in}{2.223046in}}%
\pgfpathlineto{\pgfqpoint{2.214737in}{2.276896in}}%
\pgfpathlineto{\pgfqpoint{2.162643in}{2.333746in}}%
\pgfpathlineto{\pgfqpoint{2.110550in}{2.393871in}}%
\pgfpathlineto{\pgfqpoint{2.058456in}{2.457580in}}%
\pgfpathlineto{\pgfqpoint{2.006362in}{2.525222in}}%
\pgfpathlineto{\pgfqpoint{1.954269in}{2.597197in}}%
\pgfpathlineto{\pgfqpoint{1.902175in}{2.673961in}}%
\pgfpathlineto{\pgfqpoint{1.850081in}{2.756040in}}%
\pgfpathlineto{\pgfqpoint{1.797988in}{2.844041in}}%
\pgfpathlineto{\pgfqpoint{1.745894in}{2.938668in}}%
\pgfpathlineto{\pgfqpoint{1.693800in}{3.040750in}}%
\pgfpathlineto{\pgfqpoint{1.641707in}{3.151261in}}%
\pgfpathlineto{\pgfqpoint{1.589613in}{3.271359in}}%
\pgfpathlineto{\pgfqpoint{1.537519in}{3.402429in}}%
\pgfpathlineto{\pgfqpoint{1.485426in}{3.546151in}}%
\pgfpathlineto{\pgfqpoint{1.433332in}{3.704572in}}%
\pgfpathlineto{\pgfqpoint{1.381238in}{3.880223in}}%
\pgfpathlineto{\pgfqpoint{1.329145in}{4.076275in}}%
\pgfpathlineto{\pgfqpoint{1.277051in}{4.296751in}}%
\pgfpathlineto{\pgfqpoint{1.224957in}{4.546846in}}%
\pgfpathlineto{\pgfqpoint{1.172864in}{4.833400in}}%
\pgfusepath{stroke}%
\end{pgfscope}%
\begin{pgfscope}%
\pgfsetrectcap%
\pgfsetmiterjoin%
\pgfsetlinewidth{0.803000pt}%
\definecolor{currentstroke}{rgb}{0.000000,0.000000,0.000000}%
\pgfsetstrokecolor{currentstroke}%
\pgfsetdash{}{0pt}%
\pgfpathmoveto{\pgfqpoint{0.915000in}{0.629200in}}%
\pgfpathlineto{\pgfqpoint{0.915000in}{5.033600in}}%
\pgfusepath{stroke}%
\end{pgfscope}%
\begin{pgfscope}%
\pgfsetrectcap%
\pgfsetmiterjoin%
\pgfsetlinewidth{0.803000pt}%
\definecolor{currentstroke}{rgb}{0.000000,0.000000,0.000000}%
\pgfsetstrokecolor{currentstroke}%
\pgfsetdash{}{0pt}%
\pgfpathmoveto{\pgfqpoint{6.588000in}{0.629200in}}%
\pgfpathlineto{\pgfqpoint{6.588000in}{5.033600in}}%
\pgfusepath{stroke}%
\end{pgfscope}%
\begin{pgfscope}%
\pgfsetrectcap%
\pgfsetmiterjoin%
\pgfsetlinewidth{0.803000pt}%
\definecolor{currentstroke}{rgb}{0.000000,0.000000,0.000000}%
\pgfsetstrokecolor{currentstroke}%
\pgfsetdash{}{0pt}%
\pgfpathmoveto{\pgfqpoint{0.915000in}{0.629200in}}%
\pgfpathlineto{\pgfqpoint{6.588000in}{0.629200in}}%
\pgfusepath{stroke}%
\end{pgfscope}%
\begin{pgfscope}%
\pgfsetrectcap%
\pgfsetmiterjoin%
\pgfsetlinewidth{0.803000pt}%
\definecolor{currentstroke}{rgb}{0.000000,0.000000,0.000000}%
\pgfsetstrokecolor{currentstroke}%
\pgfsetdash{}{0pt}%
\pgfpathmoveto{\pgfqpoint{0.915000in}{5.033600in}}%
\pgfpathlineto{\pgfqpoint{6.588000in}{5.033600in}}%
\pgfusepath{stroke}%
\end{pgfscope}%
\begin{pgfscope}%
\definecolor{textcolor}{rgb}{0.000000,0.000000,0.000000}%
\pgfsetstrokecolor{textcolor}%
\pgfsetfillcolor{textcolor}%
\pgftext[x=1.172864in,y=4.853465in,left,base]{\color{textcolor}{\sffamily\fontsize{10.000000}{12.000000}\selectfont\catcode`\^=\active\def^{\ifmmode\sp\else\^{}\fi}\catcode`\%=\active\def%{\%}State 1}}%
\end{pgfscope}%
\begin{pgfscope}%
\definecolor{textcolor}{rgb}{0.000000,0.000000,0.000000}%
\pgfsetstrokecolor{textcolor}%
\pgfsetfillcolor{textcolor}%
\pgftext[x=3.594610in,y=0.849465in,left,base]{\color{textcolor}{\sffamily\fontsize{10.000000}{12.000000}\selectfont\catcode`\^=\active\def^{\ifmmode\sp\else\^{}\fi}\catcode`\%=\active\def%{\%}State 2}}%
\end{pgfscope}%
\begin{pgfscope}%
\definecolor{textcolor}{rgb}{0.000000,0.000000,0.000000}%
\pgfsetstrokecolor{textcolor}%
\pgfsetfillcolor{textcolor}%
\pgftext[x=6.330136in,y=0.849465in,left,base]{\color{textcolor}{\sffamily\fontsize{10.000000}{12.000000}\selectfont\catcode`\^=\active\def^{\ifmmode\sp\else\^{}\fi}\catcode`\%=\active\def%{\%}State 3}}%
\end{pgfscope}%
\begin{pgfscope}%
\pgfsetbuttcap%
\pgfsetmiterjoin%
\definecolor{currentfill}{rgb}{1.000000,1.000000,1.000000}%
\pgfsetfillcolor{currentfill}%
\pgfsetfillopacity{0.800000}%
\pgfsetlinewidth{1.003750pt}%
\definecolor{currentstroke}{rgb}{0.800000,0.800000,0.800000}%
\pgfsetstrokecolor{currentstroke}%
\pgfsetstrokeopacity{0.800000}%
\pgfsetdash{}{0pt}%
\pgfpathmoveto{\pgfqpoint{5.283801in}{4.310917in}}%
\pgfpathlineto{\pgfqpoint{6.490778in}{4.310917in}}%
\pgfpathquadraticcurveto{\pgfqpoint{6.518556in}{4.310917in}}{\pgfqpoint{6.518556in}{4.338695in}}%
\pgfpathlineto{\pgfqpoint{6.518556in}{4.936378in}}%
\pgfpathquadraticcurveto{\pgfqpoint{6.518556in}{4.964156in}}{\pgfqpoint{6.490778in}{4.964156in}}%
\pgfpathlineto{\pgfqpoint{5.283801in}{4.964156in}}%
\pgfpathquadraticcurveto{\pgfqpoint{5.256023in}{4.964156in}}{\pgfqpoint{5.256023in}{4.936378in}}%
\pgfpathlineto{\pgfqpoint{5.256023in}{4.338695in}}%
\pgfpathquadraticcurveto{\pgfqpoint{5.256023in}{4.310917in}}{\pgfqpoint{5.283801in}{4.310917in}}%
\pgfpathlineto{\pgfqpoint{5.283801in}{4.310917in}}%
\pgfpathclose%
\pgfusepath{stroke,fill}%
\end{pgfscope}%
\begin{pgfscope}%
\pgfsetrectcap%
\pgfsetroundjoin%
\pgfsetlinewidth{1.505625pt}%
\definecolor{currentstroke}{rgb}{0.121569,0.466667,0.705882}%
\pgfsetstrokecolor{currentstroke}%
\pgfsetdash{}{0pt}%
\pgfpathmoveto{\pgfqpoint{5.311579in}{4.851688in}}%
\pgfpathlineto{\pgfqpoint{5.450468in}{4.851688in}}%
\pgfpathlineto{\pgfqpoint{5.589356in}{4.851688in}}%
\pgfusepath{stroke}%
\end{pgfscope}%
\begin{pgfscope}%
\definecolor{textcolor}{rgb}{0.000000,0.000000,0.000000}%
\pgfsetstrokecolor{textcolor}%
\pgfsetfillcolor{textcolor}%
\pgftext[x=5.700468in,y=4.803077in,left,base]{\color{textcolor}{\rmfamily\fontsize{10.000000}{12.000000}\selectfont\catcode`\^=\active\def^{\ifmmode\sp\else\^{}\fi}\catcode`\%=\active\def%{\%}Isothermal}}%
\end{pgfscope}%
\begin{pgfscope}%
\pgfsetrectcap%
\pgfsetroundjoin%
\pgfsetlinewidth{1.505625pt}%
\definecolor{currentstroke}{rgb}{1.000000,0.498039,0.054902}%
\pgfsetstrokecolor{currentstroke}%
\pgfsetdash{}{0pt}%
\pgfpathmoveto{\pgfqpoint{5.311579in}{4.647831in}}%
\pgfpathlineto{\pgfqpoint{5.450468in}{4.647831in}}%
\pgfpathlineto{\pgfqpoint{5.589356in}{4.647831in}}%
\pgfusepath{stroke}%
\end{pgfscope}%
\begin{pgfscope}%
\definecolor{textcolor}{rgb}{0.000000,0.000000,0.000000}%
\pgfsetstrokecolor{textcolor}%
\pgfsetfillcolor{textcolor}%
\pgftext[x=5.700468in,y=4.599220in,left,base]{\color{textcolor}{\rmfamily\fontsize{10.000000}{12.000000}\selectfont\catcode`\^=\active\def^{\ifmmode\sp\else\^{}\fi}\catcode`\%=\active\def%{\%}Isochoric}}%
\end{pgfscope}%
\begin{pgfscope}%
\pgfsetrectcap%
\pgfsetroundjoin%
\pgfsetlinewidth{1.505625pt}%
\definecolor{currentstroke}{rgb}{0.172549,0.627451,0.172549}%
\pgfsetstrokecolor{currentstroke}%
\pgfsetdash{}{0pt}%
\pgfpathmoveto{\pgfqpoint{5.311579in}{4.443974in}}%
\pgfpathlineto{\pgfqpoint{5.450468in}{4.443974in}}%
\pgfpathlineto{\pgfqpoint{5.589356in}{4.443974in}}%
\pgfusepath{stroke}%
\end{pgfscope}%
\begin{pgfscope}%
\definecolor{textcolor}{rgb}{0.000000,0.000000,0.000000}%
\pgfsetstrokecolor{textcolor}%
\pgfsetfillcolor{textcolor}%
\pgftext[x=5.700468in,y=4.395362in,left,base]{\color{textcolor}{\rmfamily\fontsize{10.000000}{12.000000}\selectfont\catcode`\^=\active\def^{\ifmmode\sp\else\^{}\fi}\catcode`\%=\active\def%{\%}Adiabatic}}%
\end{pgfscope}%
\end{pgfpicture}%
\makeatother%
\endgroup%
}
        \caption{
          PV graph of the three-step cycle as described in
          Problem 3.20.
        }
        \label{fig:s20}
      \end{figure}
    \item ~
      \begin{enumerate}[label=\roman*.]
        \item State 1:
          Use the ideal gas equation to determine $V$ which is
          0.0249~\unit{ m\cubed\per\mole }
        \item State 2:
          The path from state 1 to state 2 is isothermal therefore
          $T_{2}=300~\unit{ \kelvin }$. Determining $V_{2}$ from the
          ideal gas equation yields 0.05~\unit{ m\cubed\per\mole }
        \item State 3:
          The path from state 2 to state 3 is isochoric therefore
          $V_{3}=0.05~\unit{ m\cubed\per\mole }$. The path from state
          3 to state 1 is adiabatic so we can use one of the
          adiabatic equations; I will use this
          \begin{equation*}
            PV^{\gamma }=\text{constant}
          \end{equation*}
          to get the value of $P_{3}$ which is about 0.38~\unit{ \bar
          }. $T_{3}$ can be solved by using the ideal gas law which
          is about 226~\unit{ \kelvin }
      \end{enumerate}
    \item
      \begin{enumerate}[label=\roman*.]
        \item Path 1
          \begin{gather*}
            Q=-W\\
            Q=RT\ln\frac{P_{1}}{P_{2}}
          \end{gather*}
          \begin{empheq}[box=\widefbox]{gather*}
            \Delta H=\Delta U=0\\
            Q=-4.01~\unit{ \kilo\joule\per\mole }\\
            W=4.01~\unit{ \kilo\joule\per\mole }
          \end{empheq}
        \item Path 2
          \begin{gather*}
            Q=\Delta U\\
            Q=C_{V}\Delta T\\
            \Delta H=C_{P}\Delta T
          \end{gather*}
          \begin{empheq}[box=\widefbox]{gather*}
            W=0\\
            Q=\Delta U=-1.54~\unit{ \kilo\joule\per\mole }\\
            \Delta H=-2.15~\unit{ \kilo\joule\per\mole }
          \end{empheq}
        \item Path 3
          \begin{gather*}
            W=\Delta U\\
            W=C_{V}\Delta T\\
            \Delta H=C_{P}\Delta T
          \end{gather*}
          \begin{empheq}[box=\widefbox]{gather*}
            Q=0\\
            W=\Delta U=1.54~\unit{ \kilo\joule\per\mole }\\
            \Delta H=2.15~\unit{ \kilo\joule\per\mole }
          \end{empheq}
      \end{enumerate}
  \end{enumerate}
\end{solution}

\section*{Problem 3.21}
\addcontentsline{toc}{section}{Problem 3.21}
The state of an ideal gas with $C_P = \frac{5}{2}R$ is changed from
$P_1 = \SI{1}{\bar}$ and $V_1 = \SI{12}{\meter\cubed}$ to $P_2 =
\SI{12}{\bar}$ and $V_2 = \SI{1}{\meter\cubed}$ by the following
mechanically reversible processes:
\begin{enumerate}[label=(\alph*)]
  \item Isothermal compression.
  \item Adiabatic compression followed by cooling at constant pressure.
  \item Adiabatic compression followed by cooling at constant volume.
  \item Heating at constant volume followed by cooling at constant pressure.
  \item Cooling at constant pressure followed by heating at constant volume.
\end{enumerate}
Calculate $Q$, $W$, $\Delta U$, and $\Delta H$ for all processes, and
sketch the paths of all processes on a single $PV$ diagram.

\begin{solution}
  \begin{enumerate}[label=(\alph*)]
    \item
      \begin{gather*}
        T=\frac{PV}{R}\\
        T=144334.86~\unit{ \kelvin }\\
        Q=-W\\
        Q=RT\ln\frac{P_{1}}{P_{2}}\\
      \end{gather*}
      \begin{empheq}[box=\widefbox]{gather*}
        \Delta H=\Delta U=0\\
        Q=-2982.06~\unit{ \kilo\joule\per\mole }\\
        W=2982.06~\unit{ \kilo\joule\per\mole }
      \end{empheq}
      $\Delta U$ and $\Delta H$ for all processes is 0.
    \item
      \begin{gather*}
        V_{1.5}=\left( \frac{P_{1}V_{1}^{\gamma }}{P_{1.5}}
        \right)^{1/\gamma }\\
        V_{1.5}=2.70~\unit{ m\cubed\per\mole }\\
        \intertext{After some derivation for the work:}
        W = W_{1\to1.5}+W_{1.5\to2}\\
        W_{1\to1.5}=-Rc\int_{V_{1}}^{V_{1.5}} V^{-\gamma } \, dV \qquad
        \text{where} \qquad c = P_{1}V_{1}^{\gamma}\\
        W_{1.5\to2}=-P_{2}\left( V_{2}-V_{1.5} \right)\\
        Q=-W
      \end{gather*}
      \begin{empheq}[box=\widefbox]{gather*}
        W=2040.25~\unit{ \kilo\joule\per\mole }\\
        Q=-2040.25~\unit{ \kilo\joule\per\mole }
      \end{empheq}
    \item
      \begin{gather*}
        P_{1.5}=\frac{P_{1}V_{1}^{\gamma }}{V_{1.5}^{\gamma }}\\
        P_{1.5}=62.898~\unit{ \bar }\\
        T_{1.5}=\frac{P_{1.5}V_{1.5}}{R}\\
        T_{1.5}=756528.668~\unit{ \kelvin }\\
        T_{2}= \frac{P_{2}V_{2}}{R}\\
        T_{2}=144334.8569~\unit{ \kelvin }\\
        W=W_{1\to1.5} = -c\int_{V_{1}}^{V_{1.5}} V^{-\gamma }
        \, dV \qquad where
        \qquad c=P_{1}V_{1}^{\gamma }\\
        Q_{1\to1.5}=-W_{1\to1.5}\\
        Q_{1.5\to2}=C_{V}\Delta T\\
        Q=Q_{1\to1.5}+Q_{1.5\to2}
      \end{gather*}
      \begin{empheq}[box=\widefbox]{gather*}
        W=-1.645\times10^{7}~\unit{ \kilo\joule\per\mole }\\
        Q=1.644\times10^{7}~\unit{ \kilo\joule\per\mole }
      \end{empheq}
    \item
      \begin{gather*}
        T=\frac{PV}{R}\\
        T_{1}=T_{2}=144334.8569~\unit{ \kelvin }\\
        T_{1.5}=1732018.282~\unit{ \kelvin }\\
        Q_{1\to1.5}=C_{V}(T_{1.5}-T_{1})\\
        Q_{1.5\to2}=C_{P}(T_{2}-T_{1.5})\\
        Q=Q_{1\to1.5}+Q_{1.5\to2}\\
        W=-P(V_{2}-V_{1.5})
      \end{gather*}
      \begin{empheq}[box=\widefbox]{gather*}
        Q=-13200~\unit{ \kilo\joule\per\mole }\\
        W=13200~\unit{ \kilo\joule\per\mole }
      \end{empheq}
      It is much easier to do $W=-P(V_{2}-V_{1.5})$ and solving for $Q=-W$.
    \item
      \begin{gather*}
        W=-P(V_{1.5}-V_{1})\\
        Q=-W
      \end{gather*}
      \begin{empheq}[box=\widefbox]{gather*}
        W=-1100~\unit{ \kilo\joule\per\mole }\\
        Q=1100~\unit{ \kilo\joule\per\mole }
      \end{empheq}
  \end{enumerate}
  \begin{figure}[h!]
    \centering
    \scalebox{0.5}{%% Creator: Matplotlib, PGF backend
%%
%% To include the figure in your LaTeX document, write
%%   \input{<filename>.pgf}
%%
%% Make sure the required packages are loaded in your preamble
%%   \usepackage{pgf}
%%
%% Also ensure that all the required font packages are loaded; for instance,
%% the lmodern package is sometimes necessary when using math font.
%%   \usepackage{lmodern}
%%
%% Figures using additional raster images can only be included by \input if
%% they are in the same directory as the main LaTeX file. For loading figures
%% from other directories you can use the `import` package
%%   \usepackage{import}
%%
%% and then include the figures with
%%   \import{<path to file>}{<filename>.pgf}
%%
%% Matplotlib used the following preamble
%%   \def\mathdefault#1{#1}
%%   \everymath=\expandafter{\the\everymath\displaystyle}
%%   \IfFileExists{scrextend.sty}{
%%     \usepackage[fontsize=10.000000pt]{scrextend}
%%   }{
%%     \renewcommand{\normalsize}{\fontsize{10.000000}{12.000000}\selectfont}
%%     \normalsize
%%   }
%%   
%%   \ifdefined\pdftexversion\else  % non-pdftex case.
%%     \usepackage{fontspec}
%%     \setmainfont{DejaVuSerif.ttf}[Path=\detokenize{C:/Users/Jian/AppData/Local/Programs/Python/Python313/Lib/site-packages/matplotlib/mpl-data/fonts/ttf/}]
%%     \setsansfont{DejaVuSans.ttf}[Path=\detokenize{C:/Users/Jian/AppData/Local/Programs/Python/Python313/Lib/site-packages/matplotlib/mpl-data/fonts/ttf/}]
%%     \setmonofont{DejaVuSansMono.ttf}[Path=\detokenize{C:/Users/Jian/AppData/Local/Programs/Python/Python313/Lib/site-packages/matplotlib/mpl-data/fonts/ttf/}]
%%   \fi
%%   \makeatletter\@ifpackageloaded{underscore}{}{\usepackage[strings]{underscore}}\makeatother
%%
\begingroup%
\makeatletter%
\begin{pgfpicture}%
\pgfpathrectangle{\pgfpointorigin}{\pgfqpoint{8.740000in}{6.260000in}}%
\pgfusepath{use as bounding box, clip}%
\begin{pgfscope}%
\pgfsetbuttcap%
\pgfsetmiterjoin%
\definecolor{currentfill}{rgb}{1.000000,1.000000,1.000000}%
\pgfsetfillcolor{currentfill}%
\pgfsetlinewidth{0.000000pt}%
\definecolor{currentstroke}{rgb}{1.000000,1.000000,1.000000}%
\pgfsetstrokecolor{currentstroke}%
\pgfsetdash{}{0pt}%
\pgfpathmoveto{\pgfqpoint{0.000000in}{0.000000in}}%
\pgfpathlineto{\pgfqpoint{8.740000in}{0.000000in}}%
\pgfpathlineto{\pgfqpoint{8.740000in}{6.260000in}}%
\pgfpathlineto{\pgfqpoint{0.000000in}{6.260000in}}%
\pgfpathlineto{\pgfqpoint{0.000000in}{0.000000in}}%
\pgfpathclose%
\pgfusepath{fill}%
\end{pgfscope}%
\begin{pgfscope}%
\pgfsetbuttcap%
\pgfsetmiterjoin%
\definecolor{currentfill}{rgb}{1.000000,1.000000,1.000000}%
\pgfsetfillcolor{currentfill}%
\pgfsetlinewidth{0.000000pt}%
\definecolor{currentstroke}{rgb}{0.000000,0.000000,0.000000}%
\pgfsetstrokecolor{currentstroke}%
\pgfsetstrokeopacity{0.000000}%
\pgfsetdash{}{0pt}%
\pgfpathmoveto{\pgfqpoint{1.092500in}{0.688600in}}%
\pgfpathlineto{\pgfqpoint{7.866000in}{0.688600in}}%
\pgfpathlineto{\pgfqpoint{7.866000in}{5.508800in}}%
\pgfpathlineto{\pgfqpoint{1.092500in}{5.508800in}}%
\pgfpathlineto{\pgfqpoint{1.092500in}{0.688600in}}%
\pgfpathclose%
\pgfusepath{fill}%
\end{pgfscope}%
\begin{pgfscope}%
\pgfpathrectangle{\pgfqpoint{1.092500in}{0.688600in}}{\pgfqpoint{6.773500in}{4.820200in}}%
\pgfusepath{clip}%
\pgfsetbuttcap%
\pgfsetroundjoin%
\pgfsetlinewidth{1.003750pt}%
\definecolor{currentstroke}{rgb}{0.000000,0.000000,0.000000}%
\pgfsetstrokecolor{currentstroke}%
\pgfsetdash{}{0pt}%
\pgfpathmoveto{\pgfqpoint{1.400386in}{5.240595in}}%
\pgfpathcurveto{\pgfqpoint{1.413409in}{5.240595in}}{\pgfqpoint{1.425900in}{5.245769in}}{\pgfqpoint{1.435109in}{5.254978in}}%
\pgfpathcurveto{\pgfqpoint{1.444317in}{5.264186in}}{\pgfqpoint{1.449491in}{5.276677in}}{\pgfqpoint{1.449491in}{5.289700in}}%
\pgfpathcurveto{\pgfqpoint{1.449491in}{5.302723in}}{\pgfqpoint{1.444317in}{5.315214in}}{\pgfqpoint{1.435109in}{5.324422in}}%
\pgfpathcurveto{\pgfqpoint{1.425900in}{5.333631in}}{\pgfqpoint{1.413409in}{5.338805in}}{\pgfqpoint{1.400386in}{5.338805in}}%
\pgfpathcurveto{\pgfqpoint{1.387364in}{5.338805in}}{\pgfqpoint{1.374873in}{5.333631in}}{\pgfqpoint{1.365664in}{5.324422in}}%
\pgfpathcurveto{\pgfqpoint{1.356456in}{5.315214in}}{\pgfqpoint{1.351282in}{5.302723in}}{\pgfqpoint{1.351282in}{5.289700in}}%
\pgfpathcurveto{\pgfqpoint{1.351282in}{5.276677in}}{\pgfqpoint{1.356456in}{5.264186in}}{\pgfqpoint{1.365664in}{5.254978in}}%
\pgfpathcurveto{\pgfqpoint{1.374873in}{5.245769in}}{\pgfqpoint{1.387364in}{5.240595in}}{\pgfqpoint{1.400386in}{5.240595in}}%
\pgfpathlineto{\pgfqpoint{1.400386in}{5.240595in}}%
\pgfpathclose%
\pgfusepath{stroke}%
\end{pgfscope}%
\begin{pgfscope}%
\pgfpathrectangle{\pgfqpoint{1.092500in}{0.688600in}}{\pgfqpoint{6.773500in}{4.820200in}}%
\pgfusepath{clip}%
\pgfsetbuttcap%
\pgfsetroundjoin%
\pgfsetlinewidth{1.003750pt}%
\definecolor{currentstroke}{rgb}{0.000000,0.000000,0.000000}%
\pgfsetstrokecolor{currentstroke}%
\pgfsetdash{}{0pt}%
\pgfpathmoveto{\pgfqpoint{2.494690in}{0.858595in}}%
\pgfpathcurveto{\pgfqpoint{2.507713in}{0.858595in}}{\pgfqpoint{2.520204in}{0.863769in}}{\pgfqpoint{2.529413in}{0.872978in}}%
\pgfpathcurveto{\pgfqpoint{2.538621in}{0.882186in}}{\pgfqpoint{2.543795in}{0.894677in}}{\pgfqpoint{2.543795in}{0.907700in}}%
\pgfpathcurveto{\pgfqpoint{2.543795in}{0.920723in}}{\pgfqpoint{2.538621in}{0.933214in}}{\pgfqpoint{2.529413in}{0.942422in}}%
\pgfpathcurveto{\pgfqpoint{2.520204in}{0.951631in}}{\pgfqpoint{2.507713in}{0.956805in}}{\pgfqpoint{2.494690in}{0.956805in}}%
\pgfpathcurveto{\pgfqpoint{2.481668in}{0.956805in}}{\pgfqpoint{2.469177in}{0.951631in}}{\pgfqpoint{2.459968in}{0.942422in}}%
\pgfpathcurveto{\pgfqpoint{2.450760in}{0.933214in}}{\pgfqpoint{2.445586in}{0.920723in}}{\pgfqpoint{2.445586in}{0.907700in}}%
\pgfpathcurveto{\pgfqpoint{2.445586in}{0.894677in}}{\pgfqpoint{2.450760in}{0.882186in}}{\pgfqpoint{2.459968in}{0.872978in}}%
\pgfpathcurveto{\pgfqpoint{2.469177in}{0.863769in}}{\pgfqpoint{2.481668in}{0.858595in}}{\pgfqpoint{2.494690in}{0.858595in}}%
\pgfpathlineto{\pgfqpoint{2.494690in}{0.858595in}}%
\pgfpathclose%
\pgfusepath{stroke}%
\end{pgfscope}%
\begin{pgfscope}%
\pgfpathrectangle{\pgfqpoint{1.092500in}{0.688600in}}{\pgfqpoint{6.773500in}{4.820200in}}%
\pgfusepath{clip}%
\pgfsetrectcap%
\pgfsetroundjoin%
\pgfsetlinewidth{0.803000pt}%
\definecolor{currentstroke}{rgb}{0.690196,0.690196,0.690196}%
\pgfsetstrokecolor{currentstroke}%
\pgfsetdash{}{0pt}%
\pgfpathmoveto{\pgfqpoint{1.300904in}{0.688600in}}%
\pgfpathlineto{\pgfqpoint{1.300904in}{5.508800in}}%
\pgfusepath{stroke}%
\end{pgfscope}%
\begin{pgfscope}%
\pgfsetbuttcap%
\pgfsetroundjoin%
\definecolor{currentfill}{rgb}{0.000000,0.000000,0.000000}%
\pgfsetfillcolor{currentfill}%
\pgfsetlinewidth{0.803000pt}%
\definecolor{currentstroke}{rgb}{0.000000,0.000000,0.000000}%
\pgfsetstrokecolor{currentstroke}%
\pgfsetdash{}{0pt}%
\pgfsys@defobject{currentmarker}{\pgfqpoint{0.000000in}{-0.048611in}}{\pgfqpoint{0.000000in}{0.000000in}}{%
\pgfpathmoveto{\pgfqpoint{0.000000in}{0.000000in}}%
\pgfpathlineto{\pgfqpoint{0.000000in}{-0.048611in}}%
\pgfusepath{stroke,fill}%
}%
\begin{pgfscope}%
\pgfsys@transformshift{1.300904in}{0.688600in}%
\pgfsys@useobject{currentmarker}{}%
\end{pgfscope}%
\end{pgfscope}%
\begin{pgfscope}%
\definecolor{textcolor}{rgb}{0.000000,0.000000,0.000000}%
\pgfsetstrokecolor{textcolor}%
\pgfsetfillcolor{textcolor}%
\pgftext[x=1.300904in,y=0.591378in,,top]{\color{textcolor}{\sffamily\fontsize{10.000000}{12.000000}\selectfont\catcode`\^=\active\def^{\ifmmode\sp\else\^{}\fi}\catcode`\%=\active\def%{\%}0}}%
\end{pgfscope}%
\begin{pgfscope}%
\pgfpathrectangle{\pgfqpoint{1.092500in}{0.688600in}}{\pgfqpoint{6.773500in}{4.820200in}}%
\pgfusepath{clip}%
\pgfsetrectcap%
\pgfsetroundjoin%
\pgfsetlinewidth{0.803000pt}%
\definecolor{currentstroke}{rgb}{0.690196,0.690196,0.690196}%
\pgfsetstrokecolor{currentstroke}%
\pgfsetdash{}{0pt}%
\pgfpathmoveto{\pgfqpoint{2.295726in}{0.688600in}}%
\pgfpathlineto{\pgfqpoint{2.295726in}{5.508800in}}%
\pgfusepath{stroke}%
\end{pgfscope}%
\begin{pgfscope}%
\pgfsetbuttcap%
\pgfsetroundjoin%
\definecolor{currentfill}{rgb}{0.000000,0.000000,0.000000}%
\pgfsetfillcolor{currentfill}%
\pgfsetlinewidth{0.803000pt}%
\definecolor{currentstroke}{rgb}{0.000000,0.000000,0.000000}%
\pgfsetstrokecolor{currentstroke}%
\pgfsetdash{}{0pt}%
\pgfsys@defobject{currentmarker}{\pgfqpoint{0.000000in}{-0.048611in}}{\pgfqpoint{0.000000in}{0.000000in}}{%
\pgfpathmoveto{\pgfqpoint{0.000000in}{0.000000in}}%
\pgfpathlineto{\pgfqpoint{0.000000in}{-0.048611in}}%
\pgfusepath{stroke,fill}%
}%
\begin{pgfscope}%
\pgfsys@transformshift{2.295726in}{0.688600in}%
\pgfsys@useobject{currentmarker}{}%
\end{pgfscope}%
\end{pgfscope}%
\begin{pgfscope}%
\definecolor{textcolor}{rgb}{0.000000,0.000000,0.000000}%
\pgfsetstrokecolor{textcolor}%
\pgfsetfillcolor{textcolor}%
\pgftext[x=2.295726in,y=0.591378in,,top]{\color{textcolor}{\sffamily\fontsize{10.000000}{12.000000}\selectfont\catcode`\^=\active\def^{\ifmmode\sp\else\^{}\fi}\catcode`\%=\active\def%{\%}10}}%
\end{pgfscope}%
\begin{pgfscope}%
\pgfpathrectangle{\pgfqpoint{1.092500in}{0.688600in}}{\pgfqpoint{6.773500in}{4.820200in}}%
\pgfusepath{clip}%
\pgfsetrectcap%
\pgfsetroundjoin%
\pgfsetlinewidth{0.803000pt}%
\definecolor{currentstroke}{rgb}{0.690196,0.690196,0.690196}%
\pgfsetstrokecolor{currentstroke}%
\pgfsetdash{}{0pt}%
\pgfpathmoveto{\pgfqpoint{3.290548in}{0.688600in}}%
\pgfpathlineto{\pgfqpoint{3.290548in}{5.508800in}}%
\pgfusepath{stroke}%
\end{pgfscope}%
\begin{pgfscope}%
\pgfsetbuttcap%
\pgfsetroundjoin%
\definecolor{currentfill}{rgb}{0.000000,0.000000,0.000000}%
\pgfsetfillcolor{currentfill}%
\pgfsetlinewidth{0.803000pt}%
\definecolor{currentstroke}{rgb}{0.000000,0.000000,0.000000}%
\pgfsetstrokecolor{currentstroke}%
\pgfsetdash{}{0pt}%
\pgfsys@defobject{currentmarker}{\pgfqpoint{0.000000in}{-0.048611in}}{\pgfqpoint{0.000000in}{0.000000in}}{%
\pgfpathmoveto{\pgfqpoint{0.000000in}{0.000000in}}%
\pgfpathlineto{\pgfqpoint{0.000000in}{-0.048611in}}%
\pgfusepath{stroke,fill}%
}%
\begin{pgfscope}%
\pgfsys@transformshift{3.290548in}{0.688600in}%
\pgfsys@useobject{currentmarker}{}%
\end{pgfscope}%
\end{pgfscope}%
\begin{pgfscope}%
\definecolor{textcolor}{rgb}{0.000000,0.000000,0.000000}%
\pgfsetstrokecolor{textcolor}%
\pgfsetfillcolor{textcolor}%
\pgftext[x=3.290548in,y=0.591378in,,top]{\color{textcolor}{\sffamily\fontsize{10.000000}{12.000000}\selectfont\catcode`\^=\active\def^{\ifmmode\sp\else\^{}\fi}\catcode`\%=\active\def%{\%}20}}%
\end{pgfscope}%
\begin{pgfscope}%
\pgfpathrectangle{\pgfqpoint{1.092500in}{0.688600in}}{\pgfqpoint{6.773500in}{4.820200in}}%
\pgfusepath{clip}%
\pgfsetrectcap%
\pgfsetroundjoin%
\pgfsetlinewidth{0.803000pt}%
\definecolor{currentstroke}{rgb}{0.690196,0.690196,0.690196}%
\pgfsetstrokecolor{currentstroke}%
\pgfsetdash{}{0pt}%
\pgfpathmoveto{\pgfqpoint{4.285370in}{0.688600in}}%
\pgfpathlineto{\pgfqpoint{4.285370in}{5.508800in}}%
\pgfusepath{stroke}%
\end{pgfscope}%
\begin{pgfscope}%
\pgfsetbuttcap%
\pgfsetroundjoin%
\definecolor{currentfill}{rgb}{0.000000,0.000000,0.000000}%
\pgfsetfillcolor{currentfill}%
\pgfsetlinewidth{0.803000pt}%
\definecolor{currentstroke}{rgb}{0.000000,0.000000,0.000000}%
\pgfsetstrokecolor{currentstroke}%
\pgfsetdash{}{0pt}%
\pgfsys@defobject{currentmarker}{\pgfqpoint{0.000000in}{-0.048611in}}{\pgfqpoint{0.000000in}{0.000000in}}{%
\pgfpathmoveto{\pgfqpoint{0.000000in}{0.000000in}}%
\pgfpathlineto{\pgfqpoint{0.000000in}{-0.048611in}}%
\pgfusepath{stroke,fill}%
}%
\begin{pgfscope}%
\pgfsys@transformshift{4.285370in}{0.688600in}%
\pgfsys@useobject{currentmarker}{}%
\end{pgfscope}%
\end{pgfscope}%
\begin{pgfscope}%
\definecolor{textcolor}{rgb}{0.000000,0.000000,0.000000}%
\pgfsetstrokecolor{textcolor}%
\pgfsetfillcolor{textcolor}%
\pgftext[x=4.285370in,y=0.591378in,,top]{\color{textcolor}{\sffamily\fontsize{10.000000}{12.000000}\selectfont\catcode`\^=\active\def^{\ifmmode\sp\else\^{}\fi}\catcode`\%=\active\def%{\%}30}}%
\end{pgfscope}%
\begin{pgfscope}%
\pgfpathrectangle{\pgfqpoint{1.092500in}{0.688600in}}{\pgfqpoint{6.773500in}{4.820200in}}%
\pgfusepath{clip}%
\pgfsetrectcap%
\pgfsetroundjoin%
\pgfsetlinewidth{0.803000pt}%
\definecolor{currentstroke}{rgb}{0.690196,0.690196,0.690196}%
\pgfsetstrokecolor{currentstroke}%
\pgfsetdash{}{0pt}%
\pgfpathmoveto{\pgfqpoint{5.280191in}{0.688600in}}%
\pgfpathlineto{\pgfqpoint{5.280191in}{5.508800in}}%
\pgfusepath{stroke}%
\end{pgfscope}%
\begin{pgfscope}%
\pgfsetbuttcap%
\pgfsetroundjoin%
\definecolor{currentfill}{rgb}{0.000000,0.000000,0.000000}%
\pgfsetfillcolor{currentfill}%
\pgfsetlinewidth{0.803000pt}%
\definecolor{currentstroke}{rgb}{0.000000,0.000000,0.000000}%
\pgfsetstrokecolor{currentstroke}%
\pgfsetdash{}{0pt}%
\pgfsys@defobject{currentmarker}{\pgfqpoint{0.000000in}{-0.048611in}}{\pgfqpoint{0.000000in}{0.000000in}}{%
\pgfpathmoveto{\pgfqpoint{0.000000in}{0.000000in}}%
\pgfpathlineto{\pgfqpoint{0.000000in}{-0.048611in}}%
\pgfusepath{stroke,fill}%
}%
\begin{pgfscope}%
\pgfsys@transformshift{5.280191in}{0.688600in}%
\pgfsys@useobject{currentmarker}{}%
\end{pgfscope}%
\end{pgfscope}%
\begin{pgfscope}%
\definecolor{textcolor}{rgb}{0.000000,0.000000,0.000000}%
\pgfsetstrokecolor{textcolor}%
\pgfsetfillcolor{textcolor}%
\pgftext[x=5.280191in,y=0.591378in,,top]{\color{textcolor}{\sffamily\fontsize{10.000000}{12.000000}\selectfont\catcode`\^=\active\def^{\ifmmode\sp\else\^{}\fi}\catcode`\%=\active\def%{\%}40}}%
\end{pgfscope}%
\begin{pgfscope}%
\pgfpathrectangle{\pgfqpoint{1.092500in}{0.688600in}}{\pgfqpoint{6.773500in}{4.820200in}}%
\pgfusepath{clip}%
\pgfsetrectcap%
\pgfsetroundjoin%
\pgfsetlinewidth{0.803000pt}%
\definecolor{currentstroke}{rgb}{0.690196,0.690196,0.690196}%
\pgfsetstrokecolor{currentstroke}%
\pgfsetdash{}{0pt}%
\pgfpathmoveto{\pgfqpoint{6.275013in}{0.688600in}}%
\pgfpathlineto{\pgfqpoint{6.275013in}{5.508800in}}%
\pgfusepath{stroke}%
\end{pgfscope}%
\begin{pgfscope}%
\pgfsetbuttcap%
\pgfsetroundjoin%
\definecolor{currentfill}{rgb}{0.000000,0.000000,0.000000}%
\pgfsetfillcolor{currentfill}%
\pgfsetlinewidth{0.803000pt}%
\definecolor{currentstroke}{rgb}{0.000000,0.000000,0.000000}%
\pgfsetstrokecolor{currentstroke}%
\pgfsetdash{}{0pt}%
\pgfsys@defobject{currentmarker}{\pgfqpoint{0.000000in}{-0.048611in}}{\pgfqpoint{0.000000in}{0.000000in}}{%
\pgfpathmoveto{\pgfqpoint{0.000000in}{0.000000in}}%
\pgfpathlineto{\pgfqpoint{0.000000in}{-0.048611in}}%
\pgfusepath{stroke,fill}%
}%
\begin{pgfscope}%
\pgfsys@transformshift{6.275013in}{0.688600in}%
\pgfsys@useobject{currentmarker}{}%
\end{pgfscope}%
\end{pgfscope}%
\begin{pgfscope}%
\definecolor{textcolor}{rgb}{0.000000,0.000000,0.000000}%
\pgfsetstrokecolor{textcolor}%
\pgfsetfillcolor{textcolor}%
\pgftext[x=6.275013in,y=0.591378in,,top]{\color{textcolor}{\sffamily\fontsize{10.000000}{12.000000}\selectfont\catcode`\^=\active\def^{\ifmmode\sp\else\^{}\fi}\catcode`\%=\active\def%{\%}50}}%
\end{pgfscope}%
\begin{pgfscope}%
\pgfpathrectangle{\pgfqpoint{1.092500in}{0.688600in}}{\pgfqpoint{6.773500in}{4.820200in}}%
\pgfusepath{clip}%
\pgfsetrectcap%
\pgfsetroundjoin%
\pgfsetlinewidth{0.803000pt}%
\definecolor{currentstroke}{rgb}{0.690196,0.690196,0.690196}%
\pgfsetstrokecolor{currentstroke}%
\pgfsetdash{}{0pt}%
\pgfpathmoveto{\pgfqpoint{7.269835in}{0.688600in}}%
\pgfpathlineto{\pgfqpoint{7.269835in}{5.508800in}}%
\pgfusepath{stroke}%
\end{pgfscope}%
\begin{pgfscope}%
\pgfsetbuttcap%
\pgfsetroundjoin%
\definecolor{currentfill}{rgb}{0.000000,0.000000,0.000000}%
\pgfsetfillcolor{currentfill}%
\pgfsetlinewidth{0.803000pt}%
\definecolor{currentstroke}{rgb}{0.000000,0.000000,0.000000}%
\pgfsetstrokecolor{currentstroke}%
\pgfsetdash{}{0pt}%
\pgfsys@defobject{currentmarker}{\pgfqpoint{0.000000in}{-0.048611in}}{\pgfqpoint{0.000000in}{0.000000in}}{%
\pgfpathmoveto{\pgfqpoint{0.000000in}{0.000000in}}%
\pgfpathlineto{\pgfqpoint{0.000000in}{-0.048611in}}%
\pgfusepath{stroke,fill}%
}%
\begin{pgfscope}%
\pgfsys@transformshift{7.269835in}{0.688600in}%
\pgfsys@useobject{currentmarker}{}%
\end{pgfscope}%
\end{pgfscope}%
\begin{pgfscope}%
\definecolor{textcolor}{rgb}{0.000000,0.000000,0.000000}%
\pgfsetstrokecolor{textcolor}%
\pgfsetfillcolor{textcolor}%
\pgftext[x=7.269835in,y=0.591378in,,top]{\color{textcolor}{\sffamily\fontsize{10.000000}{12.000000}\selectfont\catcode`\^=\active\def^{\ifmmode\sp\else\^{}\fi}\catcode`\%=\active\def%{\%}60}}%
\end{pgfscope}%
\begin{pgfscope}%
\definecolor{textcolor}{rgb}{0.000000,0.000000,0.000000}%
\pgfsetstrokecolor{textcolor}%
\pgfsetfillcolor{textcolor}%
\pgftext[x=4.479250in,y=0.401409in,,top]{\color{textcolor}{\sffamily\fontsize{14.000000}{16.800000}\selectfont\catcode`\^=\active\def^{\ifmmode\sp\else\^{}\fi}\catcode`\%=\active\def%{\%}Pressure/$\mathrm{bar}$}}%
\end{pgfscope}%
\begin{pgfscope}%
\pgfpathrectangle{\pgfqpoint{1.092500in}{0.688600in}}{\pgfqpoint{6.773500in}{4.820200in}}%
\pgfusepath{clip}%
\pgfsetrectcap%
\pgfsetroundjoin%
\pgfsetlinewidth{0.803000pt}%
\definecolor{currentstroke}{rgb}{0.690196,0.690196,0.690196}%
\pgfsetstrokecolor{currentstroke}%
\pgfsetdash{}{0pt}%
\pgfpathmoveto{\pgfqpoint{1.092500in}{1.306064in}}%
\pgfpathlineto{\pgfqpoint{7.866000in}{1.306064in}}%
\pgfusepath{stroke}%
\end{pgfscope}%
\begin{pgfscope}%
\pgfsetbuttcap%
\pgfsetroundjoin%
\definecolor{currentfill}{rgb}{0.000000,0.000000,0.000000}%
\pgfsetfillcolor{currentfill}%
\pgfsetlinewidth{0.803000pt}%
\definecolor{currentstroke}{rgb}{0.000000,0.000000,0.000000}%
\pgfsetstrokecolor{currentstroke}%
\pgfsetdash{}{0pt}%
\pgfsys@defobject{currentmarker}{\pgfqpoint{-0.048611in}{0.000000in}}{\pgfqpoint{-0.000000in}{0.000000in}}{%
\pgfpathmoveto{\pgfqpoint{-0.000000in}{0.000000in}}%
\pgfpathlineto{\pgfqpoint{-0.048611in}{0.000000in}}%
\pgfusepath{stroke,fill}%
}%
\begin{pgfscope}%
\pgfsys@transformshift{1.092500in}{1.306064in}%
\pgfsys@useobject{currentmarker}{}%
\end{pgfscope}%
\end{pgfscope}%
\begin{pgfscope}%
\definecolor{textcolor}{rgb}{0.000000,0.000000,0.000000}%
\pgfsetstrokecolor{textcolor}%
\pgfsetfillcolor{textcolor}%
\pgftext[x=0.906913in, y=1.253302in, left, base]{\color{textcolor}{\sffamily\fontsize{10.000000}{12.000000}\selectfont\catcode`\^=\active\def^{\ifmmode\sp\else\^{}\fi}\catcode`\%=\active\def%{\%}2}}%
\end{pgfscope}%
\begin{pgfscope}%
\pgfpathrectangle{\pgfqpoint{1.092500in}{0.688600in}}{\pgfqpoint{6.773500in}{4.820200in}}%
\pgfusepath{clip}%
\pgfsetrectcap%
\pgfsetroundjoin%
\pgfsetlinewidth{0.803000pt}%
\definecolor{currentstroke}{rgb}{0.690196,0.690196,0.690196}%
\pgfsetstrokecolor{currentstroke}%
\pgfsetdash{}{0pt}%
\pgfpathmoveto{\pgfqpoint{1.092500in}{2.102791in}}%
\pgfpathlineto{\pgfqpoint{7.866000in}{2.102791in}}%
\pgfusepath{stroke}%
\end{pgfscope}%
\begin{pgfscope}%
\pgfsetbuttcap%
\pgfsetroundjoin%
\definecolor{currentfill}{rgb}{0.000000,0.000000,0.000000}%
\pgfsetfillcolor{currentfill}%
\pgfsetlinewidth{0.803000pt}%
\definecolor{currentstroke}{rgb}{0.000000,0.000000,0.000000}%
\pgfsetstrokecolor{currentstroke}%
\pgfsetdash{}{0pt}%
\pgfsys@defobject{currentmarker}{\pgfqpoint{-0.048611in}{0.000000in}}{\pgfqpoint{-0.000000in}{0.000000in}}{%
\pgfpathmoveto{\pgfqpoint{-0.000000in}{0.000000in}}%
\pgfpathlineto{\pgfqpoint{-0.048611in}{0.000000in}}%
\pgfusepath{stroke,fill}%
}%
\begin{pgfscope}%
\pgfsys@transformshift{1.092500in}{2.102791in}%
\pgfsys@useobject{currentmarker}{}%
\end{pgfscope}%
\end{pgfscope}%
\begin{pgfscope}%
\definecolor{textcolor}{rgb}{0.000000,0.000000,0.000000}%
\pgfsetstrokecolor{textcolor}%
\pgfsetfillcolor{textcolor}%
\pgftext[x=0.906913in, y=2.050029in, left, base]{\color{textcolor}{\sffamily\fontsize{10.000000}{12.000000}\selectfont\catcode`\^=\active\def^{\ifmmode\sp\else\^{}\fi}\catcode`\%=\active\def%{\%}4}}%
\end{pgfscope}%
\begin{pgfscope}%
\pgfpathrectangle{\pgfqpoint{1.092500in}{0.688600in}}{\pgfqpoint{6.773500in}{4.820200in}}%
\pgfusepath{clip}%
\pgfsetrectcap%
\pgfsetroundjoin%
\pgfsetlinewidth{0.803000pt}%
\definecolor{currentstroke}{rgb}{0.690196,0.690196,0.690196}%
\pgfsetstrokecolor{currentstroke}%
\pgfsetdash{}{0pt}%
\pgfpathmoveto{\pgfqpoint{1.092500in}{2.899518in}}%
\pgfpathlineto{\pgfqpoint{7.866000in}{2.899518in}}%
\pgfusepath{stroke}%
\end{pgfscope}%
\begin{pgfscope}%
\pgfsetbuttcap%
\pgfsetroundjoin%
\definecolor{currentfill}{rgb}{0.000000,0.000000,0.000000}%
\pgfsetfillcolor{currentfill}%
\pgfsetlinewidth{0.803000pt}%
\definecolor{currentstroke}{rgb}{0.000000,0.000000,0.000000}%
\pgfsetstrokecolor{currentstroke}%
\pgfsetdash{}{0pt}%
\pgfsys@defobject{currentmarker}{\pgfqpoint{-0.048611in}{0.000000in}}{\pgfqpoint{-0.000000in}{0.000000in}}{%
\pgfpathmoveto{\pgfqpoint{-0.000000in}{0.000000in}}%
\pgfpathlineto{\pgfqpoint{-0.048611in}{0.000000in}}%
\pgfusepath{stroke,fill}%
}%
\begin{pgfscope}%
\pgfsys@transformshift{1.092500in}{2.899518in}%
\pgfsys@useobject{currentmarker}{}%
\end{pgfscope}%
\end{pgfscope}%
\begin{pgfscope}%
\definecolor{textcolor}{rgb}{0.000000,0.000000,0.000000}%
\pgfsetstrokecolor{textcolor}%
\pgfsetfillcolor{textcolor}%
\pgftext[x=0.906913in, y=2.846757in, left, base]{\color{textcolor}{\sffamily\fontsize{10.000000}{12.000000}\selectfont\catcode`\^=\active\def^{\ifmmode\sp\else\^{}\fi}\catcode`\%=\active\def%{\%}6}}%
\end{pgfscope}%
\begin{pgfscope}%
\pgfpathrectangle{\pgfqpoint{1.092500in}{0.688600in}}{\pgfqpoint{6.773500in}{4.820200in}}%
\pgfusepath{clip}%
\pgfsetrectcap%
\pgfsetroundjoin%
\pgfsetlinewidth{0.803000pt}%
\definecolor{currentstroke}{rgb}{0.690196,0.690196,0.690196}%
\pgfsetstrokecolor{currentstroke}%
\pgfsetdash{}{0pt}%
\pgfpathmoveto{\pgfqpoint{1.092500in}{3.696245in}}%
\pgfpathlineto{\pgfqpoint{7.866000in}{3.696245in}}%
\pgfusepath{stroke}%
\end{pgfscope}%
\begin{pgfscope}%
\pgfsetbuttcap%
\pgfsetroundjoin%
\definecolor{currentfill}{rgb}{0.000000,0.000000,0.000000}%
\pgfsetfillcolor{currentfill}%
\pgfsetlinewidth{0.803000pt}%
\definecolor{currentstroke}{rgb}{0.000000,0.000000,0.000000}%
\pgfsetstrokecolor{currentstroke}%
\pgfsetdash{}{0pt}%
\pgfsys@defobject{currentmarker}{\pgfqpoint{-0.048611in}{0.000000in}}{\pgfqpoint{-0.000000in}{0.000000in}}{%
\pgfpathmoveto{\pgfqpoint{-0.000000in}{0.000000in}}%
\pgfpathlineto{\pgfqpoint{-0.048611in}{0.000000in}}%
\pgfusepath{stroke,fill}%
}%
\begin{pgfscope}%
\pgfsys@transformshift{1.092500in}{3.696245in}%
\pgfsys@useobject{currentmarker}{}%
\end{pgfscope}%
\end{pgfscope}%
\begin{pgfscope}%
\definecolor{textcolor}{rgb}{0.000000,0.000000,0.000000}%
\pgfsetstrokecolor{textcolor}%
\pgfsetfillcolor{textcolor}%
\pgftext[x=0.906913in, y=3.643484in, left, base]{\color{textcolor}{\sffamily\fontsize{10.000000}{12.000000}\selectfont\catcode`\^=\active\def^{\ifmmode\sp\else\^{}\fi}\catcode`\%=\active\def%{\%}8}}%
\end{pgfscope}%
\begin{pgfscope}%
\pgfpathrectangle{\pgfqpoint{1.092500in}{0.688600in}}{\pgfqpoint{6.773500in}{4.820200in}}%
\pgfusepath{clip}%
\pgfsetrectcap%
\pgfsetroundjoin%
\pgfsetlinewidth{0.803000pt}%
\definecolor{currentstroke}{rgb}{0.690196,0.690196,0.690196}%
\pgfsetstrokecolor{currentstroke}%
\pgfsetdash{}{0pt}%
\pgfpathmoveto{\pgfqpoint{1.092500in}{4.492973in}}%
\pgfpathlineto{\pgfqpoint{7.866000in}{4.492973in}}%
\pgfusepath{stroke}%
\end{pgfscope}%
\begin{pgfscope}%
\pgfsetbuttcap%
\pgfsetroundjoin%
\definecolor{currentfill}{rgb}{0.000000,0.000000,0.000000}%
\pgfsetfillcolor{currentfill}%
\pgfsetlinewidth{0.803000pt}%
\definecolor{currentstroke}{rgb}{0.000000,0.000000,0.000000}%
\pgfsetstrokecolor{currentstroke}%
\pgfsetdash{}{0pt}%
\pgfsys@defobject{currentmarker}{\pgfqpoint{-0.048611in}{0.000000in}}{\pgfqpoint{-0.000000in}{0.000000in}}{%
\pgfpathmoveto{\pgfqpoint{-0.000000in}{0.000000in}}%
\pgfpathlineto{\pgfqpoint{-0.048611in}{0.000000in}}%
\pgfusepath{stroke,fill}%
}%
\begin{pgfscope}%
\pgfsys@transformshift{1.092500in}{4.492973in}%
\pgfsys@useobject{currentmarker}{}%
\end{pgfscope}%
\end{pgfscope}%
\begin{pgfscope}%
\definecolor{textcolor}{rgb}{0.000000,0.000000,0.000000}%
\pgfsetstrokecolor{textcolor}%
\pgfsetfillcolor{textcolor}%
\pgftext[x=0.818547in, y=4.440211in, left, base]{\color{textcolor}{\sffamily\fontsize{10.000000}{12.000000}\selectfont\catcode`\^=\active\def^{\ifmmode\sp\else\^{}\fi}\catcode`\%=\active\def%{\%}10}}%
\end{pgfscope}%
\begin{pgfscope}%
\pgfpathrectangle{\pgfqpoint{1.092500in}{0.688600in}}{\pgfqpoint{6.773500in}{4.820200in}}%
\pgfusepath{clip}%
\pgfsetrectcap%
\pgfsetroundjoin%
\pgfsetlinewidth{0.803000pt}%
\definecolor{currentstroke}{rgb}{0.690196,0.690196,0.690196}%
\pgfsetstrokecolor{currentstroke}%
\pgfsetdash{}{0pt}%
\pgfpathmoveto{\pgfqpoint{1.092500in}{5.289700in}}%
\pgfpathlineto{\pgfqpoint{7.866000in}{5.289700in}}%
\pgfusepath{stroke}%
\end{pgfscope}%
\begin{pgfscope}%
\pgfsetbuttcap%
\pgfsetroundjoin%
\definecolor{currentfill}{rgb}{0.000000,0.000000,0.000000}%
\pgfsetfillcolor{currentfill}%
\pgfsetlinewidth{0.803000pt}%
\definecolor{currentstroke}{rgb}{0.000000,0.000000,0.000000}%
\pgfsetstrokecolor{currentstroke}%
\pgfsetdash{}{0pt}%
\pgfsys@defobject{currentmarker}{\pgfqpoint{-0.048611in}{0.000000in}}{\pgfqpoint{-0.000000in}{0.000000in}}{%
\pgfpathmoveto{\pgfqpoint{-0.000000in}{0.000000in}}%
\pgfpathlineto{\pgfqpoint{-0.048611in}{0.000000in}}%
\pgfusepath{stroke,fill}%
}%
\begin{pgfscope}%
\pgfsys@transformshift{1.092500in}{5.289700in}%
\pgfsys@useobject{currentmarker}{}%
\end{pgfscope}%
\end{pgfscope}%
\begin{pgfscope}%
\definecolor{textcolor}{rgb}{0.000000,0.000000,0.000000}%
\pgfsetstrokecolor{textcolor}%
\pgfsetfillcolor{textcolor}%
\pgftext[x=0.818547in, y=5.236938in, left, base]{\color{textcolor}{\sffamily\fontsize{10.000000}{12.000000}\selectfont\catcode`\^=\active\def^{\ifmmode\sp\else\^{}\fi}\catcode`\%=\active\def%{\%}12}}%
\end{pgfscope}%
\begin{pgfscope}%
\definecolor{textcolor}{rgb}{0.000000,0.000000,0.000000}%
\pgfsetstrokecolor{textcolor}%
\pgfsetfillcolor{textcolor}%
\pgftext[x=0.762992in,y=3.098700in,,bottom,rotate=90.000000]{\color{textcolor}{\sffamily\fontsize{14.000000}{16.800000}\selectfont\catcode`\^=\active\def^{\ifmmode\sp\else\^{}\fi}\catcode`\%=\active\def%{\%}Volume/$\mathrm{m}^3\mathrm{mol}^{-1}$}}%
\end{pgfscope}%
\begin{pgfscope}%
\pgfpathrectangle{\pgfqpoint{1.092500in}{0.688600in}}{\pgfqpoint{6.773500in}{4.820200in}}%
\pgfusepath{clip}%
\pgfsetrectcap%
\pgfsetroundjoin%
\pgfsetlinewidth{3.513125pt}%
\definecolor{currentstroke}{rgb}{0.000000,0.000000,1.000000}%
\pgfsetstrokecolor{currentstroke}%
\pgfsetstrokeopacity{0.850000}%
\pgfsetdash{}{0pt}%
\pgfpathmoveto{\pgfqpoint{1.400386in}{5.289700in}}%
\pgfpathlineto{\pgfqpoint{1.411440in}{4.811664in}}%
\pgfpathlineto{\pgfqpoint{1.422494in}{4.420543in}}%
\pgfpathlineto{\pgfqpoint{1.433547in}{4.094609in}}%
\pgfpathlineto{\pgfqpoint{1.444601in}{3.818819in}}%
\pgfpathlineto{\pgfqpoint{1.455654in}{3.582427in}}%
\pgfpathlineto{\pgfqpoint{1.466708in}{3.377555in}}%
\pgfpathlineto{\pgfqpoint{1.477761in}{3.198291in}}%
\pgfpathlineto{\pgfqpoint{1.488815in}{3.040117in}}%
\pgfpathlineto{\pgfqpoint{1.499869in}{2.899518in}}%
\pgfpathlineto{\pgfqpoint{1.510922in}{2.773719in}}%
\pgfpathlineto{\pgfqpoint{1.521976in}{2.660500in}}%
\pgfpathlineto{\pgfqpoint{1.533029in}{2.558064in}}%
\pgfpathlineto{\pgfqpoint{1.544083in}{2.464940in}}%
\pgfpathlineto{\pgfqpoint{1.555136in}{2.379913in}}%
\pgfpathlineto{\pgfqpoint{1.566190in}{2.301973in}}%
\pgfpathlineto{\pgfqpoint{1.577244in}{2.230267in}}%
\pgfpathlineto{\pgfqpoint{1.588297in}{2.164078in}}%
\pgfpathlineto{\pgfqpoint{1.599351in}{2.102791in}}%
\pgfpathlineto{\pgfqpoint{1.610404in}{2.045882in}}%
\pgfpathlineto{\pgfqpoint{1.621458in}{1.992897in}}%
\pgfpathlineto{\pgfqpoint{1.632511in}{1.943445in}}%
\pgfpathlineto{\pgfqpoint{1.643565in}{1.897184in}}%
\pgfpathlineto{\pgfqpoint{1.654619in}{1.853814in}}%
\pgfpathlineto{\pgfqpoint{1.665672in}{1.813072in}}%
\pgfpathlineto{\pgfqpoint{1.676726in}{1.774727in}}%
\pgfpathlineto{\pgfqpoint{1.687779in}{1.738573in}}%
\pgfpathlineto{\pgfqpoint{1.698833in}{1.704427in}}%
\pgfpathlineto{\pgfqpoint{1.709886in}{1.672128in}}%
\pgfpathlineto{\pgfqpoint{1.720940in}{1.641528in}}%
\pgfpathlineto{\pgfqpoint{1.731994in}{1.612497in}}%
\pgfpathlineto{\pgfqpoint{1.743047in}{1.584918in}}%
\pgfpathlineto{\pgfqpoint{1.754101in}{1.558684in}}%
\pgfpathlineto{\pgfqpoint{1.765154in}{1.533700in}}%
\pgfpathlineto{\pgfqpoint{1.776208in}{1.509878in}}%
\pgfpathlineto{\pgfqpoint{1.787261in}{1.487138in}}%
\pgfpathlineto{\pgfqpoint{1.798315in}{1.465409in}}%
\pgfpathlineto{\pgfqpoint{1.809369in}{1.444625in}}%
\pgfpathlineto{\pgfqpoint{1.820422in}{1.424725in}}%
\pgfpathlineto{\pgfqpoint{1.831476in}{1.405655in}}%
\pgfpathlineto{\pgfqpoint{1.842529in}{1.387362in}}%
\pgfpathlineto{\pgfqpoint{1.853583in}{1.369802in}}%
\pgfpathlineto{\pgfqpoint{1.864637in}{1.352930in}}%
\pgfpathlineto{\pgfqpoint{1.875690in}{1.336707in}}%
\pgfpathlineto{\pgfqpoint{1.886744in}{1.321096in}}%
\pgfpathlineto{\pgfqpoint{1.897797in}{1.306064in}}%
\pgfpathlineto{\pgfqpoint{1.908851in}{1.291578in}}%
\pgfpathlineto{\pgfqpoint{1.919904in}{1.277609in}}%
\pgfpathlineto{\pgfqpoint{1.930958in}{1.264131in}}%
\pgfpathlineto{\pgfqpoint{1.942012in}{1.251117in}}%
\pgfpathlineto{\pgfqpoint{1.953065in}{1.238544in}}%
\pgfpathlineto{\pgfqpoint{1.964119in}{1.226391in}}%
\pgfpathlineto{\pgfqpoint{1.975172in}{1.214636in}}%
\pgfpathlineto{\pgfqpoint{1.986226in}{1.203260in}}%
\pgfpathlineto{\pgfqpoint{1.997279in}{1.192245in}}%
\pgfpathlineto{\pgfqpoint{2.008333in}{1.181575in}}%
\pgfpathlineto{\pgfqpoint{2.019387in}{1.171233in}}%
\pgfpathlineto{\pgfqpoint{2.030440in}{1.161204in}}%
\pgfpathlineto{\pgfqpoint{2.041494in}{1.151475in}}%
\pgfpathlineto{\pgfqpoint{2.052547in}{1.142032in}}%
\pgfpathlineto{\pgfqpoint{2.063601in}{1.132862in}}%
\pgfpathlineto{\pgfqpoint{2.074654in}{1.123955in}}%
\pgfpathlineto{\pgfqpoint{2.085708in}{1.115298in}}%
\pgfpathlineto{\pgfqpoint{2.096762in}{1.106882in}}%
\pgfpathlineto{\pgfqpoint{2.107815in}{1.098696in}}%
\pgfpathlineto{\pgfqpoint{2.118869in}{1.090732in}}%
\pgfpathlineto{\pgfqpoint{2.129922in}{1.082980in}}%
\pgfpathlineto{\pgfqpoint{2.140976in}{1.075432in}}%
\pgfpathlineto{\pgfqpoint{2.152029in}{1.068080in}}%
\pgfpathlineto{\pgfqpoint{2.163083in}{1.060917in}}%
\pgfpathlineto{\pgfqpoint{2.174137in}{1.053935in}}%
\pgfpathlineto{\pgfqpoint{2.185190in}{1.047127in}}%
\pgfpathlineto{\pgfqpoint{2.196244in}{1.040488in}}%
\pgfpathlineto{\pgfqpoint{2.207297in}{1.034010in}}%
\pgfpathlineto{\pgfqpoint{2.218351in}{1.027689in}}%
\pgfpathlineto{\pgfqpoint{2.229405in}{1.021518in}}%
\pgfpathlineto{\pgfqpoint{2.240458in}{1.015493in}}%
\pgfpathlineto{\pgfqpoint{2.251512in}{1.009607in}}%
\pgfpathlineto{\pgfqpoint{2.262565in}{1.003857in}}%
\pgfpathlineto{\pgfqpoint{2.273619in}{0.998237in}}%
\pgfpathlineto{\pgfqpoint{2.284672in}{0.992744in}}%
\pgfpathlineto{\pgfqpoint{2.295726in}{0.987373in}}%
\pgfpathlineto{\pgfqpoint{2.306780in}{0.982120in}}%
\pgfpathlineto{\pgfqpoint{2.317833in}{0.976981in}}%
\pgfpathlineto{\pgfqpoint{2.328887in}{0.971952in}}%
\pgfpathlineto{\pgfqpoint{2.339940in}{0.967031in}}%
\pgfpathlineto{\pgfqpoint{2.350994in}{0.962213in}}%
\pgfpathlineto{\pgfqpoint{2.362047in}{0.957495in}}%
\pgfpathlineto{\pgfqpoint{2.373101in}{0.952875in}}%
\pgfpathlineto{\pgfqpoint{2.384155in}{0.948349in}}%
\pgfpathlineto{\pgfqpoint{2.395208in}{0.943915in}}%
\pgfpathlineto{\pgfqpoint{2.406262in}{0.939569in}}%
\pgfpathlineto{\pgfqpoint{2.417315in}{0.935309in}}%
\pgfpathlineto{\pgfqpoint{2.428369in}{0.931133in}}%
\pgfpathlineto{\pgfqpoint{2.439422in}{0.927038in}}%
\pgfpathlineto{\pgfqpoint{2.450476in}{0.923022in}}%
\pgfpathlineto{\pgfqpoint{2.461530in}{0.919082in}}%
\pgfpathlineto{\pgfqpoint{2.472583in}{0.915216in}}%
\pgfpathlineto{\pgfqpoint{2.483637in}{0.911423in}}%
\pgfpathlineto{\pgfqpoint{2.494690in}{0.907700in}}%
\pgfusepath{stroke}%
\end{pgfscope}%
\begin{pgfscope}%
\pgfpathrectangle{\pgfqpoint{1.092500in}{0.688600in}}{\pgfqpoint{6.773500in}{4.820200in}}%
\pgfusepath{clip}%
\pgfsetbuttcap%
\pgfsetroundjoin%
\pgfsetlinewidth{3.513125pt}%
\definecolor{currentstroke}{rgb}{1.000000,0.647059,0.000000}%
\pgfsetstrokecolor{currentstroke}%
\pgfsetstrokeopacity{0.850000}%
\pgfsetdash{{12.950000pt}{5.600000pt}}{0.000000pt}%
\pgfpathmoveto{\pgfqpoint{1.400386in}{5.289700in}}%
\pgfpathlineto{\pgfqpoint{1.411440in}{4.996857in}}%
\pgfpathlineto{\pgfqpoint{1.422494in}{4.747433in}}%
\pgfpathlineto{\pgfqpoint{1.433547in}{4.531852in}}%
\pgfpathlineto{\pgfqpoint{1.444601in}{4.343233in}}%
\pgfpathlineto{\pgfqpoint{1.455654in}{4.176494in}}%
\pgfpathlineto{\pgfqpoint{1.466708in}{4.027789in}}%
\pgfpathlineto{\pgfqpoint{1.477761in}{3.894148in}}%
\pgfpathlineto{\pgfqpoint{1.488815in}{3.773238in}}%
\pgfpathlineto{\pgfqpoint{1.499869in}{3.663200in}}%
\pgfpathlineto{\pgfqpoint{1.510922in}{3.562530in}}%
\pgfpathlineto{\pgfqpoint{1.521976in}{3.469996in}}%
\pgfpathlineto{\pgfqpoint{1.533029in}{3.384581in}}%
\pgfpathlineto{\pgfqpoint{1.544083in}{3.305437in}}%
\pgfpathlineto{\pgfqpoint{1.555136in}{3.231848in}}%
\pgfpathlineto{\pgfqpoint{1.566190in}{3.163207in}}%
\pgfpathlineto{\pgfqpoint{1.577244in}{3.098994in}}%
\pgfpathlineto{\pgfqpoint{1.588297in}{3.038765in}}%
\pgfpathlineto{\pgfqpoint{1.599351in}{2.982132in}}%
\pgfpathlineto{\pgfqpoint{1.610404in}{2.928758in}}%
\pgfpathlineto{\pgfqpoint{1.621458in}{2.878350in}}%
\pgfpathlineto{\pgfqpoint{1.632511in}{2.830649in}}%
\pgfpathlineto{\pgfqpoint{1.643565in}{2.785426in}}%
\pgfpathlineto{\pgfqpoint{1.654619in}{2.742479in}}%
\pgfpathlineto{\pgfqpoint{1.665672in}{2.701627in}}%
\pgfpathlineto{\pgfqpoint{1.676726in}{2.662709in}}%
\pgfpathlineto{\pgfqpoint{1.687779in}{2.625580in}}%
\pgfpathlineto{\pgfqpoint{1.698833in}{2.590110in}}%
\pgfpathlineto{\pgfqpoint{1.709886in}{2.556183in}}%
\pgfpathlineto{\pgfqpoint{1.720940in}{2.523693in}}%
\pgfpathlineto{\pgfqpoint{1.731994in}{2.492542in}}%
\pgfpathlineto{\pgfqpoint{1.743047in}{2.462643in}}%
\pgfpathlineto{\pgfqpoint{1.754101in}{2.433917in}}%
\pgfpathlineto{\pgfqpoint{1.765154in}{2.406291in}}%
\pgfpathlineto{\pgfqpoint{1.776208in}{2.379697in}}%
\pgfpathlineto{\pgfqpoint{1.787261in}{2.354075in}}%
\pgfpathlineto{\pgfqpoint{1.798315in}{2.329368in}}%
\pgfpathlineto{\pgfqpoint{1.809369in}{2.305524in}}%
\pgfpathlineto{\pgfqpoint{1.820422in}{2.282496in}}%
\pgfpathlineto{\pgfqpoint{1.831476in}{2.260238in}}%
\pgfpathlineto{\pgfqpoint{1.842529in}{2.238710in}}%
\pgfpathlineto{\pgfqpoint{1.853583in}{2.217874in}}%
\pgfpathlineto{\pgfqpoint{1.864637in}{2.197694in}}%
\pgfpathlineto{\pgfqpoint{1.875690in}{2.178137in}}%
\pgfpathlineto{\pgfqpoint{1.886744in}{2.159173in}}%
\pgfpathlineto{\pgfqpoint{1.897797in}{2.140773in}}%
\pgfpathlineto{\pgfqpoint{1.908851in}{2.122910in}}%
\pgfpathlineto{\pgfqpoint{1.919904in}{2.105560in}}%
\pgfpathlineto{\pgfqpoint{1.930958in}{2.088698in}}%
\pgfpathlineto{\pgfqpoint{1.942012in}{2.072303in}}%
\pgfpathlineto{\pgfqpoint{1.953065in}{2.056354in}}%
\pgfpathlineto{\pgfqpoint{1.964119in}{2.040832in}}%
\pgfpathlineto{\pgfqpoint{1.975172in}{2.025718in}}%
\pgfpathlineto{\pgfqpoint{1.986226in}{2.010996in}}%
\pgfpathlineto{\pgfqpoint{1.997279in}{1.996648in}}%
\pgfpathlineto{\pgfqpoint{2.008333in}{1.982661in}}%
\pgfpathlineto{\pgfqpoint{2.019387in}{1.969019in}}%
\pgfpathlineto{\pgfqpoint{2.030440in}{1.955709in}}%
\pgfpathlineto{\pgfqpoint{2.041494in}{1.942717in}}%
\pgfpathlineto{\pgfqpoint{2.052547in}{1.930032in}}%
\pgfpathlineto{\pgfqpoint{2.063601in}{1.917642in}}%
\pgfpathlineto{\pgfqpoint{2.074654in}{1.905536in}}%
\pgfpathlineto{\pgfqpoint{2.085708in}{1.893704in}}%
\pgfpathlineto{\pgfqpoint{2.096762in}{1.882135in}}%
\pgfpathlineto{\pgfqpoint{2.107815in}{1.870821in}}%
\pgfpathlineto{\pgfqpoint{2.118869in}{1.859752in}}%
\pgfpathlineto{\pgfqpoint{2.129922in}{1.848920in}}%
\pgfpathlineto{\pgfqpoint{2.140976in}{1.838316in}}%
\pgfpathlineto{\pgfqpoint{2.152029in}{1.827933in}}%
\pgfpathlineto{\pgfqpoint{2.163083in}{1.817764in}}%
\pgfpathlineto{\pgfqpoint{2.174137in}{1.807801in}}%
\pgfpathlineto{\pgfqpoint{2.185190in}{1.798038in}}%
\pgfpathlineto{\pgfqpoint{2.196244in}{1.788469in}}%
\pgfpathlineto{\pgfqpoint{2.207297in}{1.779086in}}%
\pgfpathlineto{\pgfqpoint{2.218351in}{1.769885in}}%
\pgfpathlineto{\pgfqpoint{2.229405in}{1.760859in}}%
\pgfpathlineto{\pgfqpoint{2.240458in}{1.752004in}}%
\pgfpathlineto{\pgfqpoint{2.251512in}{1.743314in}}%
\pgfpathlineto{\pgfqpoint{2.262565in}{1.734784in}}%
\pgfpathlineto{\pgfqpoint{2.273619in}{1.726410in}}%
\pgfpathlineto{\pgfqpoint{2.284672in}{1.718186in}}%
\pgfpathlineto{\pgfqpoint{2.295726in}{1.710109in}}%
\pgfpathlineto{\pgfqpoint{2.306780in}{1.702175in}}%
\pgfpathlineto{\pgfqpoint{2.317833in}{1.694378in}}%
\pgfpathlineto{\pgfqpoint{2.328887in}{1.686716in}}%
\pgfpathlineto{\pgfqpoint{2.339940in}{1.679185in}}%
\pgfpathlineto{\pgfqpoint{2.350994in}{1.671781in}}%
\pgfpathlineto{\pgfqpoint{2.362047in}{1.664501in}}%
\pgfpathlineto{\pgfqpoint{2.373101in}{1.657340in}}%
\pgfpathlineto{\pgfqpoint{2.384155in}{1.650297in}}%
\pgfpathlineto{\pgfqpoint{2.395208in}{1.643368in}}%
\pgfpathlineto{\pgfqpoint{2.406262in}{1.636551in}}%
\pgfpathlineto{\pgfqpoint{2.417315in}{1.629841in}}%
\pgfpathlineto{\pgfqpoint{2.428369in}{1.623237in}}%
\pgfpathlineto{\pgfqpoint{2.439422in}{1.616735in}}%
\pgfpathlineto{\pgfqpoint{2.450476in}{1.610334in}}%
\pgfpathlineto{\pgfqpoint{2.461530in}{1.604031in}}%
\pgfpathlineto{\pgfqpoint{2.472583in}{1.597823in}}%
\pgfpathlineto{\pgfqpoint{2.483637in}{1.591707in}}%
\pgfpathlineto{\pgfqpoint{2.494690in}{1.585683in}}%
\pgfpathlineto{\pgfqpoint{2.494690in}{1.585683in}}%
\pgfpathlineto{\pgfqpoint{2.494690in}{0.907700in}}%
\pgfusepath{stroke}%
\end{pgfscope}%
\begin{pgfscope}%
\pgfpathrectangle{\pgfqpoint{1.092500in}{0.688600in}}{\pgfqpoint{6.773500in}{4.820200in}}%
\pgfusepath{clip}%
\pgfsetbuttcap%
\pgfsetroundjoin%
\pgfsetlinewidth{3.513125pt}%
\definecolor{currentstroke}{rgb}{0.000000,0.501961,0.000000}%
\pgfsetstrokecolor{currentstroke}%
\pgfsetstrokeopacity{0.850000}%
\pgfsetdash{{3.500000pt}{5.775000pt}}{0.000000pt}%
\pgfpathmoveto{\pgfqpoint{1.400386in}{5.289700in}}%
\pgfpathlineto{\pgfqpoint{1.462586in}{4.081341in}}%
\pgfpathlineto{\pgfqpoint{1.524785in}{3.447650in}}%
\pgfpathlineto{\pgfqpoint{1.586984in}{3.045724in}}%
\pgfpathlineto{\pgfqpoint{1.649183in}{2.763324in}}%
\pgfpathlineto{\pgfqpoint{1.711383in}{2.551704in}}%
\pgfpathlineto{\pgfqpoint{1.773582in}{2.385925in}}%
\pgfpathlineto{\pgfqpoint{1.835781in}{2.251768in}}%
\pgfpathlineto{\pgfqpoint{1.897980in}{2.140472in}}%
\pgfpathlineto{\pgfqpoint{1.960180in}{2.046315in}}%
\pgfpathlineto{\pgfqpoint{2.022379in}{1.965383in}}%
\pgfpathlineto{\pgfqpoint{2.084578in}{1.894901in}}%
\pgfpathlineto{\pgfqpoint{2.146778in}{1.832839in}}%
\pgfpathlineto{\pgfqpoint{2.208977in}{1.777677in}}%
\pgfpathlineto{\pgfqpoint{2.271176in}{1.728247in}}%
\pgfpathlineto{\pgfqpoint{2.333375in}{1.683643in}}%
\pgfpathlineto{\pgfqpoint{2.395575in}{1.643141in}}%
\pgfpathlineto{\pgfqpoint{2.457774in}{1.606162in}}%
\pgfpathlineto{\pgfqpoint{2.519973in}{1.572233in}}%
\pgfpathlineto{\pgfqpoint{2.582172in}{1.540967in}}%
\pgfpathlineto{\pgfqpoint{2.644372in}{1.512038in}}%
\pgfpathlineto{\pgfqpoint{2.706571in}{1.485177in}}%
\pgfpathlineto{\pgfqpoint{2.768770in}{1.460152in}}%
\pgfpathlineto{\pgfqpoint{2.830969in}{1.436768in}}%
\pgfpathlineto{\pgfqpoint{2.893169in}{1.414858in}}%
\pgfpathlineto{\pgfqpoint{2.955368in}{1.394276in}}%
\pgfpathlineto{\pgfqpoint{3.017567in}{1.374896in}}%
\pgfpathlineto{\pgfqpoint{3.079767in}{1.356608in}}%
\pgfpathlineto{\pgfqpoint{3.141966in}{1.339315in}}%
\pgfpathlineto{\pgfqpoint{3.204165in}{1.322933in}}%
\pgfpathlineto{\pgfqpoint{3.266364in}{1.307385in}}%
\pgfpathlineto{\pgfqpoint{3.328564in}{1.292606in}}%
\pgfpathlineto{\pgfqpoint{3.390763in}{1.278534in}}%
\pgfpathlineto{\pgfqpoint{3.452962in}{1.265117in}}%
\pgfpathlineto{\pgfqpoint{3.515161in}{1.252306in}}%
\pgfpathlineto{\pgfqpoint{3.577361in}{1.240059in}}%
\pgfpathlineto{\pgfqpoint{3.639560in}{1.228335in}}%
\pgfpathlineto{\pgfqpoint{3.701759in}{1.217100in}}%
\pgfpathlineto{\pgfqpoint{3.763958in}{1.206322in}}%
\pgfpathlineto{\pgfqpoint{3.826158in}{1.195970in}}%
\pgfpathlineto{\pgfqpoint{3.888357in}{1.186018in}}%
\pgfpathlineto{\pgfqpoint{3.950556in}{1.176442in}}%
\pgfpathlineto{\pgfqpoint{4.012756in}{1.167219in}}%
\pgfpathlineto{\pgfqpoint{4.074955in}{1.158329in}}%
\pgfpathlineto{\pgfqpoint{4.137154in}{1.149751in}}%
\pgfpathlineto{\pgfqpoint{4.199353in}{1.141470in}}%
\pgfpathlineto{\pgfqpoint{4.261553in}{1.133468in}}%
\pgfpathlineto{\pgfqpoint{4.323752in}{1.125730in}}%
\pgfpathlineto{\pgfqpoint{4.385951in}{1.118243in}}%
\pgfpathlineto{\pgfqpoint{4.448150in}{1.110994in}}%
\pgfpathlineto{\pgfqpoint{4.510350in}{1.103971in}}%
\pgfpathlineto{\pgfqpoint{4.572549in}{1.097162in}}%
\pgfpathlineto{\pgfqpoint{4.634748in}{1.090557in}}%
\pgfpathlineto{\pgfqpoint{4.696947in}{1.084146in}}%
\pgfpathlineto{\pgfqpoint{4.759147in}{1.077921in}}%
\pgfpathlineto{\pgfqpoint{4.821346in}{1.071872in}}%
\pgfpathlineto{\pgfqpoint{4.883545in}{1.065991in}}%
\pgfpathlineto{\pgfqpoint{4.945744in}{1.060272in}}%
\pgfpathlineto{\pgfqpoint{5.007944in}{1.054707in}}%
\pgfpathlineto{\pgfqpoint{5.070143in}{1.049289in}}%
\pgfpathlineto{\pgfqpoint{5.132342in}{1.044013in}}%
\pgfpathlineto{\pgfqpoint{5.194542in}{1.038872in}}%
\pgfpathlineto{\pgfqpoint{5.256741in}{1.033860in}}%
\pgfpathlineto{\pgfqpoint{5.318940in}{1.028973in}}%
\pgfpathlineto{\pgfqpoint{5.381139in}{1.024206in}}%
\pgfpathlineto{\pgfqpoint{5.443339in}{1.019553in}}%
\pgfpathlineto{\pgfqpoint{5.505538in}{1.015011in}}%
\pgfpathlineto{\pgfqpoint{5.567737in}{1.010575in}}%
\pgfpathlineto{\pgfqpoint{5.629936in}{1.006242in}}%
\pgfpathlineto{\pgfqpoint{5.692136in}{1.002007in}}%
\pgfpathlineto{\pgfqpoint{5.754335in}{0.997866in}}%
\pgfpathlineto{\pgfqpoint{5.816534in}{0.993818in}}%
\pgfpathlineto{\pgfqpoint{5.878733in}{0.989857in}}%
\pgfpathlineto{\pgfqpoint{5.940933in}{0.985982in}}%
\pgfpathlineto{\pgfqpoint{6.003132in}{0.982189in}}%
\pgfpathlineto{\pgfqpoint{6.065331in}{0.978476in}}%
\pgfpathlineto{\pgfqpoint{6.127531in}{0.974839in}}%
\pgfpathlineto{\pgfqpoint{6.189730in}{0.971276in}}%
\pgfpathlineto{\pgfqpoint{6.251929in}{0.967785in}}%
\pgfpathlineto{\pgfqpoint{6.314128in}{0.964364in}}%
\pgfpathlineto{\pgfqpoint{6.376328in}{0.961010in}}%
\pgfpathlineto{\pgfqpoint{6.438527in}{0.957721in}}%
\pgfpathlineto{\pgfqpoint{6.500726in}{0.954495in}}%
\pgfpathlineto{\pgfqpoint{6.562925in}{0.951331in}}%
\pgfpathlineto{\pgfqpoint{6.625125in}{0.948225in}}%
\pgfpathlineto{\pgfqpoint{6.687324in}{0.945177in}}%
\pgfpathlineto{\pgfqpoint{6.749523in}{0.942185in}}%
\pgfpathlineto{\pgfqpoint{6.811722in}{0.939247in}}%
\pgfpathlineto{\pgfqpoint{6.873922in}{0.936362in}}%
\pgfpathlineto{\pgfqpoint{6.936121in}{0.933528in}}%
\pgfpathlineto{\pgfqpoint{6.998320in}{0.930743in}}%
\pgfpathlineto{\pgfqpoint{7.060520in}{0.928007in}}%
\pgfpathlineto{\pgfqpoint{7.122719in}{0.925317in}}%
\pgfpathlineto{\pgfqpoint{7.184918in}{0.922673in}}%
\pgfpathlineto{\pgfqpoint{7.247117in}{0.920073in}}%
\pgfpathlineto{\pgfqpoint{7.309317in}{0.917517in}}%
\pgfpathlineto{\pgfqpoint{7.371516in}{0.915002in}}%
\pgfpathlineto{\pgfqpoint{7.433715in}{0.912529in}}%
\pgfpathlineto{\pgfqpoint{7.495914in}{0.910095in}}%
\pgfpathlineto{\pgfqpoint{7.558114in}{0.907700in}}%
\pgfpathlineto{\pgfqpoint{7.558114in}{0.907700in}}%
\pgfpathlineto{\pgfqpoint{2.494690in}{0.907700in}}%
\pgfusepath{stroke}%
\end{pgfscope}%
\begin{pgfscope}%
\pgfpathrectangle{\pgfqpoint{1.092500in}{0.688600in}}{\pgfqpoint{6.773500in}{4.820200in}}%
\pgfusepath{clip}%
\pgfsetbuttcap%
\pgfsetroundjoin%
\pgfsetlinewidth{3.513125pt}%
\definecolor{currentstroke}{rgb}{1.000000,0.000000,0.000000}%
\pgfsetstrokecolor{currentstroke}%
\pgfsetstrokeopacity{0.850000}%
\pgfsetdash{{22.400000pt}{5.600000pt}{3.500000pt}{5.600000pt}}{0.000000pt}%
\pgfpathmoveto{\pgfqpoint{1.400386in}{5.289700in}}%
\pgfpathlineto{\pgfqpoint{2.494690in}{5.289700in}}%
\pgfpathlineto{\pgfqpoint{2.494690in}{0.907700in}}%
\pgfusepath{stroke}%
\end{pgfscope}%
\begin{pgfscope}%
\pgfpathrectangle{\pgfqpoint{1.092500in}{0.688600in}}{\pgfqpoint{6.773500in}{4.820200in}}%
\pgfusepath{clip}%
\pgfsetrectcap%
\pgfsetroundjoin%
\pgfsetlinewidth{3.513125pt}%
\definecolor{currentstroke}{rgb}{0.501961,0.000000,0.501961}%
\pgfsetstrokecolor{currentstroke}%
\pgfsetstrokeopacity{0.850000}%
\pgfsetdash{}{0pt}%
\pgfpathmoveto{\pgfqpoint{1.400386in}{5.289700in}}%
\pgfpathlineto{\pgfqpoint{1.400386in}{0.907700in}}%
\pgfpathlineto{\pgfqpoint{2.494690in}{0.907700in}}%
\pgfusepath{stroke}%
\end{pgfscope}%
\begin{pgfscope}%
\pgfsetrectcap%
\pgfsetmiterjoin%
\pgfsetlinewidth{0.803000pt}%
\definecolor{currentstroke}{rgb}{0.000000,0.000000,0.000000}%
\pgfsetstrokecolor{currentstroke}%
\pgfsetdash{}{0pt}%
\pgfpathmoveto{\pgfqpoint{1.092500in}{0.688600in}}%
\pgfpathlineto{\pgfqpoint{1.092500in}{5.508800in}}%
\pgfusepath{stroke}%
\end{pgfscope}%
\begin{pgfscope}%
\pgfsetrectcap%
\pgfsetmiterjoin%
\pgfsetlinewidth{0.803000pt}%
\definecolor{currentstroke}{rgb}{0.000000,0.000000,0.000000}%
\pgfsetstrokecolor{currentstroke}%
\pgfsetdash{}{0pt}%
\pgfpathmoveto{\pgfqpoint{7.866000in}{0.688600in}}%
\pgfpathlineto{\pgfqpoint{7.866000in}{5.508800in}}%
\pgfusepath{stroke}%
\end{pgfscope}%
\begin{pgfscope}%
\pgfsetrectcap%
\pgfsetmiterjoin%
\pgfsetlinewidth{0.803000pt}%
\definecolor{currentstroke}{rgb}{0.000000,0.000000,0.000000}%
\pgfsetstrokecolor{currentstroke}%
\pgfsetdash{}{0pt}%
\pgfpathmoveto{\pgfqpoint{1.092500in}{0.688600in}}%
\pgfpathlineto{\pgfqpoint{7.866000in}{0.688600in}}%
\pgfusepath{stroke}%
\end{pgfscope}%
\begin{pgfscope}%
\pgfsetrectcap%
\pgfsetmiterjoin%
\pgfsetlinewidth{0.803000pt}%
\definecolor{currentstroke}{rgb}{0.000000,0.000000,0.000000}%
\pgfsetstrokecolor{currentstroke}%
\pgfsetdash{}{0pt}%
\pgfpathmoveto{\pgfqpoint{1.092500in}{5.508800in}}%
\pgfpathlineto{\pgfqpoint{7.866000in}{5.508800in}}%
\pgfusepath{stroke}%
\end{pgfscope}%
\begin{pgfscope}%
\definecolor{textcolor}{rgb}{0.000000,0.000000,0.000000}%
\pgfsetstrokecolor{textcolor}%
\pgfsetfillcolor{textcolor}%
\pgftext[x=1.400386in,y=5.369373in,left,base]{\color{textcolor}{\rmfamily\fontsize{14.000000}{16.800000}\selectfont\catcode`\^=\active\def^{\ifmmode\sp\else\^{}\fi}\catcode`\%=\active\def%{\%}State 1}}%
\end{pgfscope}%
\begin{pgfscope}%
\definecolor{textcolor}{rgb}{0.000000,0.000000,0.000000}%
\pgfsetstrokecolor{textcolor}%
\pgfsetfillcolor{textcolor}%
\pgftext[x=2.494690in,y=0.768273in,left,base]{\color{textcolor}{\rmfamily\fontsize{14.000000}{16.800000}\selectfont\catcode`\^=\active\def^{\ifmmode\sp\else\^{}\fi}\catcode`\%=\active\def%{\%}State 2}}%
\end{pgfscope}%
\begin{pgfscope}%
\pgfsetbuttcap%
\pgfsetmiterjoin%
\definecolor{currentfill}{rgb}{1.000000,1.000000,1.000000}%
\pgfsetfillcolor{currentfill}%
\pgfsetfillopacity{0.800000}%
\pgfsetlinewidth{1.003750pt}%
\definecolor{currentstroke}{rgb}{0.800000,0.800000,0.800000}%
\pgfsetstrokecolor{currentstroke}%
\pgfsetstrokeopacity{0.800000}%
\pgfsetdash{}{0pt}%
\pgfpathmoveto{\pgfqpoint{6.875996in}{4.378403in}}%
\pgfpathlineto{\pgfqpoint{7.768778in}{4.378403in}}%
\pgfpathquadraticcurveto{\pgfqpoint{7.796556in}{4.378403in}}{\pgfqpoint{7.796556in}{4.406181in}}%
\pgfpathlineto{\pgfqpoint{7.796556in}{5.411578in}}%
\pgfpathquadraticcurveto{\pgfqpoint{7.796556in}{5.439356in}}{\pgfqpoint{7.768778in}{5.439356in}}%
\pgfpathlineto{\pgfqpoint{6.875996in}{5.439356in}}%
\pgfpathquadraticcurveto{\pgfqpoint{6.848218in}{5.439356in}}{\pgfqpoint{6.848218in}{5.411578in}}%
\pgfpathlineto{\pgfqpoint{6.848218in}{4.406181in}}%
\pgfpathquadraticcurveto{\pgfqpoint{6.848218in}{4.378403in}}{\pgfqpoint{6.875996in}{4.378403in}}%
\pgfpathlineto{\pgfqpoint{6.875996in}{4.378403in}}%
\pgfpathclose%
\pgfusepath{stroke,fill}%
\end{pgfscope}%
\begin{pgfscope}%
\pgfsetrectcap%
\pgfsetroundjoin%
\pgfsetlinewidth{3.513125pt}%
\definecolor{currentstroke}{rgb}{0.000000,0.000000,1.000000}%
\pgfsetstrokecolor{currentstroke}%
\pgfsetstrokeopacity{0.850000}%
\pgfsetdash{}{0pt}%
\pgfpathmoveto{\pgfqpoint{6.903774in}{5.326888in}}%
\pgfpathlineto{\pgfqpoint{7.042663in}{5.326888in}}%
\pgfpathlineto{\pgfqpoint{7.181552in}{5.326888in}}%
\pgfusepath{stroke}%
\end{pgfscope}%
\begin{pgfscope}%
\definecolor{textcolor}{rgb}{0.000000,0.000000,0.000000}%
\pgfsetstrokecolor{textcolor}%
\pgfsetfillcolor{textcolor}%
\pgftext[x=7.292663in,y=5.278277in,left,base]{\color{textcolor}{\rmfamily\fontsize{10.000000}{12.000000}\selectfont\catcode`\^=\active\def^{\ifmmode\sp\else\^{}\fi}\catcode`\%=\active\def%{\%}Path a}}%
\end{pgfscope}%
\begin{pgfscope}%
\pgfsetbuttcap%
\pgfsetroundjoin%
\pgfsetlinewidth{3.513125pt}%
\definecolor{currentstroke}{rgb}{1.000000,0.647059,0.000000}%
\pgfsetstrokecolor{currentstroke}%
\pgfsetstrokeopacity{0.850000}%
\pgfsetdash{{12.950000pt}{5.600000pt}}{0.000000pt}%
\pgfpathmoveto{\pgfqpoint{6.903774in}{5.123031in}}%
\pgfpathlineto{\pgfqpoint{7.042663in}{5.123031in}}%
\pgfpathlineto{\pgfqpoint{7.181552in}{5.123031in}}%
\pgfusepath{stroke}%
\end{pgfscope}%
\begin{pgfscope}%
\definecolor{textcolor}{rgb}{0.000000,0.000000,0.000000}%
\pgfsetstrokecolor{textcolor}%
\pgfsetfillcolor{textcolor}%
\pgftext[x=7.292663in,y=5.074420in,left,base]{\color{textcolor}{\rmfamily\fontsize{10.000000}{12.000000}\selectfont\catcode`\^=\active\def^{\ifmmode\sp\else\^{}\fi}\catcode`\%=\active\def%{\%}Path b}}%
\end{pgfscope}%
\begin{pgfscope}%
\pgfsetbuttcap%
\pgfsetroundjoin%
\pgfsetlinewidth{3.513125pt}%
\definecolor{currentstroke}{rgb}{0.000000,0.501961,0.000000}%
\pgfsetstrokecolor{currentstroke}%
\pgfsetstrokeopacity{0.850000}%
\pgfsetdash{{3.500000pt}{5.775000pt}}{0.000000pt}%
\pgfpathmoveto{\pgfqpoint{6.903774in}{4.919174in}}%
\pgfpathlineto{\pgfqpoint{7.042663in}{4.919174in}}%
\pgfpathlineto{\pgfqpoint{7.181552in}{4.919174in}}%
\pgfusepath{stroke}%
\end{pgfscope}%
\begin{pgfscope}%
\definecolor{textcolor}{rgb}{0.000000,0.000000,0.000000}%
\pgfsetstrokecolor{textcolor}%
\pgfsetfillcolor{textcolor}%
\pgftext[x=7.292663in,y=4.870563in,left,base]{\color{textcolor}{\rmfamily\fontsize{10.000000}{12.000000}\selectfont\catcode`\^=\active\def^{\ifmmode\sp\else\^{}\fi}\catcode`\%=\active\def%{\%}Path c}}%
\end{pgfscope}%
\begin{pgfscope}%
\pgfsetbuttcap%
\pgfsetroundjoin%
\pgfsetlinewidth{3.513125pt}%
\definecolor{currentstroke}{rgb}{1.000000,0.000000,0.000000}%
\pgfsetstrokecolor{currentstroke}%
\pgfsetstrokeopacity{0.850000}%
\pgfsetdash{{22.400000pt}{5.600000pt}{3.500000pt}{5.600000pt}}{0.000000pt}%
\pgfpathmoveto{\pgfqpoint{6.903774in}{4.715316in}}%
\pgfpathlineto{\pgfqpoint{7.042663in}{4.715316in}}%
\pgfpathlineto{\pgfqpoint{7.181552in}{4.715316in}}%
\pgfusepath{stroke}%
\end{pgfscope}%
\begin{pgfscope}%
\definecolor{textcolor}{rgb}{0.000000,0.000000,0.000000}%
\pgfsetstrokecolor{textcolor}%
\pgfsetfillcolor{textcolor}%
\pgftext[x=7.292663in,y=4.666705in,left,base]{\color{textcolor}{\rmfamily\fontsize{10.000000}{12.000000}\selectfont\catcode`\^=\active\def^{\ifmmode\sp\else\^{}\fi}\catcode`\%=\active\def%{\%}Path d}}%
\end{pgfscope}%
\begin{pgfscope}%
\pgfsetrectcap%
\pgfsetroundjoin%
\pgfsetlinewidth{3.513125pt}%
\definecolor{currentstroke}{rgb}{0.501961,0.000000,0.501961}%
\pgfsetstrokecolor{currentstroke}%
\pgfsetstrokeopacity{0.850000}%
\pgfsetdash{}{0pt}%
\pgfpathmoveto{\pgfqpoint{6.903774in}{4.511459in}}%
\pgfpathlineto{\pgfqpoint{7.042663in}{4.511459in}}%
\pgfpathlineto{\pgfqpoint{7.181552in}{4.511459in}}%
\pgfusepath{stroke}%
\end{pgfscope}%
\begin{pgfscope}%
\definecolor{textcolor}{rgb}{0.000000,0.000000,0.000000}%
\pgfsetstrokecolor{textcolor}%
\pgfsetfillcolor{textcolor}%
\pgftext[x=7.292663in,y=4.462848in,left,base]{\color{textcolor}{\rmfamily\fontsize{10.000000}{12.000000}\selectfont\catcode`\^=\active\def^{\ifmmode\sp\else\^{}\fi}\catcode`\%=\active\def%{\%}Path e}}%
\end{pgfscope}%
\end{pgfpicture}%
\makeatother%
\endgroup%
}
    \caption{
      PV graph of the five different paths as described in
      Problem 3.21.
    }
    \label{fig:s21}
  \end{figure}
\end{solution}

\end{document}
